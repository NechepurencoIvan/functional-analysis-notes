\documentclass[main]{subfiles}

\begin{document}

\section{Компактные операторы}

Исторически, понятие компактности оператора
появилось под названием «вполне непрерывность».
Оператор \( A : E_1 \to E_2 \)
назывался вполне непрерывным, если
под его действием любая слабо сходящаяся последовательность
начинала сходиться по норме.
Современное определение ушло от понятия последовательности
и аналогично определению ограниченного оператора
как оператора переводящего ограниченное множество в ограниченное.

\begin{definition}
  Оператор \( A \in \Linears{E_1, E_2} \)
  называется \emph{компактным}, если
  для любого ограниченного множества \( X \subset E_1 \)
  его образ \( A X \) предкомпактен.
  Множество компактных операторов, действующих
  из \( E_1 \) в \( E_2 \), будем обозначать как
  \( C(E_1, E_2) \) или, если \( E_1 = E_2 = E \),
  \( C(E) \).
\end{definition}

\begin{remark}
  В линейных нормированных пространствах
  предкомпактность эквивалентна вполне ограниченности;
  однако, в определении мы говорим о предкомпактности —
  это делает его переносимым на топологические линейные пространства.
\end{remark}

\begin{exercise}
  Если \( A \overline{B}(0, 1) \) "--- предкомпакт,
  то \( A \) компактен.
\end{exercise}

\begin{example}
  Если \( \dim E < \infty \), то все операторы из
  \( \Linears{E} \) компактны.
\end{example}

\begin{example}
  Если \( \dim E = \infty \), то \( I_E \) не является компактным.
\end{example}

\begin{example}
  Рассмотрим \( E = C[0, 1] \), определим \( A : E \to E \):
  \[
    (Af)(x) = \int_0^x f(t) dt.
  \]
  Благодаря теореме Арцелла-Асколли достаточно показать, что
  \( A \overline{B}(0, 1) \) ограниченно и обладает
  свойством равностепенной непрерывности.
  Ограниченность оператора очевидна:
  \[
    |(Af)(x)| = |\int_0^x f(t) dt| \le
    \int_0^x |f(t)| dt \le x \cdot ||f|| \le ||f||,
  \]
  а потому \( ||Af|| \le ||f|| \le 1 \).
  Кроме того,
  \[
    |(Af)(x) - (Af)(y)| =
    \left|\int_y^x f(t) dt \right| \le |x - y| \cdot ||f||,
  \]
  а значит, в определении равностепенной непрерывности
  мы можем брать \( \delta = \epsilon \).
  Итак, \( A \) "--- компактен.
\end{example}

\begin{exercise}
  \( C(E) \) "--- «идеал» в \( \Linears{E} \),
  т. е. если \( A \in C(E) \) и \( B \in \Linears{E} \),
  то \( AB, BA \in C(E) \).
\end{exercise}

\begin{corollary}
  В бесконечномерном пространстве компактный оператор
  не имеет ограниченного обратного оператора.
\end{corollary}
\begin{proof}
  Пусть \( E \) "--- ЛНП, \( \dim E = \infty \), \( A \in C(E) \).
  Тогда если \( A^{-1} \in \Linears{E} \), то
  \( I = A A^{-1} \in C(E) \), что неверно.
\end{proof}

\begin{theorem}
  Пусть \( E_1 \) и \( E_2 \) "--- линейные нормированные пространства,
  \( \{ A_n \}_{n=1}^\infty \subset \Linears{E_1, E_2} \) "---
  последовательность компактных операторов и \( A_n \to A \).
  Тогда \( A \) также компактен.
\end{theorem}
\begin{proof}
  Т. к. \( A_n \to A \), для произвольного \( \epsilon > 0 \)
  найдётся \( N \) такое, что
  для произовльного \( x \in \overline{B}(0, 1) \)
  \( \Forall{n \ge N} ||Ax - A_nx|| \le \epsilon \),
  ведь \( ||Ax - A_n x|| \le ||A - A_n|| \cdot ||x|| \le
  ||A - A_n|| \to 0 \).

  Построить 2epsilon сеть.
\end{proof}

\begin{theorem}
  Пусть \( E(\Complex) \) "--- ЛНП, \( A \in C(E) \).
  Тогда \( \Forall{\lambda \ne 0} \dim \Ker A_\lambda < \infty \)
  (т. е. собственное пространство, соответствующее \( \lambda \),
  конечномерно).
\end{theorem}
\begin{proof}
  Воспользуемя теоремой Рисса (т 4.1).
  Пусть \( M = \Ker A_\lambda \) (это подпространство!),
  выберем последовательность \( \{ x_n \} \subset M \) такую,
  что \(  ||x_n|| = 1 \).
  ж
\end{proof}

\begin{theorem}%13.3
  Пусть \( E(\Complex) \) "--- БП, \( A \in C(E) \).
  Тогда для любого \( \delta > 0 \) вне круга
  \( \{ |\lambda| \le \delta \} \) может быть
  только конечное число собственных значений оператора \( A \).
\end{theorem}
\begin{proof}[Доказательство для гильбертова пространства и самосопряжённого оператора]
  Предположим противное: существует бесконечнае последовательность
  собственных значений \( \{ \lambda_n \} \)
  таких, что \( |\lambda_n| > \delta \) для любого \( n \).
  Выберем для каждого собственный вектор \( x_n \),
  при том такой, что \( ||x_n|| = 1 \).
  В таком случае, в \( \{ A x_n \} \)
  существует сходящаяся подпоследовательность
  \( \{ A x_{n_k} \} \).
  Но, поскольку у ССО
  собственные вектора соответствующие различным СЗ
  ортогональны,
  \[
    ||A x_{n_{k_1}} - A x_{n_{k_2}}||^2 =
    ||\lambda_{n_{k_1}} x_{n_{k_1}} - \lambda_{n_{k_2}} x_{n_{k_2}}||^2 =
    |\lambda_{n_{k_1}}|^2 + |\lambda_{n_{k_2}}|^2 > 2 \delta^2,
  \]
  что невозможно для сходящейся подпоследовательности.
\end{proof}

\begin{corollary}
  \[
    \Forall{\delta > 0} \sum_{|\lambda| > \delta} \dim \Ker A_\lambda < \infty
  \]
\end{corollary}

\begin{corollary}
  Если \( A \in C(E) \), то \( \sigma_P(A) \) "---
  не более, чем счётное множество.
\end{corollary}

\begin{problem}
  Если \( A \in C(E_1, E_2) \),
  а последовательность \( \{ x_n \} \subset E_1 \)
  слабо сходится к \( x \),
  то \( A x_n \to A x \) (по норме \( E_2 \)).
\end{problem}

%Линал
%Ax = y разрешима \( \oTTo \) y ортогонален всем
%решениям \( A^T z = 0 \) (\( A^* z = 0 \) на безкоординатном языке).

\begin{lemma}
  Пусть \( H \) "--- комплексное гильбертово пространство,
  \( A \in \Linears{H} \) "--- компактный самосопряжённый оператор.
  Тогда если \( \lambda \ne 0 \) "--- точка спектра \( A \),
  то \( \lambda \) "--- собственное значение \( A \).
\end{lemma}
\begin{proof}
  По т. 12.3 \( \lambda \in \sigma(A) \) \(\oTTo\) 
  существует последовательность \( \{ x_n \} \)
  такая, что \( ||x_n|| = 1 \) и
  \( ||(A - \lambda I) x_n|| \to 0 \).
  \( \{ A x_n \} \) "--- предкомкакт,
  найдётся сходящаяся подпоследовательность:
  \( A x_{n_k} \to y \);
  кроме того, конечно,
  \( A x_{n_k} - \lambda x_{n_k} \to 0 \)ю
  Поскольку \( \lambda \ne 0 \),
  \( \frac1\lambda A x_{n_k} - x_{n_k} \to 0 \),
  и тогда \( x_{n_k} - \frac1\lambda y \to 0 \).
\end{proof}

\begin{lemma}%3
  \( A \) "--- ССО в \( H(C) \),
  подпостранство \( M \) в \( H \)
  инавриантно относительно \( A \),
  т. е. \( AM \subset M \).
  Тогда \( M^\perp \) тоже инвариантно
  относительно \( A \).
\end{lemma}
\begin{proof}
  \( \Forall{y \in M^\perp} \Forall{x \in M}
  \Inner{x, y} = 0 \); при этом,
  \( Ay \in M^\perp \oTTo \Inner{x, Ay} = 0 \),
  \( \Inner{x, Ay} = \Inner{Ax, y} = 0 \).
\end{proof}

\begin{lemma}%2
  В тех же условиях, если \( \lambda \ne 0 \),
  то \( \overline{\Img A_\lambda} = \Img A_\lambda \).
\end{lemma}
\begin{proof}
  По т. 12.2 \( \overline{\Img A_\lambda} \oplus \Ker A_\lambda = H \).
  \( \Ker A_\lambda \) "--- инвариантно отонсительно \( A \),
  а потому и \( \overline{\Img A_\lambda} \).
  Рассмотрим \( \tilde A = A \bigr|_{\overline{A_\lambda}} \);
  он также компактный и самосопряжённый.
  Поскольку мы исключили \( \Ker A_\lambda \),
  \( \lambda \) не может быть собственным значением \( \tilde A \),
  а значит, по лемме 1, \( \lambda \in \rho(\tilde A) \).
  Тогда, конечно, образ оператора \( \tilde A \) равен
  его области определения "--- \( \overline{\Img A_\lambda} \).
  Наконец,
  \[
    \Img A_\lambda \subset
    \overline{\Img A_\lambda} =
    \Img \tilde A \subset
    \Img A_\lambda. \qedhere
  \]
\end{proof}

\begin{theorem}[Фредгольма]
  
\end{theorem}
\begin{proof}
  Если \( \lambda \) не является собственны значением \( A \),
  то \( \Ker A_\lambda = \{ 0 \} \),
  \( \lambda \in \rho(A) \), \( \Img A_\lambda = H \).
  Значит,
\end{proof}

\begin{lemma}
  Пусть \( H \) "--- комплексное гильбертово пространство,
  \( A \in \Linears{H} \) "--- КССО, \( A \ne 0 \).
  Тогда у \( A \) существует ненулевое собственное значение.
\end{lemma}

\end{document}
