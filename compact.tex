\documentclass[main]{subfiles}

\begin{document}
\section{Компактность}

\begin{definition}
  Топологическое пространство \( X \) компактно,
  если для любого набора открытых множеств
  \( {\{ G_\alpha \}}_{\alpha \in A} \),
  являющегося покрытием
  (т. е. такого, что \( X = \bigcup_\alpha G_\alpha \))
  существует конечное подпокрытие, т. е.
  \( \Exists{\alpha_1, \dots, \alpha_n \in A}
  X = G_{\alpha_1} \cup \dots \cup G_{\alpha_n} \).
\end{definition}

\begin{exercise}
  Если \( X \) "--- компактное топологическое пространство,
  \( Y \) "--- топологическое пространство,
  и \( f : X \to Y \) непрерывно,
  то \( f(X) \) "--- компакт.
\end{exercise}

\begin{exercise}
  Докажите, что интервал \( (0, 1) \) гомеоморфен \( \Real \),
  но не отрезку \( [0, 1] \).
\end{exercise}

\begin{definition}
  В топологическом пространстве \( X \)
  \emph{центрированной системой}
  называется такое семейство его подмножеств
  \( {\{ B_\alpha \}}_{\alpha \in A} \),
  что любой конечный набор из них имеет непустое пересечение,
  т. е.
  \[
    \Forall{\alpha_1, \dots, \alpha_n \in A}
    \bigcap_{k = 1}^n B_{\alpha_k} \ne \emptyset.
  \]
\end{definition}

\begin{theorem}\label{thm:compact-centered}
  Топологическое пространство \( X \) компактно
  \( \oTTo \)
  в \( X \) любая центрированная система замкнутых
  множеств имеет непустое пересечение.
\end{theorem}
\begin{proof}
  Пусть \( X \) "--- компакт. Рассмотрим
  произвольную центрированную систему замкнутых множеств
  \( {\{ F_\alpha \}}_{\alpha \in A} \) и обозначим
  \( G_\alpha = X \setminus F_\alpha \).

  Заметим, что из \( \{ G_\alpha \} \) нельзя выбрать
  конечное покрытие. Действительно, для произвольных
  \( \alpha_1, \dots, \alpha_n \in A \)
  \[
    X \setminus \left( \bigcup_{k=1}^n G_{\alpha_k} \right) =
    \bigcap_{k = 1}^n (X \setminus G_{\alpha_k}) =
    \bigcap_{k = 1}^n F_{\alpha_k} \ne \emptyset,
  \]
  т. к. \( \{ F_\alpha \} \) центрирована.

  В таком случае, по определению компактности \( \{ G_\alpha \} \)
  не может быть покрытием и
  \[
    \emptyset \ne
    X \setminus \left( \bigcup_{\alpha \in A } G_{\alpha} \right) =
    \bigcap_{\alpha \in A} F_{\alpha}.
  \]

  Доказательство в обратную сторону остаётся в качестве упражнения.
\end{proof}

\begin{definition}
  Пусть \( X \) "--- метрическое пространство, \( A \subset X \).
  \( B \subset X \) называется \emph{\(\epsilon\)-сетью для \( A \)},
  если
  \[
    A \subset \bigcup_{x \in B} \Cl{B}(x, \epsilon).
  \]
\end{definition}

\begin{definition}
  Подмножество метрического пространства называется
  \emph{вполне ограниченным}, если для любого
  \( \epsilon > 0 \) для него найдётся конечная \( \epsilon \)-сеть.
\end{definition}

\begin{theorem}\label{thm:compact}
  Если \( X \) "--- метрическое пространство,
  то следующие условия эквивалентны:
  \begin{enumerate}
    \item \( X \) "--- компакт
    \item \( X \) "--- полное и вполне ограниченное
    \item Из любой последовательности в \( X \)
      можно выделить сходящуюся подпоследовательность
    \item Всякое бесконечное множество имеет предельную точку
  \end{enumerate}
\end{theorem}
\begin{itemproof}
\item[$1 \To 2$]
  Сначала докажем полноту.
  Выберем произвольную фундаментальную последовательность
  \( \{ x_n \} \) и рассмотрим её "<хвосты">:
  \( A_n = \{ x_n, x_{n+1}, \dots \} \).
  Очевидно, \( \{ \Cl{A}_n \} \) "---
  центрированная система замкнутых множеств
  (ведь \( \Cl{A}_{n+1} \subset \Cl{A}_n \)).
  Тогда по теореме~\ref{thm:compact-centered}
  \( \bigcap \overline{A_n} \ne \emptyset \);
  выберем \( x_0 \in \bigcap \overline{A_n} \) и покажем,
  что \( x_n \to x_0 \).
  Т. к. \( \{ x_n \} \) фундаментальна, 
  \( \Forall{\epsilon > 0} \Exists{N} \Forall{n, m \ge N}
  \rho(x_n, x_m) < \epsilon \).
  Тогда при фиксированном \( \epsilon \) выберем \( n \ge N \),
  и получим, что \( A_n \subset B(x_n, \epsilon) \To
  \Cl{A}_n \subset \Cl{B(x_n, \epsilon)}
  \subset \Cl{B}(x_n, \epsilon) \). В частности,
  \( x_0 \in \Cl{B}(x_n, \epsilon) \To \rho(x_n, x_0) \le \epsilon \),
  т. е. \( x_n \to x_0 \).

  Теперь докажем вполне ограниченность \( X \).
  Очевидно, для произвольного \( \epsilon > 0 \)
  \[ X = \bigcup_{x \in X} B(x, \epsilon). \]
  Значит, это покрытие открытыми множествами,
  и мы можем выбрать конечное
  подпокрытие \( B(x_1, \epsilon), \dots, B(x_n, \epsilon) \),
  а тогда \( \{ x_1, \dots, x_n \} \)
  будет конечной \( \epsilon \)-сетью для \( X \).
\item[$2 \To 3$]
  Зафиксируем произвольную последовательность \( \{ x_n \} \).
  С помощью вполне ограниченности \( X \) выберем фундаментальную
  подпоследовательность, а из полноты мы получим её сходимость.

  Для \( \epsilon > 0 \), выберем
  \( \epsilon \)-сеть \( B(z_1, \epsilon), \dots, B(z_k, \epsilon) \).
  В последовательности \( \{ x_n \} \) бесконечно много элементов,
  а потому в одном из шаров будет также находиться
  бесконечное число элементов, и они образуют подпоследовательность.
  Взяв \( \epsilon = 1 \), получим последовательность
  \( x_n^{(1)} \); из неё выделим таким же образом
  подпоследовательность \( x_n^{(2)} \) для \( \epsilon = \frac{1}{2} \),
  и продолжим этот процесс на \( n \)-ом шаге выбирая
  \( \epsilon = \frac{1}{n} \).
  Заметим: \( x_n^{(n)} \) и \( x_{n+p}^{(n+p)} \)
  лежат в одном шаре радиуса \( \frac{1}{n} \),
  а потому \( \rho(x_n^{(n)}, x_{n+p}^{(n+p)}) < \frac{2}{n} \).
  Отсюда получаем, что \( \{ x_n^{(n)} \} \) фундаментальна,
  а потому сходится.
\item[$3 \To 1$]
  Для начала, покажем, что \( X \) вполне ограниченно.
  Пусть это не так, т. е. для некоторого \( \epsilon_0 \)
  никакое конечное множество не является \( \epsilon_0 \)-сетью.
  Выберем произвольную точку \( x_1 \in X \);
  \( \{ x_1 \} \) "--- не \( \epsilon_0 \)-сеть,
  поэтому \( \Exists{x_2 \in X} \rho(x_1, x_2) > \epsilon_0 \).
  Но \( \{ x_1, x_2 \} \) также не является \( \epsilon_0 \)-сетью,
  а потому \( \Exists{x_3 \in X} \rho(x_1, x_3), \rho(x_2, x_3) > \epsilon_0 \).
  Продолжая этот процесс, построим последовательность \( \{ x_n \} \)
  такую, что \( \Forall{i \ne j} \rho(x_i, x_j) \ge \epsilon_0 \).
  Ясно, что из неё нельзя выделить сходящуюся подпоследовательность,
  что противоречит предположению, а потому \( X \) "--- вполне ограниченно.

  Предположим теперь, что \( X \) не компактно:
  из покрытия \( \{ G_\alpha \} \) нельзя выбрать конечное
  подпокрытие. Для произвольного \( \epsilon > 0 \)
  можем выбрать конечную \( \epsilon \)-сеть,
  и тогда некоторый шар \( B(x_\epsilon, \epsilon) \)
  (свойство вполне ограниченности останется тем же,
  если в определении \( \epsilon \)-сети заменить
  замкнутые шары на открытые)
  нельзя покрыть конечным числом множеств из \( \{ G_\alpha \} \).
  Положим \( x_n \) центром такого шара при \( \epsilon = \frac{1}{n} \).
  Тогда мы можем выбрать подпоследовательность
  \( x_{n_k} \to x_0 \). Кроме того, \( \Exists{\alpha_0} x_0 \in G_{\alpha_0} \).
  Это множетсво октрыто, а потому \( \exists B(x_0) \subset G_{\alpha_0} \).
  Очевидно, для некоторого \( k_0 \)
  \( B(x_{n_{k_0}}, \frac{1}{n_{k_0}}) \subset B(x_0)
  \subset G_{\alpha_0} \),
  т. е. \( G_{\alpha_0} \) покрывает
  \( B(x_{n_{k_0}}, \frac{1}{n_{k_0}}) \), что невозможно по построению.
  Значит, предположение неверно и \( X \) компактно.
\item[$3 \To 4$]
  Пусть \( M \subset X \) бесконечно, тогда
  в нём можно выбрать последовательность
  \( \{ x_n \} \), в которой элементы различны.
  Значит, у неё есть подпоследовательность с пределом
  \( x_0 \), и любая окрестность \( x_0 \) будет
  содержать бесконечный "<хвост"> этой подпоследовательности,
  а потому \( x_0 \) "--- предельная точка \( M \).
\item[$4 \To 3$]
  Зафиксируем произвольную последовательность \( \{ x_n \} \).
  Если множество её значений конечно, то некоторый
  элемент \( x_0 \) встречается в \( \{ x_n \} \)
  бесконечное число раз и мы можем выбрать
  подпоследовательность \( \{ x_{n_k} \} \)
  такую, что \( x_{n_k} = x_0 \),
  которая, конечно, сходится к \( x_0 \).
  Иначе, у множества значений \( \{ x_n \} \)
  есть предельная точка \( x_0 \)
  и мы можем построить подпоследовательность,
  сходящуюся к \( x_0 \).
\end{itemproof}

\begin{corollary}
  Пусть \( X \) "--- МП, \( M \subset X \).
  \begin{enumerate}
    \item Если \( M \) "--- компакт,
      то \( M \) замкнуто и вполне ограничено в \( X \)
    \item Если \( X \) "--- полно, а \( M \) "---
      замкнуто и вполне ограниченно, то \( M \) компактно.
    \item Если \( X = \Real^n \), а \( M \)
      замкнуто и ограниченно, то \( M \) "--- компактно.
  \end{enumerate}
\end{corollary}

\begin{exercise}
  Метрическое пространство компактно \( \oTTo \)
  \( C(X, \Real) \subset B(X, \Real) \)
  (т. е. всякая непрерывная функция ограничена).
\end{exercise}

\begin{definition}
  Пусть \( (X, \rho_X) \), \( (Y, \rho_Y) \) "---
  метрические пространства, тогда
  отображение \( f : X \to Y \) называется
  \emph{равномерно непрерывным}, если
  \[
    \Forall{\epsilon > 0}
    \Exists{\delta > 0}
    \rho_X(x, y) < \delta \To
    \rho_Y(f(x), f(y)) < \epsilon.
  \]
\end{definition}

\begin{theorem*}[Кантора]
  Пусть \( X \) "--- компактное метрическое пространство
  и \( f : X \to \Real \) непрерывна,
  тогда \( f \) "--- равномерно непрерывна на \( X \).
\end{theorem*}
\begin{proof}[Прямое доказательство]
  Для произвольного \( \epsilon > 0 \)
  \[
    \Forall {x \in X} \Exists{r(x)}
    \Forall{y \in B(x, r(x))}
    |f(x) - f(y)| < \epsilon.
  \]
  Выберем из открытого покрытия
  \( \{ B(x, \frac{r(x)}{2}) \} \) конечное
  подпокрытие \( B_1, \dots, B_n \)
  и обозначим \[ \delta = \min_i \frac{r(x_i)}{2} > 0. \]
  Покжаем, что \( \Forall{x, y \in X} \rho(x, y) < \delta
  \To |f(x) - f(y)| < \epsilon \).
  Для некоторого \( i \) \( x \in B_i \subset B(x_i, r(x_i)) \),
  и при этом
  \[
    \rho(x_i, y) \le
    \rho(x_i, x) + \rho(x, y) <
    \frac{r(x_i)}{2} + \delta \le
    r(x_i) \To y \in B(x_i, r(x_i)).
  \]
  Наконец,
  \[
    |f(x) - f(y)| \le
    |f(x) - f(x_i)| + |f(x_i) - f(y)| <
    2 \epsilon.
  \]
\end{proof}
\begin{proof}[Доказательство от противного]
  Пусть это не так, т. е.
  \[
    \Exists{\epsilon_0 > 0}
    \Forall{\delta > 0}
    \Exists{x, y, \rho(x, y) < \delta}
    |f(x) - f(y)| \ge \epsilon_0.
  \]
  Выберем последовательность таких пар \( (x_n, y_n) \)
  для \( \delta_n = \frac{1}{n} \); \( \rho(x_n, y_n) \to 0 \),
  \( |f(x_n) - f(y_n)| \ge \epsilon_0 \).
  По теореме~\ref{thm:compact} можно выбрать подопоследовательность.
  \( x_{n_k} \to x_0 \). Но тогда и \( y_{n_k} \to x_0 \),
  а т. к. \( f \) непрерывна, то \( f(x_{n_k}), f(y_{n_k}) \to f(x_0) \),
  что противоречит тому, что
  \( |f(x_{n_k}) - f(y_{n_k})| \ge \epsilon_0 \).
\end{proof}

\begin{definition}
  Пусть \( X \) "--- метрическое пространство. Множество \( M \subset X \)
  называется \emph{предкомпактным}, если \( \Cl M \) компактно.
\end{definition}
\begin{example}
  \( X = \Real \), \( M = (0, 1) \).
\end{example}

\begin{exercise}
  Если \( X \) "--- метрическое пространство и \( M \subset X \)
  предкомпактно, то \( M \) "--- вполне ограниченно.
  Если же \( X \) полно, а \( M \) вполне ограниченно,
  то \( M \) "--- предкомпактно.
\end{exercise}

\begin{exampleslist}
  \item \( X = \Real^n \): \( M \) предкомпактно \( \oTTo \) \( M \) ограниченно.
  \item \( X = l_2 \): \( M \) компактно \( \oTTo \)
    \( M \) замкнуто, ограниченно и
    \[ \Forall{\epsilon > 0} \Exists{N} \Forall{ \{ x_n \} \in M }
    \sum_{n = N + 1}^{\infty} |x_n|^2 < \epsilon. \]
  \item \( L_p \): см. Колмогоров-Фомин
\end{exampleslist}

\begin{definition}
  \( M \subset C(X) \) равностепенно непрерывно, если
  \[
    \Forall{\epsilon > 0}
    \Exists{\delta > 0}
    \Forall{x, y} \rho(x, y) < \delta \To \Forall{f \in M}
    |f(x) - f(y)| < \epsilonю
  \]
\end{definition}

\begin{theorem}[Арцела"---Асколи]
  Пусть \( X \) "--- компактное метрическое пространство.
  Тогда \( M \subset C(X) \)
  предкомпактно \( \oTTo \)
  \( M \) ограниченно в \( C(X) \) (равномерно ограниченно)
  и равностепенно непрерывно.
\end{theorem}
\begin{proof}
  Т. к. \( C(x) \) полно, достаточно показать эквивалентность
  вполне ограниченности и данных условий.

  Если \( M \) вполне ограниченно, то
  ограниченность следует очевидным образом,
  докажем равностепенную непрерывность.
  Зафиксируем \( \epsilon > 0 \).
  Выберем конечную \( \epsilon \)-сеть
  \( \{ \phi_1, \dots, \phi_n \} \).
  Все эти функции непрерывны, т. е.
  \( \Exists{\delta_i > 0} \rho(x, y) < \delta_i \To
  |\phi_i(x) - \phi_i(y)| < \epsilon \);
  обозначим \( \delta = \min \{ \delta_1, \dots, \delta_n \} \).
  По определению \( \epsilon \)-сети,
  \[ \Forall{f \in M} \Exists{i} ||f - \phi_i|| \le \epsilon, \]
  т. е. \( \Forall{x \in X} |f(x) - \phi_i(x)| \le \epsilon \).
  Возьмём \( x, y \in X \) такие, что \( \rho(x, y) < \delta \),
  тогда
  \[
    |f(x) - f(y)| \le
    |f(x) - \phi_i(x)| + |\phi_i(x) - \phi_i(y)| + |\phi_i(y) - f(y)| <
    3\epsilon, \]
  что и требовалось.

  Доказательство в обратную сторону проведём только для
  \( X = [a, b] \), общий случай разобран в Колмогорове-Фомине.
  Пусть теперь \( M \) равностепенно непрерывно и ограниченно,
  построим для произвольного \( \epsilon > 0 \) конечную \( \epsilon \)-сеть.
  По условию равностепенной непрерывности
  \[ \Exists{\delta > 0} |x - y| \le \delta \To \Forall{f \in M} |f(x) - f(y)| < \epsilon. \]
  Кроме того, из-за ограниченности, функции из \( M \) принимают значения
  только в отрезке \( [-K, K] \) для некоторого \( K > 0 \).
  Покроем \( [a, b] \times [-K, K] \) сеткой с клетками размера
  \( \delta \times \epsilon \)
  (при необходимости, граничные клетки сделаем меньших размеров),
  и рассмотрим \( \{ \psi_k \} \) "---
  множество всех функций-ломанных,
  принимающих значения в узлах этой сетки
  (очевидно, их конечно много);
  данное множество является \( 5\epsilon \)-сетью для \( M \).

  Действительно, если \( a = x_0 \le x_1 < \dots, x_n = b \) "---
  координаты вертикальных линий нашей сетки,
  то для произвольной \( f \in M \)
  мы можем выбрать \( \psi \) из нашего набора
  так, что
  \[ \Forall{k} |f(x_k) - \psi(x_k)| < \epsilon. \]
  Выберем теперь произвольный \( x \in [a, b] \), для него найдётся
  \( k \) такое, что \( x \in [x_k, x_{k+1}] \). Тогда
  \[
    |f(x) - \psi(x)| \le |f(x) - f(x_k)| + |f(x_k) - \psi(x_k)| +
    |\psi(x_k) - \psi(x)|.
  \]
  По построению, \( |f(x) - f(x_k)| < \epsilon \) и
  \( |f(x_k) - \psi(x_k)| < \epsilon \).
  Кроме того,
  т. к. на отрезке \( [x_k, x_{k+1}] \) \( \psi \) линейна,
  \[
    |\psi(x) - \psi(x_k)| \le
    |\psi(x_{k+1}) - \psi(x_k)| \le
    |\psi(x_{k + 1}) - f(x_{k + 1})| +
    |f(x_{k + 1}) - f(x_k)| +
    |f(x_{k}) - \psi(x_k)|,
  \]
  и вновь применяя те же соображения мы получаем
  \( |f(x) - \psi(x)| < 5 \epsilon \).
\end{proof}

\end{document}
