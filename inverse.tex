\documentclass[main]{subfiles}

\begin{document}
\section{Обратный оператор}

\begin{definition}
  Пусть на линейных пространствах \( E_1 \), \( E_2 \)
  задан линейный оператор \( A : E_1 \to E_2 \).
  Тогда \emph{образом} и \emph{ядром} оператора \( A \)
  называются, соотвественно,
  \begin{align}
    &\Img A := \{ A x \mid x \in E_1 \}, \\
    &\Ker A := \{ x \in E_1 \mid A x = 0 \}.
  \end{align}
\end{definition}

\begin{remark}
  \( \Img A \) и \( \Ker A \)
  являются линейными многообразиями.
  Более того, если \( A \) ограничен, то
  \( \Ker A \) "--- подпространство \( E_1 \).
\end{remark}

\begin{definition}
  Пусть на линейных пространствах \( E_1 \), \( E_2 \)
  задан линейный оператор \( A : E_1 \to E_2 \).
  Тогда оператор \( B : \Img A \to E_1 \) называется
  \begin{itemize}
    \item \emph{правым обратным} к \( A \),
      если \( AB = I_{\Img A} \);
    \item \emph{левым обратным} к \( A \),
      если \( BA = I_{E_1} \);
    \item \emph{обратным} к \( A \),
      если он является одновременно правым и левым обратным
      к \( A \) (обозначается как \( A^{-1} \)).
  \end{itemize}
\end{definition}

\begin{remark}
  Существование правого обратного оператора эквивалентно
  разрешимости для любого \( y \in \Img A \)
  уравнения \( A x = y \) относительно \( x \),
  а существование левого обратного "---
  единственности такого решения (если оно существует).
\end{remark}

\begin{remark}
  Если существуют одновременно
  правый и левый обратные операторы
  \( B \) и \( C \),
  то они равны
  (и являются обратным оператором к \( A \)):
  \[
    AB = I, CA = I \To
    C = C I = C (AB) = (CA) B = I B = B.
  \]
  Следовательно, обратный к \( A \) оператор
  единственнен, если он существует.
\end{remark}

В случае конечномерных линейных пространств
любой линейный оператор ограничен,
и \( A \) обратим тогда и только тогда,
когда \( \Ker A = \{ 0 \} \).
Далее в этом параграфе мы рассмотрим
достаточные и необходимые условия
существования ограниченного обратного оператора
в случае бесконечномерных пространств.

\begin{theorem}
  Пусть \( E_1 \), \( E_2 \) "---
  нормированные пространства,
  \( A \in \Linears{E_1}[E_2] \).
  Тогда
  \[
    \exists A^{-1} \in \Linears{\Img A}[E_1] \oTTo
    \Exists{m > 0} \Forall{x \in E_1} ||Ax|| \ge m ||x||.
  \]
\end{theorem}
\begin{itemproof}
\item [\( \To \)]
  Вследствие ограниченности \( A^{-1} \)
  \( \Exists{M > 0} \Forall{y \in \Img A} ||A^{-1} y|| \le M||y|| \),
  и положив для произвольного \( x \in E_1 \) \( y = Ax \)
  мы получим \( ||x|| = ||A^{-1} (A x)|| \le M ||Ax|| \),
  т. е. подойдёт \( m = \frac{1}{M} \).
\item[\( \oT \)] % \( \Exists{m} ||Ax|| \ge m ||x|| \).
  Если \( x \ne 0 \), то \( ||Ax|| \ge m ||x|| > 0 \To Ax \ne 0 \),
  т. е. \( \Ker A = \{ 0 \} \).
  Значит, \( A \) инъективен и существует
  обратный линейный оператор \( A^{-1} : \Img A \to E_1 \).
  Кроме того, для произвольного \( y \in \Img A \)
  найдётся \( x \in E_1 \) такой, что \( y = Ax \),
  и тогда
  \[
    ||A^{-1} y|| = ||A^{-1} A x|| = ||x|| \le
    \frac{1}{m} ||Ax|| = \frac1m ||y||,
  \]
  таким образом \( ||A^{-1}|| \le \frac{1}{m} \)
  и \( A^{-1} \in \Linears{\Img A}[E_1] \).
\end{itemproof}

\begin{theorem}\label{thm:inverse-neumann} % 6.2
  Пусть \( E \) "--- банахово пространство,
  \( A \in \Linears{E} \) и \( ||A|| < 1 \).
  Тогда \( \exists (I + A)^{-1} \in \Linears{E} \).
\end{theorem}
\begin{proof}
  Идея доказательства состоит
  в переносе на операторы формулы
  \[
    \frac{1}{1 + x} = \sum_{n=0}^\infty {(-1)}^n x^n,
  \]
  справедливой для вещественных чисел \( x \in (-1, 1) \).

  Рассмотрим последовательность
  \[
    S_n = I - A + A^2 - \ldots + {(-1)}^n A^n,
  \]
  покажем, что у неё существует предел \( S \)
  и что \( S = {(I + A)}^{-1} \).

  Заметим, что \( \{ S_n \} \) фундаментальна,
  ведь для любых \( n, p > 0 \)
  \[
    ||S_{n + p} - S_n|| =
    ||\sum_{k = n + 1}^{n + p} A^k|| \le
    \sum_{k = n + 1}^{n + p} ||A^k|| \le
    \sum_{k = n + 1}^{n + p} ||A||^k \le
    \sum_{k = n + 1}^{\infty} ||A||^k =
    \frac{||A||^{n+1}}{1 - ||A||}  \to 0.
  \]
  Тогда, поскольку из полноты \( E \) следует
  полнота \( \Linears{E} \),
  \( S_n \to S \in \Linears{E} \).

  Покажем теперь, что \( S \)
  является и правым, и левым обратным к \( I + A \).
  Для начала отметим, что для произвольного
  \( B \in \Linears{E} \) \( B S_n \to B S \),
  ведь \( ||B S_n - B S|| \le ||B|| \cdot ||S_n - S|| \to 0 \).
  Далее, заметим, что
  \[
    (I + A) S_n = (I + A)(I - A^2 + A^3 - \ldots) =
    I + {(-1)}^n A^{n+1},
  \]
  и тогда \( ||(I + A) S_n - I|| = ||A^{n+1}|| \le ||A||^{n+1} \to 0 \),
  откуда следует, что \( S \) "--- правый обратный к \( I + A \).
  Доказать, что \( S \) "--- левый обратный к \( I + A \)
  можно абсолютно аналогично.
  \[ (I + A)(I - A + A^2 - \dots + (-1)^k A^k =
  I + (-1)^k A^{n + 1}) \to I, \]
  т. е. \( S \) "--- действительно обратный оператор к \( I + A \).
\end{proof}
\begin{remark}
  Вообще говоря, композиция операторов
  некоммутативна,
  но если \( p_1 \) и \( p_2 \) "--- многочлены,
  то при естественном определении их значений
  на операторах для произвольного
  линейного оператора \( A : E \to E \) будет выполняться
  равенство
  \[
    p_1(A) \circ p_2(A) = (p_1 \cdot p_2)(A) =
    (p_2 \cdot p_1)(A) = p_2(A) \circ p_1(A),
  \]
  и в доказательстве теоремы можно
  было бы воспользоваться этим фактом.
\end{remark}

\begin{corollary}
  В условиях теоремы
  \[
    ||(I + A)^{-1}|| \le \frac{1}{1 - ||A||}, \quad
    ||(I + A)^{-1} - I|| \le \frac{||A||}{1 - ||A||}.
  \]
\end{corollary}

\begin{remark}
  Мы показали, что ряд Неймана
  \[
    \sum_{n=0}^\infty (-1)^n A^n
  \]
  соходится при \( ||A|| < 1 \).
  На самом деле, можно (и нужно для выполнения задания)
  показать, что сходимость этого ряда эквивалентна
  существованию \( n \) такого, что \( ||A^n|| < 1 \).
\end{remark}

\begin{example}
  Случай \( n > 1 \) возможен.
  Например, можно показать, что для
  оператора \( A \),
  определённого на \( C[a, b] \) как
  \[
    (Af)(x) = \int\limits_0^x f(t) dt
  \]
  \( ||A^n|| = \frac{1}{n!} \).
\end{example}

\begin{example}
  \( E = C[0, 1] \), \( (Af)(x) = \int\limits_0^x f(t) dt \);
  \( ||A^n|| = \frac{1}{n!} \), поэтому уточнение существенно.
\end{example}

\begin{theorem} % 6.3
  Пусть \( E_1 \) "--- банахово пространство,
  \( E_2 \) "--- нормированное пространство
  и \( A \in \Linears{E_1}[E_2] \)
  имеет обратный оператор
  \( A^{-1} \in \Linears{E_2}[E_1] \).
  Тогда для любого оператора
  \( \Delta \in \Linears{E_1}[E_2] \)
  такого, что \( ||\Delta|| < ||A^{-1}||^{-1} \),
  \[
    \exists {(A + \Delta)}^{-1} \in \Linears{E_2}[E_1].
  \]
\end{theorem}
\begin{proof}
  Заметим, что
  \( A + \Delta = A (I + A^{-1} \Delta) \),
  при этом \( A^{-1} \Delta \in \Linears{E_1} \)
  и \( ||A^{-1} \Delta|| \le ||A^{-1}|| \cdot ||\Delta|| < 1 \).
  Значит, применима теорема~\ref{thm:inverse-neumann}
  и существует \( {(I + A^{-1} \Delta)}^{-1} \in \Linears{E_1} \),
  и тогда
  \[
    {(A + \Delta)}^{-1} = {(I + A^{-1} \Delta)}^{-1} A^{-1}.
  \]
\end{proof}

\begin{exercise}
  Оцените \( ||{(A + \Delta)}^{-1}|| \)
  и \( ||A^{-1} - {(A + \Delta)}^{-1}|| \).
\end{exercise}

\begin{theorem}[Банах, об обратном операторе, б/д] % 6.4
  Пусть \( E_1 \), \( E_2 \) "--- банаховы пространства,
  \( A \in \mathcal{L}(E_1, E_2) \) "--- биекция.
  Тогда \( A^{-1} \in \mathcal{L}(E_2, E_1) \).
\end{theorem}

\end{document}
