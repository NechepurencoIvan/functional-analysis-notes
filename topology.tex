\documentclass[main]{subfiles}

\begin{document}

\section{Топологические пространства}

\begin{definition}
  Назовём пару \( (X, \tau) \), где \( \tau \subset 2^X \),
  топологическим пространством, если выполнены следующие
  условия:
  \begin{enumerate}
    \item \( \emptyset, X \in \tau \)
    \item \( \bigcup_\alpha G_\alpha \in \tau \), если
      \( \Forall{\alpha} G_\alpha \in \tau \)
    \item \( \bigcap_{k = 1}^n G_k \), если
      \( \Forall{k \in \{ 1, \dots, n \}} G_k \in \tau \)
  \end{enumerate}
  Элементы \( \tau \) называются открытыми множествами,
  а их дополнения "--- замкнутыми.
\end{definition}

\begin{definition}
  Если \( x \in X \) и \( M \subset X \), то \( x \)
  называется
  \begin{description}
    \item[точкой прикосновения] \( M \), если
      \( \Forall{G \in \tau, x \in G} M \cap G \ne \emptyset \)
    \item[предельной точкой] \( M \), если
      \( \Forall{U(x)} \Exists{m \in M} x \ne m \in U(x) \).
    \item[внутренней точкой] \( M \), если
      \( \Exists{U(x)} U(x) \subset M \).
  \end{description}
\end{definition}

\begin{definition}
  Последовательность \( x_n \) сходится к \( x \), если
  \( \Forall{U(x)} \Exists{N} \Forall{n \ge N} x_n \in U \).
\end{definition}

\begin{exercise}
  Доказать эквивалентность данных определений аналогичным
  метрическим.
\end{exercise}

\begin{remark}
  Как и в метрических пространствах,
  что из существовании последовательности
  \( \{ x_n \} \subset M \) т. ч. \( x_n \to x \)
  следует, что \( x \) "--- точка прикосновения \( m \).
  Обратное, вообще говоря, не верно.
\end{remark}

\begin{definition}
  \( A \) плотно в \( B \), если \( B \subset \overline{A} \).
\end{definition}

\begin{exercise}
  Можно ли заменить в определениее нигде не плотного множества в МП
  шар на произвольную окрестность? Как быть в ТП?
\end{exercise}

\begin{example}
  Для произвольного множества \( X \) можно ввести две
  топологии: антидискретная \( tau_1 = \{ \emptyset, X \} \)
  и дискретная \( 2^X \).
\end{example}

\begin{definition}
  Топология \( tau_1 \) сильнее \( tau_2 \)
  (\( \tau_1 \preceq \tau_2 \)),
  если \( \Forall{M_1 \in \tau_1} M_1 \in \tau_2 \),
  т. е. \( \tau_1 \subset \tau_2 \).
\end{definition}

\begin{example}
  Над \( X = C[a, b] \) \( \rho_1(f, g) = \int_a^b |f(x) - g(x)| dx \),
  порождает топологию сильнее, чем
  \( \rho_2(f, g) = \max_x |f(x) - g(x)| \).
\end{example}

\begin{remark}
  Обычно, если топология метризуема,
  то она считается достаточно сильной.
  Например, в МП \( \Forall{x \ne y}
  \Exists{U(x), V(y)} U \cap V \ne \emptyset \).
\end{remark}

\begin{exercise} % даже задача
  Сходимость в \( D(\Real) \) не порождается никакой метрикой.
\end{exercise}

\begin{definition}
\end{definition}

\begin{theorem}
  Пусть \( (X, \tau_X) \), \( (Y \) "--- топологические пространтсва,
  \( f : X \to Y \). Тогда следующие утверждения
  эквивалентны:
  \begin{enumerate}
    \item \( f \) "--- непрерывна
    \item \( \Forall{G \in \tau_Y} f^{-1}(G) \in \tau_X \)
    \item Замкнутые.
  \end{enumerate}
\end{theorem}
\begin{proof}
  \( (1) \to (2) \) Дано: \( \Forall{V(f(x))} \Exists{U(x)} f(U) \subset V \).
  Пусть  \( G \in \tau_Y \), тогда для произвольного \( x \in f^{-1}(G) \)
  мы можем выбрать окрестность \( U(x) \) т. ч. \( f(U) \subset G \To
  U \subset f^{-1}(G) \). А значит,
  \( f^{-1}(G) = \bigcup_{x \in f^{-1}(G)} U(x) \).

  \( (2) \to (1) \) Для \( V(f(x)) \) возьмём \( U = f^{-1}(V) \),
  тогда это открытое множество, и т. к. \( V \) "--- окрестность \( f(x) \),
  то \( x \in U \). При этом, \( f(U) \subset V \) 
  (заметим: равенство может не достигаться).

  \( (2) \otto (3) \) Заметим: для произвольного отображения
  \( f^{-1}(Y \setminus F) = X \setminus f^{-y}(F) \).
\end{proof}

\begin{definition}
  Пусть \( X \), \( Y \) "--- топологические пространства,
  \( f : X \to Y \) "--- биекция, тогда если \( f \) и \( f^{-1} \)
  непрерывны, то \( f \) называется \emph{гомеоморфизмом}.
\end{definition}

\begin{remark}
  Для метрических пространств аналогом гомеоморфизма является
  \emph{изометрия}, т. е. биекция \( f : X \to Y \) такая,
  что \( \Forall{x_1, x_2 \in X} \rho_X(x_1, x_2) = 
  \rho_Y(f(x_1), f(x_2)) \).
\end{remark}

\begin{exercise}
  Пусть \( (X, \rho \) "--- метрическое пространство,
  тогда \( \rho \) "--- непрерывное отображение.
  Подсказка: \( |\rho(x, y) - \rho(a, b)| \le
  \rho(x, a) + \rho(y, b) \).
\end{exercise}

\begin{exercise}
  Пусть \( f : X \to Y \), \( g : Y \to Z \) "--- непрерывные
  отображения, тогда \( g \circ f \) "--- тоже непрерывно.
\end{exercise}
\begin{exercise}
  Сохраняется ли в топологических пространствах сходимость
  последовательности под действием непрерывного отображения?
\end{exercise}

\begin{definition}
  Топологическое пространство \( X \) называется \emph{несвязным},
  если \( X = G_1 \sqcup G_2 \), где
  \( G_1, G_2 \in \tau_X \setminus \{ \emptyset \} \).
  Если же \( X \) не является несвязным, то оно
  называется \emph{связным}. Множество \( Y \subset X \) называется
  связным или несвязным, если пространство
  \( Y \) с индуцированной топологией связно или ннесвязно,
  соответственно.
\end{definition}

\begin{exercise}
  Пусть \( G \subset \Real^n \) "--- открытое множество, тогда
  \( G \) связно тогда и только тогда, когда \( G \)
  линейно связно. Придумайте пример \( F \subset \Real^2 \),
  которое связно, но не является линейно связным.
\end{exercise}

\begin{definition}
  Топологическое пространство \( X \) компактно,
  если для любого набора \( \{ G_\alpha \}_{\alpha \in A} \)
  являющегося покрытием (т. е. такого,
  что \( X = \bigcup_\alpha G_\alpha \)) существует
  конечное подпокрытие, т. е.
  \( \Exists{\alpha_1, \dots, \alpha_n} X = G_{\alpha_1} \cup
  \dots \cup G_{\alpha_n} \).
\end{definition}

\begin{exercise}
  Если \( X \) "--- компактное топологическое пространство,
  \( Y \) "--- топологическое пространство,
  \( f : X \to Y \) непрерывно, тогда \( f(X) \) "--- компакт.
\end{exercise}

\begin{exercise}
  Пусть \( X \), \( Y \) "--- топологические пространства,
  \( f : X \to Y \) непрерывно, \( A \subset X \) всюду
  плотно в \( X \); тогда \( f(A) \) всюду плотно в \( f(X) \).
\end{exercise}

\begin{definition}
  Линейным \emph{нормированным пространством} называется
  линейное пространство \( E \) над \( \Real \) или \( \Complex \)
  снабжённое \emph{нормой}, т. е. отображением
  \( || \cdot || : E \to \Real_+ \), для которого выполнены
  следующие свойства:
  \begin{enumerate}
    \item \( ||x|| = 0 \oTTo x = 0 \)
    \item \( ||\alpha x|| = |\alpha| ||x|| \)
    \item \( ||x + y || \le ||x|| + ||y|| \).
  \end{enumerate}
\end{definition}

\begin{example}
  \( \Real^n_{p} \), где \( p \ge 1 \), с нормой
  \( ||x||_p = \left(\sum_{k=1}^n |x_k|^p \right)^{1/p} \).
\end{example}
\begin{example}
  \( \Real^n_\infty \) с нормой \( ||x||_\infty = \max_{k} |x_k| \).
\end{example}
\begin{example}
  \( l_\infty \).
\end{example}
\begin{example}
  Для произвольного \( X \) можно рассматривать
  \( B(X) \) "--- множество ограниченных числовых функций
  с нормой \( ||f||_\infty = \sup_{x \in X} |f(x)| \).
  Заметим: \( l_\infty = B(\Natural) \).
\end{example}

% C_p[a, b] (интегральная), C[a, b]

\begin{theorem}[принцип вложенных шаров]
  Пусть \( X \) "--- полное метрическое пространство, \( B_n \) "---
  последовательность замкнутых вложенных шаров с радиусами
  \( r_n \to 0 \). Тогда \( \bigcup B_n = \{ x \} \).
\end{theorem}
\begin{exercise}
  Без требования \( r_n \to 0 \), общей точки может не быть.
\end{exercise}

\begin{theorem}[Бэр]
  Пусть \( X \) "--- полное метрическое пространство.
  Тогда \( X \) нельзя представить в виде
  нельзя представить в виде счётного объединения
  нигде не плотных множеств, т. е.
  оно ялвяется \emph{множеством второй категории}
\end{theorem}

\begin{definition}
  \( X \) "--- метрическое пространство, 
  отображение \( f : X \to X \) называется \emph{сжимающим},
  если \( \Forall{x, y \in X} \rho(f(x), f(y)) \le \alpha \rho(x, y) \)
  для некоторого \( \alpha \in (0, 1) \).
\end{definition}

\begin{theorem}[Банах, 1922, принцип сжимающих отображений]
  Пусть \( X \) "--- полное метрическое пространство,
  \( f : X \to X \) "--- сжимающее отображение,
  тогда \( \ExistsOne{x \in X} f(x) = x \).
\end{theorem}

\begin{theorem}[Хаусдорф, б/д]
  Для любого неполного МП \( X \) существует его
  пополнение \( Y \), т. е. 
\end{theorem}

\begin{exercise}
  Доказать теорему Хаусдорфа. Подсказка: ввести
  отношение эквивалентности на классе всех
  фундаментальных последовательностей,
  где \( \{ x_n \} \sim \{ y_n \} \),
  если \( \rho(x_n, y_n) \to 0 \),
  после чего рассмотрим фактор по нему,
  и ввести на нём метрику
  \( \rho([\{ x_n \}], [\{ y_n \}]) = \lim \rho(x_n, y_n) \).
\end{exercise}

\end{document}
