\documentclass[main]{subfiles}

\begin{document}
\section{Метрические пространства}

\begin{definition}
  \emph{Метрическое пространство} "--- это
  пара \( (X, \rho) \), где \( \rho : X \times X \to \Real_+ \) и
  для любых \( x, y, z \in X \) верно:
  \begin{enumerate}
    \item \( \rho(x, y) = 0 \oTTo x = y \)
    \item \( \rho(x, y) = \rho(y, x) \)
    \item \( \rho(x, y) \le \rho(x, z) + \rho(z, y) \)
  \end{enumerate}
\end{definition}

Примеры метрических пространств:
\begin{itemize}
  \item Дискретная метрика на произвольном \( X \):
    \( \rho(x, y) = I\{ x \ne y \} \)
  \item 
    \( (\Real^n, \rho_p) \), где \( p \ge 1 \) и
    \( \rho_p(x, y) = \left( \sum_{k=1}^{n}
    |x_k - y_k|^p \right)^{1/p} \)
  \item 
    \( C[a, b] \) с метрикой
    \( \rho(f, g) = \sup_{x \in [a, b]} |f(x) - g(x)| \)
  \item \( C_2[a, b] \) с метрикой
    \( \rho(f, g) = \sqrt{\int_a^b |f(x) - g(x)| dx} \)
\end{itemize}

  Подпространство: \( Y \subset X \) и индуциорванная метрика:
  \( \rho_Y(y_1, y_2) = \rho_X(y_1, y_2) \) 

В следующих определениях \( (X, \rho) \) "---
метрическое пространство.
  
\begin{definition}
  Множество \( A \subset X \) называется ограниченным,
  если
  \[ \sup_{a, b \in A} \rho(a, b) < \infty. \]
\end{definition}

\begin{definition}
  Расстоянием между множествами \( A, B \subset X \)
  называется
  \[ \rho(A, B) = \inf_{a \in A, b \in B} \rho(a, b). \]
\end{definition}

\begin{definition}
  Открытыми и замкнутыми шарами называются, соответственно, множества вида
  \begin{gather}
    B(x, r) = \{ y \in X \mid \rho(y, x) < r \}, \\
    \overline{B}(x, r) = \{ y \in X \mid \rho(y, x) \le r \},
  \end{gather}
  где \( x \in X \) "--- центр шара и \( r > 0 \) "--- его радиус.
  Далее будем опускать центр или радиус, если это не существенно.
\end{definition}

\begin{exercise}
  \( A \subset X \) ограниченно \( \oTTo \Exists{\overline{B} \subset X}
  A \subset \overline{B} \).
\end{exercise}

\begin{definition}
  Пусть \( M \subset X \). Тогда \( x \in X \) называется:
  \begin{itemize}
    \item \emph{точкой прикосновения} множества \( M \), если
      \( \Forall{B(x)} B(x) \cap M \ne \emptyset \),
    \item \emph{предельной точкой} множества \( M \),
      если \( \Forall{B(x)} \Exists{y \in B(x) \cap M} y \ne x \),
    \item \emph{внутренней точкой} множества \( M \),
      если \( \exists B(x) \subset M \).
  \end{itemize}
  \emph{Замыканием} \( M \) называется множество его точек прикосновения
  \( \overline{M} \), а \emph{внутренностью} "--- множество его внутренних
  точек \( \Int M \). \( M \) \emph{замкнуто} или \emph{открыто},
  если \( M = \overline{M} \) или \( M = \Int M \), соответственно.
\end{definition}

\begin{remark}
  Для произвольного \( M \subset X \)
  \( \Int M \subset M \subset \overline{X} \).
\end{remark}

\begin{definition}
  Последовательность \( \{ x_n \}_{n=1}^\infty \subset X \)
  \emph{сходится} к \( x \in X \) (\( x_n \to x \)),
  если \( \rho(x_n, x) \to x \), \( n \to \infty \).
\end{definition}

\begin{exercise}
  Докажите, что \( x \) "--- точка прикосновения \( M \oTTo \)
  существует последовательность \( \{ x_n \}_{n=1}^\infty \subset M \)
  такая, что \( x_n \to x \).
\end{exercise}

\begin{definition}
  \( A \subset X \) \emph{плотно в} \( B \subset X \),
  если \( B \subset \overline{A} \).
  Множество плотне в \( X \) называется \emph{всюду плотным};
  если же не существует такого шара \( B \),
  что \( A \) плотно в \( B \),
  то \( A \) называется \emph{нигде не плотным}.
\end{definition}

\begin{definition}
  Метрическое пространство \( X \) называется сепарабельным,
  если в нём есть не более, чем счётное всюду плотное подмножество.
\end{definition}

\begin{definition}
  Если \( U \subset X \) открыто и \( x \in U \), то
  \( U \) называется \emph{окрестностью} точки \( x \).
\end{definition}

Далее под \( G \) будем подразумевать открытые множества,
а под \( F \) "--- замкнутые.

\begin{theorem}
  Пусть \( (X, \rho) \) "--- метрическое пространство,
  \( G \subset X \). Тогда \( G \) открыто \( \oTTo \)
  \( F = X \setminus G \) замкнуто.
\end{theorem}
\begin{proof}
  Пусть \( G \) открыто. Нужно доказать, что
  произвольная точка прикосновения множества \( F \)
  лежит в нём, что эквивалентно тому, что любая точка
  \( x \in G \) не является точкой прикосновения \( F \).
  Действительно: раз \( G \) открыто, то
  \( \exists B(x) \subset G \To B(x) \cap F = \emptyset \To \)
  \( x \) не есть точка прикосновения \( F \).

  Доказательно в обратную сторону остаётся в качестве упражнения.
\end{proof}

\begin{theorem}\label{thm:metric-topolotgy}
  Пусть \( (X, \rho) \) "--- метрическое пространство,
  \( \{ G_\alpha \}_{\alpha \in A} \) "--- открытые множества
  и \( \{ F_\beta \}_{\beta \in B} \) "--- замкнутые множества.
  Тогда \( \bigcup_{\alpha} G_\alpha \) и \( \bigcap_{k = 1}^n G_k \)
  открыты, а \( \bigcup_{k = 1}^n F_k \) и \( \bigcap_{\beta} F_\beta \) "---
  замкнуты.
\end{theorem}
\begin{proof}
  Пусть \( x \in G = \bigcup_{\alpha} G_\alpha \), тогда
  \( \Exists{\alpha \in A} x \in G_\alpha \), но \( G_\alpha \)
  открыто, откуда \( \exists B(x) \subset G_\alpha \subset G \).
  Значит, \( G \) "--- открыто.

  Пусть теперь \( G = \bigcap_{k = 1}^n G_k \) и \( x \in G \).
  Тогда также \( x \in G_1, \dots, G_n \), а значит
  существуют \( r_1, \dots, r_n > 0 \) такие, что
  \( B(x, r_k) \subset G_k \). Следовательно,
  при \( r = \min\{ r_1, \dots, r_n \} \)
  \( \Forall{k}  B(x, r) \subset B_k \subset G_k \),
  откуда \( B(x, r) \subset G \),
  т. е. \( x \) "--- внутренняя точка \( G \).

  Утверждения для замкнутых множеств следуют из уже доказанной половины
  теоремы, предыдущей теоремы и формул де Моргана.
\end{proof}

\begin{theorem}
  Пусть \( (X, \rho) \) "--- метрическое пространство,
  \( M \subset X \), \( x \in X \), \( r > 0 \).
  Тогда
  \begin{enumerate}
    \item \( B(x, r) \) "--- открытое множество
    \item \( \Int M \) "--- открытое множество
    \item \( \overline{B}(x, r) \) "--- замкнутое множество
    \item  \( \overline{M} \) "--- замкнутое множества.
  \end{enumerate}
\end{theorem}
\begin{proof}~
  \begin{enumerate}
    \item Мы хотим доказать, что для произвольного \( y \in B(x, r) \)
      \( \Exists{R > 0} B(y, R) \subset B(x, r) \).
      Если \( z \in B(y, R) \), то
      \( \rho(z, x) \le \rho(z, y) + \rho(y, x) < R + \rho(y, x) \),
      тогда при \( R = r - \rho(y, x) \) \( \rho(z, x) < r \To z \in B(x, r) \).
      Наконец, раз \( y \in B(x, r) \), \( \rho(y, x) < r \To R > 0 \).

    \item Заметим: \( M_1 \subset M_2 \To \Int M_1 \subset \Int M_2 \).
      Тогда если \( x \in \Int M \), то
      \( \Exists{B(x)} B(x) \subset M \To
      B(x) = \Int B(x) \subset \Int M \).
      Значит,
      \[ \Int M = \bigcup_{x \in \Int M} B(x), \]
      и тогда, по теореме~\ref{thm:metric-topolotgy},
      \( \Int M \) "--- открытое множество.
  \end{enumerate}
  Доказательство оставшихся двух пунктов остаётся в качестве упражнения.
\end{proof}

\begin{remark}
  Для произвольного \( M \subset X \)
  \begin{align}
    \Int M       &= \bigcup_{G \subset M} G, \\
    \overline{M} &= \bigcap_{M \subset F} F.
  \end{align}
\end{remark}

\end{document}
