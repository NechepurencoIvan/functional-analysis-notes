\documentstyle[main]{subfiles}

\begin{document}
Сначала пытались юзать базисы, но потом от этого отошли из-за
неэффективности.

Начало:
19000 Фредгольм, 1902 Лебег, 1906 Гильберт Френе 1906.

Отказ от сепарабельности, у Колмогорова она требовалась.
Почти всегда будет сепарабельно, но мы расширяем набор инструментов.

1887 Вольте’рра’ видит развитие анализа таким:
18в — локальное изучение функций на основе дифференцирования
19в — глобальное: т. Коши, т. Вейерштрасса о непрерывных на отрезке функциях
20в — век функционального исчисления (в английском синонимично анализу)
Для развития дифференциальных и интегральных уравнений, нужно будет рассматривать функции как точки,
и вначале, действительно, основным пространствами в рассмотрении были дифференцируемые
и интегрируемые функции ($C^n[a, b]$, $L_2[a, b]$).

Эти пространства объединяли алгебраическую и топологическую структуры, согласовывая их.

Изначально выстреливали отдельные моменты: например, Брауэр

Позже появился Банах, Гельфанд и сейчас ФА это очень обширная область,
но мы будем заниматься очень простыми вещами: основными структурами,
линейными ограниченными операторами (аналог функций типа $f(x) = kx$).

Раньше курс на физтехе был заточен под ДУ, сейчас это менее актуально
и на ФИВТе почти нет баз изучающих процессы описываемые ДУ.
Пример: Чтобы искать максимум функции, нужно научиться дифференцировать.
Значит, для оптимизации функций нужно научиться дифференцировать функционалы.

Например, у нас есть ФП $D \subset E_1(\Real)$ и отображение $F : D \to E_2$.
Тода мы ищем вид $F(x_0 + h[E_2])[E_2] - F(x_0) = A h + o(h)$.

Второе направление: хочется научиться применять что-то вроде метода касательных Ньютона
для гораздо более сложных уравнений.

1910 Брауэр: $K \subset \Real^n$ — компакт, гомеоморфный шару.
Неподвижная точка

Пример Какутани: для $K = \overline{B}(0; 1) \subset L_2$ существует
отображение $f : \oveline{B} \to \overline{B}$ без неподвижных точек.

Келлаг и Биркгоф пытались доказать аналогичное утверждение
ужесточая требования к множеств, но это неудобно, т. к. нам хочется
работать с шариками.
А Шаудер в 1929 доказал это, требуя компактности отображения.

\begin{theorem}[1922 Банах]
  Пусть $E$ "--- полное нормированное пространство, $F : E \to E$ "---
  линейный ограниченной оператор, $||F|| < 1$. Тогда
  $\ExistsOnly{x \in E} Fx = x$.
\end{theorem}

Переформулировка:
Пусть $X$ "--- ПМП, $F : X \to X$ "--- сжимающее:
$\Exists{\alpha \in (0, 1)} \Forall{x, y \in X}
d(fx, fy) \le \alpha d(x, y)$. Тогда
$\ExistsOnly{x \in X} Fx = x$.

Фейнман, поиск корня приближениями.
Нужно сделать уже данное 

\begin{problem}
  $A = (a_{ij})$, $\Forall{i} |\sum_{j} a_{ij}| < 1$.
  Хотим доказать обратимость $E + A$,
  для этого достаточно показать существование единственного
  решения уравнения $(E + A) x = b$, что эквивалентно
  $x = b - A x$. Нужно найти метрику для $\Real^n$,
  чтобы у нас работал метод последовательных приближений.
\end{problem}

\end{document}
