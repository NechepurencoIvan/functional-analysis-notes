\documentclass[main]{subfiles}

\begin{document}

\section{Мера и инеграл Лебега}%10
Для начала будем работать с отрезком $[a, b]$.
Тогда любому промежутку с концами $\alpha$, $\beta$
хочется сопоставить меру $\beta - \alpha$.

\begin{proposition}
  Наименьшая алгебра, порождённая промежутками "---
  это в точности множество конечных объединений непересекающихся
  промежутков.
\end{proposition}

Для $\Real^n$ мы поставим некоторые требования на меру:
\begin{enumerate}
  \item Мера параллелепипеда "--- это его объём в обычном понимании.
  \item Меры конгруэнтных (т. е. переводимых друг в друга изометрией) множеств
    совпадают.
  \item Счётная аддитивность (или хотя бы конечно аддитивная).
  \item Мечта: измеримость ограниченных множеств.
\end{enumerate}

\begin{theorem}[Лебег]
  Пусть $M$ "--- семейство измеримых множеств (на $[a, b]$).
  Тогда $M$ "--- $\sigma$-алгебра, на которой $\mu^*$ является
  полной счётноаддитивной мерой.
\end{theorem}

\begin{definition}
  Мера $\mu$ плона, если при $\mu E = 0$ и $E_1 \subset E$
  $E_1 \in M$ и $\mu E_1 = 0$.
\end{definition}

Определим верхнюю меру
\[ \mu^* E = \inf_{E \subset \bigcup A_k, A_k \in A} \sum_{k = 1}^\infty \mu_{\text{эл}} A_k. \]

\begin{exercise}
  $\mu^*$ "--- полуаддитивна:
  \[ \mu^*(A \cup B) \le \mu^* A + \mu^* B. \]
\end{exercise}

\begin{exercise}
  $\mu_{\text{эл}}$ "--- счётно аддитивная мера на $A$:
  если $A = \bigcup A_k$, $A_k \in A$, $A \in A$, то
  \[ \mu_{\text{эл}} A = \sum \mu_{\text{эл}} A_k. \]
\end{exercise}

\begin{exercise}
  Докажите полноту меры Лебега.
\end{exercise}

\section{Измеримые функции}

\begin{definition}
  Функция $f : [a, b] \to \Real$ называется измеримой по Борелю, если
  \[ \Forall{B \in \mathcal{B}(\Real)} f^{-1}(B) \in \mathcal{B}([a, b]). \]
  Если вместо борелевской $\sigma$-алгебры рассматривать множества из $M$,
  то получим определение измеримости по Лебегу.
\end{definition}

Канторово множество:
Замкнуто, $\mu F = 0$ и $|F| = C$.

\begin{exercise}
  $ \# F = c. $
\end{exercise}

\begin{exercise}
  Непрерывные и монотонные функции являются измеримыми по Лебегу.
\end{exercise}

\begin{definition}
  $f : [a, b] \to \Real$ является измеримой, если
  $\Forall{\alpha \in \Real} f^{-1}(-\infty; \alpha)$.
\end{definition}

\begin{theorem}
  Измеримые функции обладают следующими свойствами:
  \begin{enumerate}
    \item Линейное пространство $f g$; $f / g$ ($g \ne 0$)
    \item Это решетка:
      $h(x) = \max(f(x), g(x))$ "--- измерима.
      $f$ и $|f|$ измеримы одновременно.
    \item $f_n \to f$ (поточечно, потчи всюду) $\To $f "--- измерима.
  \end{enumerate}
\end{theorem}

\begin{exercise}
  Докажите, что если $f$ и $g$ измерима, то
  $k f$ и $f + g$ измеримы.
\end{exercise}

\section{Простые измеримые функции}

\begin{definition}
  Измеримая $f$ называется \emph{простой}, если множество её
  значений не более, чем счётно.
\end{definition}

\begin{exercise}
  $f$ простая $\oTTo [a, b] = \bigcup E_k$, $E_k \in M$,
  $f(x) = \sum c_k I_{E_k}(x)$.
\end{exercise}

\begin{exercise}
  Любая измеримая функция представима как равномерный предел
  простых функций.
\end{exercise}

\begin{theorem}[Фубини]
  Пусть $K = [a, b] \times [c, d]$, $f(x, y)$ "--- интегрируема на $K$.
  Тогда
  \[
    L \iint_K f(x, y) dx dy
    = \int_a^b dx \int_c^d dy f(x, y)
    = \int_c^d dy \int_a^b dx f(x, y).
  \]
\end{theorem}

\begin{corollary}
  Пусть $f$ измеримв на $K$ и конечен хотя бы
  один из следующих повторных интегралов:
  \[
    \int_a^b dx \int_c^d dy |f(x, y)|
    = \int_c^d dy \int_a^b dx |f(x, y)|.
  \]
  Тогда $f(x, y)$ "--- интегрируема
  и к ней можно применить теорему Фубини.
\end{corollary}
\begin{proof}
  Рассмотрим последовательность функций
  \[
    f_n(x, y) = \min \{ |f(x, y)|, n \};
  \]
  они, очевидно, неотрицательны, интегрируемы 
  и п. в. сходятся к $|f|$. Кроме того,
  к каждой из них мы можем применить теорему Фубини,
  и, считая что конечен первый интеграл из условия,
  получим
  \[
    \iint_K f_n dx dx =
    \int dx \int dy f_n \le \int dx \int dy |f| = M < \infty,
  \]
  а тогда по теореме Фату $|f|$ интегрируема,
  из чего следует и интегрируемость $f$.
\end{proof}

Пусть $X = \bigcup X_n$, $X_n \subset X_{n+1}$ "--- пространства с согласованными
мерами. Тогда можно определить
\[
  L\int_X f(ч) d\mu = \lim_{n \to \infty} \int_{X_n} f d\mu,
\]
если предел существует и независит от выбора $\{ X_n \}$.

\begin{definition}
  Измеримые функции $f$ и $g$ назвыаются
  эквивалентными, если они совпадают почти всюду.
\end{definition}

\begin{proposition}
  Для интегрируемой $f$ на $X$
  \[
    \int_X |f| d \mu = 0 \oTTo
    f = 0 \ae
  \]
\end{proposition}
\begin{proof}
  Остаётся в качестве упражнения.
  Совет: воспользуйтесь неравенством Чебышёва:
  Пусть задано пространство с мерой $(X, \mu)$,
  $f$ интегрируема на $X$. Тогда
  \[
    \int_X |f| d \mu \ge
    \int_{x : |f| \ge C} |f| d \mu
    \ge c \mu\{ x \mid |f(x)| \ge c \},
  \]
  или
  \[
    \mu\{ x \mid |f(x)| \ge c \}
    \le \frac1c \int_X |f| d\mu.
  \]
\end{proof}

Теперь мы можем построить на основе интеграла Лебега
норму 

\begin{theorem} % 10.1
  Пространство $L_p[a, b]$ "--- сепарабельное
  банахово пространство с нормой
  \[
    ||f|| = \left( \int_a^b |f(x)|^p d x \right)^{\frac1p},
  \]
  а $L_2[a, b]$ "--- сепарабельное гильбертово пространство
  со скалярным произведением
  \[
    (f, g) = \int_a^b f(x) \overline{g(x)} dx.
  \]
\end{theorem}
\begin{proof}
  Линейность следует из теоремы про $\widetilde{L}_1(x, y)$.
  Для доказательства свойств нормы
  пользуемся неравенством Минковского, выводимым из
  неравенства Гёльдера, и утверждением.

  Мы доказывали в первом задании, что ЛНП $E$
  является банаховым $\oTTo$ всякий абсолютно
  сходящийся ряд сходится; воспользуемся
  этим утверждением для доказательства
  банаховости. Пусть нам дан ряд
  \[
    \sum_{k = 1}^n u_k(x) \;
    \sum_{k = 1}^n ||u_k||_1 =
    \sum_{k = 1}^\infty \int_a^b |u_k(x)| dx
    < \infty,
  \]
  и нам нужно доказать, что он сходится.
  Применим теорему Беппо Леви к $\sum |u_k(x)|$
  и получим, что он сходится почти всюду к интегрируемой функции,
  а тогда попросту из свойств числовых рядов
  почти всюду сходится $S(x) = \sum u_k(x)$.

  Покажем теперь, что
  $||S_n - S||_1 \to  0$ и $S \in \widetilde{L}_1[a,b]$.
  Из абсолютной сходимости ряда мы получаем, что
  $\{ S_n \}$ фундаментальна:
  \[
    \Forall{\epsilon > 0} \Exists{N}
    \Forall{n, m \ge N} ||f_n(x) - f_m(x)||_1 < \epsilon,
  \]
  и, переходя к пределу по $m$ с помощью теоремы Фату,
  получаем
  $||S_n(x) - S(x)||_1 \le \epsilon$ при $n \ge N$,
  т. е. $||S_n - S||_1 \to 0$.
  Почему $S \in L_1[a, b]$? Она измерима, и при этом
  ограничена интегрируемой:
  \[
    |S(x)| = |\sum_{k = 1}^\infty u_k(x)| \le
    \sum_{k = 1}^\infty |u_k(x)|,
  \]
  а потому $|S(x)|$ интегрируема и $S \in \widetilde{L}_1[a,b]$.

  Осталось показать сепарабельность.
  Через непрерывные функции перейдём к многочленам.

  Выберем конечную сумму индикаторов,
  потом перейдём к замкнутым подмножеству
  $F$ и надмножесвту $G$
  большой меры
  \[
    \phi_b(x) = \begin{cases}
      1, & x \in F \\
      0, & x \notin G \\
      \frac{\rho(x, CG)}{\rho(x, CG) + \rho(x, F)}, & \text{иначе}
    \end{cases}
  \]
\end{proof}

\end{document}
