\documentclass[main]{subfiles}

\begin{document}

\section{Мера и интеграл Лебега}%10
Наша цель в этом параграфе "---
построить меру Лебега
(обобщение понятий длины, площади и объёма)
на отрезке \( [a, b] \subset \Real \)
и с её помощью определить интеграл Лебега.

\subsection{Мера Лебега на отрезке}
\begin{definition}
  Пусть \( X \) "--- произвольное
  множество, семейство его подмножеств
  \( \mathcal{A} \subset 2^X \)
  называется \emph{алгеброй},
  если выполнены следующие свойства:
  \begin{enumerate}
    \item \( \emptyset \in \mathcal{A} \);
    \item Если \( A \in \mathcal{A} \),
      то \( X \setminus A \in \mathcal{A} \);
    \item Если \( A_1, A_2 \in \mathcal{A} \),
      то \( A_1 \cup A_2 \in \mathcal{A} \).
  \end{enumerate}
  Если, к тому же, для любой последовательности
  множеств \( {\{ A_n \}}_{n=1}^\infty \subset \mathcal{A} \)
  \[
    \bigcup_{n=1}^\infty A_n \in \mathcal{A},
  \]
  то \( \mathcal{A} \) называется \emph{\( \sigma \)-алгеброй}.
\end{definition}

\begin{definition}
  Пусть заданы произвольное множество \( X \)
  и алгебра \( \mathcal{A} \) на нём.
  Тогда отображение \( \mu : \mathcal{A} \to [0, \infty] \)
  называется \emph{(конечно-аддитивной) мерой}
  на \( \mathcal{A} \), если \( \mu \emptyset = 0 \)
  и для непересекающихся множеств \( A_1, A_2 \in \mathcal{A} \)
  \( \mu(A_1 \cup A_2) = \mu A_1 + \mu A_2 \).
  Если, помимо этого,
  для любого счётного семейства
  попарно непересекающихся множеств
  \( {\{ A_n \}}_{n=1}^\infty \subset \mathcal{A} \)
  такого, что
  \[ A = \bigcup_{n = 1}^\infty A_n \in \mathcal{A}, \]
  выполнено равенство
  \[ \mu A = \sum_{n=1}^\infty \mu A_n, \]
  то \( \mu \) называется \emph{счётно-аддитивной} мерой.
\end{definition}

Изначальной целью Лебега было
построение меры \( \mu \) для \( \Real^n \),
удовлетворяющей следующим четырём требованиям:
\begin{enumerate}
  \item \( \mu ([a_1, b_1] \times \ldots \times [a_n, b_n]) =
    (b_1 - a_1) \cdot \ldots \cdot (b_n - a_n) \)
    при \( a_i \le b_i \),
  \item если \( A \) конгруэнтно \( B \), то
    \( \mu A = \mu B \),
  \item счётная аддитивность,
  \item измеримость всех ограниченных множеств.
\end{enumerate}

К сожалению, удовлетворить всем четырём требованиям
в \( \Real^n \) невозможно, как показал Джузеппе Витали;
ослабление свойства счётной аддитивности
до конечной аддитивности позволяет
построить такую меру на \( \Real \)
и \( \Real^2 \), но не на пространствах большей размерности.
Мы же попросту откажемся требования
измеримости всех ограниченных множеств.

Для \( \alpha < \beta \) будем обозначать
через \( \langle \alpha, \beta \rangle \) числовой промежуток
с концами \( \alpha \) и \( \beta \), т. е.
одно из множеств \( [\alpha, \beta] \),
\( (\alpha, \beta] \),
\( [\alpha, \beta) \),
\( (\alpha, \beta) \).
Определим \( \mathcal{A} \)
как минимальную по включению алгебру,
содержащую множество
\[
  \{ \langle \alpha, \beta \rangle \mid a \le \alpha < \beta \le b \}.
\]

\begin{proposition}
  \( \mathcal{A} \) есть в точности
  семейство конечных объединений непересекающихся промежутков.
\end{proposition}

Определим теперь на \( \mathcal{A} \)
\emph{элементарную меру} \( \muel \),
положив
\[
  \muel \bigcup_{i=1}^n \langle \alpha_i, \beta_i \rangle =
  \sum_{i = 1}^n (\beta_i - \alpha_i).
\]

\begin{proposition}
  \( \muel \) "--- счётно-аддитивная мера на \( \mathcal{A} \).
\end{proposition}

Далее, мы определим
для произвольного множества \( E \subset [a, b] \)
\emph{верхнюю меру}
\[
  \mu^* E = \inf \left\{
    \sum_{n = 1}^\infty \muel A_n \mid
    \{ A_n \}_{n=1}^\infty \subset \mathcal{A}: \:
    E \subset \bigcup_{n=1}^\infty A_n
  \right\}.
\]

\begin{exercise}
  \( \mu^* \) "--- полуаддитивная функция:
  \[ \Forall{E_1, E_2 \subset [a, b]} \mu^*(E_1 \cup E_1) \le \mu^* E_1 + \mu^* E_2. \]
\end{exercise}

Наконец, назовём \emph{измеримыми множествами}
семейство
\[
  \mathcal{M} = \{
    E \subset [a, b] \mid
    \Forall{\epsilon > 0}
    \Exists{A \in \mathcal{A}}
    \mu^*(E \bigtriangleup A) < \epsilon
  \}.
\]

\begin{definition}
  Мера \( \mu \) на \( \mathcal{M} \) \emph{полна},
  если для любого \( E \in \mathcal{M} \)
  такого, что \( \mu E = 0 \),
  произвольно множество \( E_1 \subset E \)
  принадлежит \( \mathcal{M} \), и при этом \( \mu E_1 = 0 \).
\end{definition}

\begin{theorem*}[Лебега]
  Семейство измеримых множеств \( \mathcal{M} \)
  образует \( \sigma \)-алгебру на \( [a, b] \)
  и при этом \( \mu = \mu^* \bigr|_{\mathcal{A}} \)
  является полной счётно-аддитивной мерой на \( \mathcal{M} \).
\end{theorem*}

Полученную в теореме меру
\( \mu \) мы будем называть
\emph{мерой Лебега} на \( [a, b] \).

\begin{exercise}
  Докажите полноту меры Лебега.
\end{exercise}

\subsection{Измеримые функции}

\begin{definition}
  Функция $f : [a, b] \to \Real$ называется \emph{измеримой по Борелю}, если
  \[ \Forall{B \in \mathcal{B}(\Real)} f^{-1}(B) \in \mathcal{B}([a, b]). \]
  \( f \) называется \emph{измеримой (по Лебегу)}, если
  \[ \Forall{A \in \mathcal{B}(\Real)} f^{-1}(B) \in \mathcal{M}. \]
\end{definition}

\begin{remark}
  \( \mathcal{B}([a, b]) \subset \mathcal{M} \),
  а потому из измеримости по Борелю следует
  измеримость по Лебегу.
  Более того, вложение строгое из соображений мощности.
  С одной стороны,
  борелевских множеств континуально много.
  С другой стороны, можно построить канторово
  множество, которое имеет нулевую меру
  и континуальную мощность, тогда
  в силу полноты меры Лебега \( \mathcal{M} \)
  уже будет содержать более, чем континуум, его подмножеств.
\end{remark}

\begin{exercise}
  Непрерывные и монотонные функции измеримы по Лебегу.
\end{exercise}

\begin{proposition}
  Функция \( f : [a, b] \to \Real \) является измеримой
  тогда и только тогда, когда
  \[
    \Forall{\alpha \in \Real} f^{-1}((-\infty; \alpha)) = 
    \{ x \in [a, b] \mid f(x) < \alpha \} \in \mathcal{M}.
  \]
\end{proposition}

\begin{definition}
  Последовательность функций
  \( \{ f_n : [a, b] \to \Real \}_{n=1}^\infty \)
  \emph{сходится почти всюду} к функции \( f : [a, b] \to \Real \)
  (\( f_n \toae f \)), если найдётся \( E \in \mathcal{M} \)
  такое, что \( \mu E = 1 \) и
  \[
    \Forall{x \in E} \lim_{n \to \infty} f_n(x) = f(x).
  \]
\end{definition}

\begin{proposition}
  Пусть \( f : [a, b] \to \Real \) измерима,
  а \( g : \Real \to \Real \) измерима по Борелю,
  тогда \( g \circ f \) измерима.
\end{proposition}

\begin{theorem*}[свойства измеримых функций, б/д]
  Пусть \( f , g : [a, b] \to \Real \) "--- измеримые функции,
  \( c \in \Real \). Тогда
  \begin{enumerate}
    \item \( c f \), \( f + g \) и \( f \cdot g \)
      измеримы, а если \( \Forall{x \in [a, b]} g(x) \ne 0 \),
      то \( f / g \) измерима
    \item
      \( (h(x) = \max \{ f(x), g(x) \} \) измерима
      (следовательно, измерима \( |f| = \max \{f, -f \} \))
    \item Если функции
      \( \{ f_n : [a, b] \to \Real \}_{n=1}^\infty \)
      измеримы и для некоторой \( f_0 : [a, b] \to \Real \)
      \( f_n \toae f_0 \), то \( f_0 \) "--- измерима
  \end{enumerate}
\end{theorem*}

\begin{exercise}
  Докажите, что если
  \( f, g : [a, b] \to \Real \) измеримы,
  то \( \Forall{c \in \Real} с f \) и \( f + g \) измеримы.
\end{exercise}

\subsection{Простые измеримые функции}

\begin{definition}
  Измеримая функция \( f : [a, b] \to \Real \)
  называется \emph{простой},
  если множество её значений
  не более, чем счётно.
\end{definition}

\begin{exercise}
  \( f : [a, b] \to \Real \) "--- простая
  функция тогда и только тогда, когда
  существует разбиение \( [a, b] \)
  на измеримые множества \( \{ E_n \}_{n = 1}^\infty \)
  такое, что
  \[
    f = \sum_{n=1}^\infty c_n I_{E_n}.
  \]
\end{exercise}

Из курса математического анализа нам
известна следующая теорема,
позволяющая представить обширный
класс функций в виде равномерных пределов
последовательностей достаточно простых функций:

\begin{theorem*}[Вейерштрасса]
  \( f \in C[a, b] \) \( \oTTo \)
  существует последовательность
  многочленов \( \{ P_n \} \),
  равномерно сходящаяся к \( f \)
  на \( [a, b] \).
\end{theorem*}

Важность простых функций заключается в том,
что они являются аналогом многочленов
для непрерывных функций в этом смысле.

\begin{theorem*}
  \( f : [a, b] \to \Real \)
  измерима \( \oTTo \)
  существует последовательность
  простых функций
  \( \{ f_n : [a, b] \to \Real \} \),
  сходящаяся к \( f \) равномерно на \( [a, b] \).
\end{theorem*}

Обратная импликация попросту следует из свойств
измеримых функций, а для доказательства слева
направо нужно рассмотреть последовательность
\[
  f_n(x) =
  \sum_{m =-\infty}^\infty \frac{m}{n}
  I(\frac{m}{n} \le f(x) < \frac{m+1}{n}),
\]
тогда для любого \( x \in [a, b] \)
\( |f_n(x) - f(x)| \le \frac1n \).

\subsection{Интеграл Лебега}

\begin{definition}
  Простая функция \( f : X \to \Real \)
  такая, что
  \[
    f = \sum_{n = 1}^\infty c_n I_{E_n},
  \]
  где \( \{ E_n \} \) "--- измеримые множества,
  называется \emph{интегрируемой (по Лебегу)} на \( X \),
  если ряд
  \[
    \sum_{n=1}^\infty c_n \mu E_n
  \]
  сходится абсолютно,
  и тогда для неё определён
  \emph{интеграл на \( X \)}
  \[
    \intl_X f d \mu :=
    \sum_{n=1}^\infty c_n \mu E_n.
  \]
\end{definition}

\begin{definition}
  Измеримая функция \( f : X \to [a, b] \)
  называется \emph{интегрируемой (по Лебегу)},
  если существует последовательность
  простых интегрируемых на \( X \) функций
  \( \{ f_n \} \), сходящаяся к \( f \)
  равномерно на \( X \).
  В этом случае её \emph{интегралом на \( X \)}
  называется
  \[
    \intl_X f d\mu :=
    \lim_{n \to \infty} \intl_X f_n d\mu.
  \]
\end{definition}

\begin{proposition}[корректность определения интеграла Лебега]
  В условиях предыдущего определения
  \begin{enumerate}
    \item Существует предел последовательности
      \( \intl f_n d\mu \)
    \item Для простых функций оба определения согласованы
    \item Значение \( \intl f d\mu \) не зависит
      от выбора \( \{ f_n \} \)
  \end{enumerate}
\end{proposition}

Будем обозначать множество интегрируемых
функций \( f : [a, b] \to \Real \)
как \( \preLp[a, b] \).

\begin{theorem*}[свойства интеграла Лебега]
  Выполняются следующие утверждения:
  \begin{enumerate}
    \item \( \preLp[a, b] \) "--- линейное пространство
    \item Если \( f \) "--- измерима, то
      \( f \in \preLp[a, b] \oTTo |f| \in \preLp[a, b] \)
    \item Если \( \mu E = 0 \) и \( f : [a, b] \to \Real \)
      принимает ненулевые значения лишь на \( E \),
      то \( f \in \preLp[a, b] \)
    \item Пусть \( f : [a, b] \to \Real \) "--- измерима
      функция, а \( g \in \preLp[a, b] \); тогда
      если \( |f| \le g \), то
      \( f \in \preLp[a, b] \)
    \item Пусть \( f, g \in \preLp[a, b] \)
      и \( f \le g \), тогда
      \[
	\intl_{[a,b]} f d\mu \le \intl_{[a,b]} g d\mu
      \]
  \end{enumerate}
\end{theorem*}

\begin{theorem*}[Егорова]
  Пусть на \( X = [a, b] \) задана
  последовательность измеримых функций
  \( \{ f_n \} \), сходящаяся к
  \( f : X \to \Real \) почти всюду.
  Тогда для любого \( \delta > 0 \)
  найдётся измеримое множество \( X_\delta \subset X \)
  такое, что \( \mu X - \mu X_\delta < \delta \)
  и при этом \( \{ f_n \} \) сходится
  к \( f \) равномерно на \( X_\delta \).
\end{theorem*}

\subsection{Предельный переход под знаком интеграла}

\begin{theorem*}[Лебега, об ограниченной сходимости]
  Пусть \( X = [a, b] \), \( \mu \) "--- мера Лебега на \( X \).
  Пусть на \( X \) задана последовательность
  измеримых функций \( \{ f_n \}_{n=1}^\infty \),
  сходящаяся почти всюду к \( f : X \to \Real \),
  и \( \Exists{g \in \preLp[a, b]} \Forall{n} |f_n| \le g \).
  Тогда \( f \) "--- интегрируема и при этом
  \[
    \intl_X f_n d \mu \to
    \intl_X f d \mu.
  \]
\end{theorem*}

\begin{theorem*}[лемма Фату]
  Пусть \( \{ f_n \} \) "--- последовательность
  неотрицательных интегрируемых функций на \( X \),
  сходящаяся почти всюду к \( f : X \to \Real \).
  Тогда если найдётся \( M \) такое, что
  \[
    \Forall{n} \intl_X f_n d \mu \le M,
  \]
  то \( f \) интегрируема на \( X \)
  и, более того,
  \[
    \intl_X f d \mu \le M.
  \]
\end{theorem*}

\begin{theorem*}[Беппо Леви]
  Пусть на \( X \) задана последовательность
  неотрицательных интегрируемых функций \( \{ u_n \} \),
  причём
  \[
    \sum_{n=1}^\infty \intl_X u_n d \mu < \infty.
  \]
  Тогда ряд \( \sum u_n \) сходится к некоторой
  интегрируемой функции \( S : X \to \Real \)
  почти всюду и
  \[
    \intl_X S d \mu = \sum_{n=1}^\infty \intl_X u_n d \mu < \infty.
  \]
\end{theorem*}
\begin{remark}
  Теорему можно переформулировать
  для неубывающей последовательности функций,
  рассматривая её как последовательность
  частичных сумм некоторого неотрицательного ряда.
\end{remark}

\begin{theorem*}
  Пусть функция \( f : [a, b] \to \Real \)
  интегрируема по Риману на отрезке \( [a, b] \).
  Тогда \( f \in \preLp[a, b] \) и при этом
  \[
    \intl_{[a,b]} f d \mu = \int\limits_a^b f(x) dx.
  \]
  Более того, это верно даже если
  \( f \) абсолютно интегрируема по Риману в несобственном смысле.
\end{theorem*}

\begin{theorem*}
  \( f : [a, b] \to \Real \) интегрируема по Риману
  на отрезке \( [a, b] \) тогда и только тогда, когда
  \( f \) ограниченна и множество её точек разрыва
  имеет нулевую меру.
\end{theorem*}

\begin{remark}
  Доказывается с помощью теоремы Лебега.
\end{remark}

\begin{theorem*}[Фубини]
  Пусть \( K = [a, b] \times [c, d] \),
  \( f(x, y) \) "--- интегрируема на \( K \).
  Тогда
  \[
    \intl_K f d \mu
    = \intl_{[a, b]} \left( \intl_{[c, d]} f(x, y) \mu(dy) \right) \mu(dx)
    = \intl_{[c, d]} \left( \intl_{[a, b]} f(x, y) \mu(dx) \right) \mu(dy).
  \]
\end{theorem*}

\begin{corollary}
  Пусть \( f \) измерима на \( K \)
  и конечен хотя бы
  один из следующих повторных интегралов:
  \[
    \intl_{[a, b]} \left( \intl_{[c, d]} |f(x, y)| \mu(dy) \right) \mu(dx),
    \quad
    \intl_{[c, d]} \left( \intl_{[a, b]} |f(x, y)| \mu(dx) \right) \mu(dy).
  \]
  Тогда \( f(x, y) \) "--- интегрируема
  и к ней можно применить теорему Фубини.
\end{corollary}
\begin{proof}
  Рассмотрим последовательность функций
  \[
    f_n(x, y) = \min \{ |f(x, y)|, n \};
  \]
  они, очевидно, неотрицательны, интегрируемы 
  и поточечно сходятся к \( |f| \). Кроме того,
  к каждой из них мы можем применить теорему Фубини,
  и, считая для примера,
  что конечен первый интеграл из условия,
  получим
  \[
    \int_K f_n d \mu =
    \int dx \int dy f_n \le \int dx \int dy |f| = M < \infty,
  \]
  Тогда, по лемме Фату, \( |f| \) интегрируема,
  из чего следует интегрируемость \( f \).
\end{proof}

Мы определили меру и, соответственно, интеграл
лишь на ограниченных множествах.
Однако, эти понятия можно перенести
на всё \( \Real^n \) с помощью следующей конструкции.
Пусть \( X = \bigcup X_n \), где
\( X_n \subset X_{n+1} \) "--- пространства
с согласованными мерами \( \mu_n \).
Тогда можно определить
\[
  \intl_X f d\mu = \lim_{n \to \infty} \intl_{X_n} f d\mu_n,
\]
если предел существует и не зависит от выбора \( \{ X_n \} \).

\begin{remark}
  Такая конструкция также сохраняет выполнимость теорем
  Лебега, Фату, Леви и Фубини.
\end{remark}

\subsection{Пространства \( \Lp_p \)}

\begin{definition}
  Измеримые функции \( f, g : X \to \Real \)
  \emph{равны почти всюду} (\( f = g \muae \)),
  если
  \[
    \mu \{ x \in X \mid f(x) \ne g(x) \} = 0.
  \]
\end{definition}

\begin{proposition}
  Для \( f \in \preLp[a, b] \)
  \[
    \intl_{[a,b]} |f| d \mu = 0 \oTTo
    f = 0 \muae
  \]
\end{proposition}
\begin{proof}
  Остаётся в качестве упражнения.
  Совет: воспользуйтесь неравенством Чебышёва.
  Пусть задано пространство с мерой \( (X, \mu) \),
  \( f \in \preLp(X) \). Тогда
  \[
    \intl_X |f| d \mu \ge
    \intl_{\{ |f| \ge c \}} |f| d \mu
    \ge c \mu \{ x \mid |f(x)| \ge c \},
  \]
  т. е.
  \[
    \mu\{ x \mid |f(x)| \ge c \}
    \le \frac1c \intl_X |f| d\mu.
  \]
\end{proof}

Определим \( \Lp[a, b] \) как
фактормножество \( \preLp[a, b] \)
по отношению равенства почти всюду.
Положим также для \( p \ge 1 \)
\[
  \Lp_p[a, b] = \left\{
    [f] \in \Lp[a, b]
    \mid
    |f|^p \in \preLp[a, b]
  \right\}.
\]

Для доказательства следующей теоремы
нам потребуется два утверждения,
доказательство которых (или его изучение)
остаётся в качестве упражнения.

\begin{proposition}[неравенство Гёльдера]
  Пусть \( f, g : [a, b] \to \Real \) "---
  измеримые функции, \( p > 1 \) и
  \( \frac1p + \frac1q = 1 \).
  Если \( |f|^p \) и \( |f|^q \) интегрируемы на \( [a, b] \),
  то \( f \cdot g \) также интегрируема на \( [a, b] \),
  и при этом
  \[
    \intl_{[a, b]} f \cdot g d\mu \le
    \left( \intl_{[a,b]} |f|^p d\mu \right)^{\frac1p}
    \left( \intl_{[a,b]} |g|^q d\mu \right)^{\frac1q}.
  \]
\end{proposition}

\begin{proposition}[неравенство Минковского]
  Пусть \( f, g : [a, b] \to \Real \) "---
  измеримые функции, \( p > 1 \).
  Если \( |f|^p \) и \( |g|^q \) интегрируемы на \( [a, b] \),
  то \( |f + g|^p \) также интегрируема на \( [a, b] \),
  и при этом
  \[
    {\left( \intl_{[a, b]} |f + g|^p d\mu \right)}^{\frac1p} \le
    {\left( \intl_{[a,b]} |f|^p d\mu \right)}^{\frac1p}
    +
    {\left( \intl_{[a,b]} |g|^p d\mu \right)}^{\frac1p}.
  \]
\end{proposition}

\begin{theorem} % 10.1
  Для произвольного \( p \ge 1 \)
  \( \Lp_p[a, b] \) "--- сепарабельное
  банахово пространство с нормой
  \[
    ||f||_p := \left( \intl_{[a,b]} |f|^p d\mu \right)^{\frac1p},
  \]
  а $L_2[a, b]$ "--- сепарабельное гильбертово пространство
  со скалярным произведением
  \[
    (f, g) := \intl_{[a,b]} f \cdot \overline{g} \: d\mu.
  \]
\end{theorem}
\begin{proof}
  Линейную структуру  мы определим
  следующим образом:
  \[ \alpha [f] + \beta [g] = [\alpha f + \beta g]. \]
  Определение корректно, т. к.
  \( [0] = \{ f \in \preLp[a, b] \mid f = 0 \muae \} \) "---
  линейное многообразие в \( \preLp \).
  Замкнутость следует из того, что
  \begin{align}
    \intl |\alpha f + \beta g|^p d\mu &=
    \intl |\alpha f + \beta g|^p I(f > g) d\mu +
    \intl |\alpha f + \beta g|^p I(f \le g) d\mu \le \\
    &\le \intl |2 \alpha f|^p I(f > g) d\mu +
    \intl |2 \beta g|^p I(f \le g) \le
    2 \alpha \intl |f|^p d\mu +
    2 \beta \intl |g|^p d\mu.
  \end{align}
  Далее мы будем отождествлять функцию
  \( f \) с её классом эквивалентности \( [f] \).

  Для доказательства того,
  что \( ||\cdot||_p \) "--- норма
  нужно воспользоваться неравенством Минковского,
  которое, по сути, означает выполнение
  третьего свойства из определения нормы для \( ||\cdot||_p \).
  Однородность попросту следует из линейности интеграла,
  а положительная определённость "---
  из сформулированного выше утверждения.

  Докажем полноту \( \Lp_1 \).
  Для этого воспользуемся задачей 4.1
  из задания первого семестра:
  полнота линейного нормированного пространства
  эквивалентна сходимости любого абсолютного сходящегося ряда.
  Итак, пусть \( \{ u_n \} \subset \Lp_1 \) и
  \[
    \sum_{n=1}^\infty ||u_n||_1 =
    \sum_{n=1}^\infty \intl |u_n| d\mu < \infty.
  \]
  Применяя теорему Леви
  к \( \{ |u_n| \} \),
  мы получаем,
  что для некоторой функции
  \( \widetilde{S} \in \preLp[a, b] \)
  почти всюду выполняется равенство
  \[
    \widetilde{S}(x) = \sum_{n=1}^\infty |u_n(x)|.
  \]
  Значит, по свойствам числовых рядов,
  можно определить функцию \( S : [a, b] \to \Real \)
  такую, что
  \[
    S(x) = \sum_{n=1}^\infty u_n(x) \muae.
  \]
  Обозначим через \( \{ S_n \} \)
  частичные суммы ряда \( \sum u_n \);
  тогда \( S_n \toae S \) и,
  поскольку \( \{ S_n \} \) измеримы,
  \( S \) "--- измерима.
  Более того, почти всюду \( |S(x)| \le \widetilde{S}(x) \),
  а потому \( S \in \preLp[a, b] \).
  Из абсолютной сходимости \( \sum u_n \)
  следует, что \( S_n \) фундаментальна,
  т. е. для произвольного \( \epsilon > 0 \)
  можно выбрать \( N \) такой,
  что для произвольных \( n, m \ge N \)
  \( ||S_n - S_m||_1 < \epsilon \).
  Зафиксируем произвольный \( n \ge N \)
  и определим \( g_m = |S_n - S_m| \),
  тогда \( g_m \toae |S_n - S| \);
  значит, по теореме Фату,
  \( |S_n - S| \in \preLp[0, 1] \)
  и \( ||S_n - S||_1 \le \epsilon \) для \( n \ge N \).
  Таким образом, по определению, \( ||S_n - S||_1 \to 0 \),
  т.~е. \( S \) "--- сумма ряда \( \sum u_n \)
  в смысле \( \Lp_1[a, b] \).

  Чтобы показать полноту \( \Lp_p[a, b] \)
  при \( p > 1 \) мы воспользуемся неравенством
  Гёльдера:
  \[
    ||f||_1 = \intl |f| \cdot 1 d\mu \le
    \left( \intl |f|^p d\mu  \right)^{\frac1p} \cdot
    \left( \intl |1|^q d\mu \right)^\frac1q =
    \sqrt[q]{b - a} ||f||_p.
  \]
  Тогда любой абсолютно сходящийся в \( \Lp_p[a, b] \)
  ряд \( \sum u_n \)
  сходится абсолютно и в \( \Lp_1[a, b] \),
  а значит,
  \[
    \sum_{n=1}^\infty u_n = S \muae
  \]
  как следует из доказательства полноты \( \Lp_1[a, b] \).
  Конечно, последовательность частичных сумм
  \( \{ S_n \} \) ограничена в \( \Lp_p[a, b] \),
  и, применив теорему Фату к последовательности
  \( \{ |S_n|^p \} \), мы покажем,
  что \( S \in \Lp_p[a, b] \).
  Наконец, сходимость ряда к \( S \)
  в смысле \( \Lp_p[a, b] \) доказывается
  аналогично случаю \( p = 1 \),
  если определить \( g_m = |S_n - S_m|^p \).

  Докажем, что \( \Lp_p[a, b] \) "--- сепарабельно.
  Зафиксируем \( f \in \Lp_p[a, b] \) и
  \( \epsilon > 0 \).
  Построим последовательность простых функций
  \( \{ f_n \} \), равномерно сходящуюся к \( f \) на
  \( [a, b] \). Тогда для некоторого \( m \)
  \( |f_m(x) - f(x)| < \frac{\epsilon}{b - a} \),
  а потому
  \[
    \left( \intl_{[a,b]} |f_m - f|^p d\mu \right)^{\frac1p} \le
    \epsilon,
  \]
  тогда \( f_m - f \in \Lp_p[a, b] \),
  \( f_m = (f_m - f) + f \in \Lp_p[a, b] \) и
  \( ||f - f_m||_p \le \epsilon \).
  Положим
  \[
    f_a = f_m = \sum_{k=1}^\infty c_k I_{E_k},
  \]
  где \( E_i \cap E_j = \emptyset \) при \( i \ne j \).
  Определим
  \[
    S_n = \sum_{k = 1}^n c_k I_{E_k}.
  \]
  Тогда \( |S_n - f_a|^p \le |f_a|^p \) и
  \( S_n \toae f_a \) (а значит, \( |S_n - f_a|^p \toae 0 \)),
  и по теореме Лебега об ограниченной сходимости
  \[
    \intl |S_n - f_a|^p d\mu \to 0.
  \]
  Значит, для некоторого \( n \) \( ||f_a - S_n|| < \epsilon \);
  положим \( f_b = S_n \).
  Пусть \( g_1, \dots, g_n \in C[a, b] \),
  определим
  \[
    f_c = \sum_{k = 1}^n c_k g_k.
  \]
  Тогда
  \[
    ||f_b - f_c||_p = ||\sum_{k=1}^n c_k(I_{E_k} - g_k)||_p \le
    \sum_{k=1}^n c_k ||I_{E_k} - g_k||_p,
  \]
  и если мы сможем построить достаточно хорошие
  непрерывные приближения для индикаторов, можно считать,
  что \( ||f_b - f_c||_p < \epsilon \).

  Пусть \( E \subset [a, b] \) "--- измеримое множество,
  \( \delta > 0 \).
  Воспользуемся фактом \emph{регулярности}
  меры Лебега на \( [a, b] \):
  найдутся такие множества
  \( F, G \subset [a, b] \),
  что \( F \) "--- замкнуто, \( G \) "--- открыто,
  \( F \subset E \subset G \)
  и \( \mu(G \setminus F) < \delta \).
  Определим
  \[
    g(x) = \begin{cases}
      1, & x \in F, \\
      0, & x \in [a, b] \setminus G, \\
      \frac{\rho(x, [a, b] \setminus G)}{
      \rho(x, F) + \rho(x, [a, b] \setminus G)},
      & x \in G \setminus F.
    \end{cases}
  \]
  Тогда \( g \in C[a, b] \), \( |I_E - g| \le 1 \)
  и при этом \( |I_E(x) - g| \ne 0 \) только
  при \( x \in G \setminus F \).
  Значит,
  \[
    ||I_E - g||_p^p =
    \intl |I_E - g|^p d\mu \le
    \intl I_{G \setminus F} d\mu =
    \mu(G \setminus F) < \delta,
  \]
  т. е. \( ||I_E - g|| < \sqrt[p]{\delta} \),
  что, в силу произвольности \( \delta \),
  позволяет положить \( ||f_b - f_c||_p < \epsilon \).

  По теореме Вейерштрасса мы сможем
  выбрать многочлен \( P \) такой,
  что \( ||f_c - P||_p < \epsilon \),
  и, благодаря плотности рациональных чисел
  в вещественных, мы сможем также выбрать
  многочлен \( Q \) с рациональными коэффициентами
  такой, что \( ||P - Q||_p < \epsilon \).
  Итак,
  \[
    ||f - Q||_p \le ||f - f_a||_p + ||f_a - f_b||_p
    + ||f_b - f_c||_p + ||f_c - P||_p + ||P - Q||_p <
    5 \epsilon,
  \]
  что, в силу счётности \( \Rational[x] \),
  означает сепарабельность \( \Lp_p[a, b] \).

  Для завершения доказательства осталось заметить,
  свойства
  что \( \langle \cdot, \cdot \rangle \) очевидным
  образом удовлетворяет определению скалярного
  произведения, и при этом
  \( ||f||_2 = \sqrt{\langle f, f \rangle} \).
\end{proof}

\begin{remark}
  Данная теорема будет также верна,
  если вместо вещественных функций
  мы будем рассматривать функции
  со значениями в \( \Complex \).
\end{remark}

\subsection{Преобразование Фурье в \( L_1(\Real^n) \)}

Преобразование Фурье обозначается с помощью крышечки:
\( f \in L_1 \mapsto \hat f (\xi) \).
\[
  \hat f (\xi) = \frac1{(2 \pi)^{n/2}} \intl_{\Real^n} f(x) e^{i\Inner{x, \xi}} dx
\]

\begin{theorem*}
  Если \( f \in L_1 \), то \( \hat f \in BC(\Real^n) \)
  (ограничена и непрерывна).
  Более того, \( \hat f(\xi) \to 0 \)  при \( \xi \to \infty \).
\end{theorem*}

Операция свёртки:
\[
  (f * g)(x) = \intl_{R^n} f(y) d(x - y) dy.
\]

\begin{theorem*}
  \( \widehat{f * g} = \hat f \cdot \hat g \).
\end{theorem*}

\( F^4 = I \)

В \( L_2 \): \( \hat f \in L_2 \).
\[
  f(x) = \intl_{\Real^n} \hat f(\xi) e^{-i \Inner{x, \xi}} d \xi.
\]

Равенство Парсеваля: \( ||\hat f||_2 = ||f||_2 \).

Равенство Блаблабла: \( \Inner{f, g} = \Inner{\hat f, \hat g} \).

\end{document}
