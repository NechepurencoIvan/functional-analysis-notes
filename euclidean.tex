\documentclass[main]{subfiles}

\begin{document}
%\section{Евклидовы пространства}

\begin{definition}
  \emph{Евклидовым пространством} называется
  линейное пространство \( E \) полем \( F \),
  где \( F = \Real \) или \( F = \Complex \),
  снабжённое \emph{скалярным произведением} "---
  отображением \( \Inner{\cdot, \cdot} : E \times E \to F \),
  удовлетворяющим следующим свойствам:
  \begin{enumerate}
    \item \( \Inner{x, x} \ge 0 \), \( \Inner{x, x} = 0 \oTTo x = 0 \)
    \item \( \Inner{x, y} = \overline{\Inner{y, x}} \)
    \item \( \Inner{\alpha x, y} = \alpha \Inner{x, y} \)
      (тогда верно и \( \Inner{x, \alpha y} = \overline{\alpha} \Inner{x, y} \))
    \item \( \Inner{x + y, z} = \Inner{x, z} + \Inner{y, z} \)
  \end{enumerate}
\end{definition}

На евклидовом пространстве можно ввести норму
\[ ||x|| := \sqrt{\Inner{x, x}}, \]
Но чтобы доказать для неё неравенство треугольника
потребуется вспомогательное утверждение.

\begin{lemma}[Неравенство Коши-Буняковского-Шварца]
  \( |\Inner{x, y}| \le ||x|| \cdot ||y|| \).
\end{lemma}
\begin{proof}[Доказательство для случая \( F = \Real \)]
  Для произвольных \( x, y \in E \) и \( \alpha \in \Real \)
  \[ 0 \le \Inner{x + \alpha y, x + \alpha y} =
  \Inner{x, x} + 2 \alpha \Inner{x, y} + \alpha^2 \Inner{y, y}. \]
  Если \( \Inner{y, y} = 0 \), то и \( \Inner{x, y} = 0 \),
  откуда тривиальным образом следует неравенство; иначе, данное неравенство
  задаёт неотрицательную параболу и тогда
  \( D = \Inner{x, y}^2 - \Inner{x, x} \cdot \Inner{y, y} \le 0 \).
\end{proof}

\begin{exercise}
  Адаптировать это доказательство для случая \( F = \Complex \).
\end{exercise}

\begin{proposition}[Неравенство Бесселя]
  Пусть \( E \) "--- евклидово пространство,
  \( {\{ e_k \}}_{k = 1}^n \subset E \) "--- ортонормированная система,
  \( x \in E \) и \( \alpha_k = \Inner{x, e_k} \). Тогда
  \[ ||x||^2 \ge \sum_{k = 1}^{n} |\alpha_k|^2. \]
\end{proposition}
\begin{proof}
  Доказать данное данное утверждение можно индукцией по \( n \).
  Случай \( n = 0 \) тривиален, а при \( n > 0 \) можно
  отбросить \( e_n \) и перейти к \( y = x - \alpha_n e_n \),
  ведь 
  \begin{align}
    0 \le \Inner{y, y} &=
    \Inner{x, x} - \alpha_n \Inner{e_n, x}
    - \overline{\alpha_n} \Inner{x, e_n}
    + \alpha_n \overline{\alpha_n} \Inner{e_n, e_n} = \\
    &= \Inner{x, x} - \alpha_n \overline{\alpha_n}
    - \overline{\alpha_n} \alpha_n +
    \alpha_n \overline{\alpha_n} \cdot 1 =
    ||x||^2 - |\alpha_n|^2,
  \end{align}
  а для \( k < n \)
  \[
    \Inner{y, e_k} =
    \Inner{x, e_k} - \alpha_n \Inner{e_n, e_k} =
    \alpha_k - \alpha_n \cdot 0 =
    \alpha_k. \qedhere
  \]
\end{proof}

\begin{proof}[Доказательство неравенства КБШ
  c помощью неравенства Бесселя]
  Если \( y = 0 \), то
  \[ 0 = \Inner{x, y} \le ||x|| \cdot ||y|| = 0, \]
  иначе "--- определим
  \[
    e_1 := \frac{y}{||y||},
  \]
  и тогда по неравенству Бесселя
  \[
    ||x||^2 \ge |\Inner{x, e_1}|^2 =
    \frac{|\Inner{x, y}|^2}{||y||^2} \To
    |\Inner{x, y}| \le ||x|| \cdot ||y||. \qedhere
  \]
\end{proof}

\begin{exercise}
  Докажите неравенство Коши-Буняковского-Шварца
  для полускалярного произведения "---
  обобщения скалярного произведения, в котором
  необязательно \( \Inner{x, x} = 0 \To x = 0 \).
\end{exercise}

Теперь докажем неравенство треугольника, точнее, что
\( ||x + y||^2 \le {(||x|| + ||y||)}^2 \):
\[
  ||x + y||^2 =
  ||x||^2 + \Inner{x, y} + \overline{\Inner{x, y}} + ||y||^2 =
  ||x||^2 + 2 \Re \Inner{x, y} + ||y||^2,
\]
и применение неравенства Коши-Буняковского-Шварца
завершает доказательство:
\[
  \Re \Inner{x, y} \le |\Inner{x, y}| \le ||x|| \cdot ||y||.
\]

\begin{definition}
  Евклидово пространство называется \emph{гильбертовым},
  если порождённое им нормированное пространство
  банахово.
\end{definition}

\begin{remark}
  В некоторых источниках (например, Колмогоров-Фомин)
  в определение гильбертова пространства также входит
  сепарабельность, что в сочетании с остальными
  условиями эквивалентно существованию базиса.
  Это позволяет проводить доказательства в более
  привычной координатной форме;
  мы же вместо этого будем активно использовать
  равенство параллелограмма.
\end{remark}

\begin{theorem}[б/д]
  Пусть \( E \) "--- линейное нормированное пространство.
  Норма в \( E \) порождается некоторым
  скалярным произведением \( \oTTo \) выполняется
  равенство
  параллелограмма:
  \[ ||a + b||^2 + ||a - b||^2 = 2||a||^2 + 2||b||^2. \]
\end{theorem}

\begin{example}
  \( C[0, 1] \) "--- не евклидово.
  Пусть \( f(x) = x \) и \( g(x) = 1 - x \).
  Тогда \( f + g = 1 \) и \( f - g = 2x - 1 \) и
  \[ ||f + g||^2 + ||f - g||^2 = 1^2 + 1^2 \ne 2 \cdot 1^2 + 2 \cdot 1^2 =
  2 ||f||^2 + 2||g||^2. \]
\end{example}

Идея доказательства проста для вещественного пространства:
определим скалярное произведение как
\[ \Inner{x, y} = \frac{||x + y||^2 - ||x - y||^2}{4}. \]
В комплексном случае требуется более сложная конструкция:
\[ \Inner{x, y} = \frac{1}{4} \sum_{\alpha \in \{ \pm 1, \pm i \}}
\alpha ||\alpha x + y||^2. \]

В конечномерном пространстве
у любого подпространства есть прямое дополнение:
если \( E_1 < E \), то \( \Exists{E_2 < E} E_1 \oplus E_2 = E \).
Что происходит в общем случае?
С помощью понятия базиса Гамеля и аксиомы выбора
можно доказать это утверждение для линейных многообразий,
а для подпространств нормированных пространств ситуация интереснее.

\begin{problem}
  В \( C[0, 1] \) существует "<недополняемое"> подпространство.
\end{problem}

\begin{exercise}
  В нормированном пространстве \( (E, ||\cdot||) \)
  \( || \cdot || \) непрерывна.
\end{exercise}

\begin{exercise}
  В евклидовом пространстве \( (E, \Inner{\cdot, \cdot}) \)
  \( \Inner{\cdot, \cdot} \) непрерывно:
  если \( x_n \to x \) и \( y_n \to y \),
  то \( \Inner{x_n, y_n} \to \Inner{x, y} \).
\end{exercise}

\begin{definition}
  Пусть \( E \) "--- линейное нормированное пространство,
  \( M \subset E \),
  \( x \in E \). Тогда \( y \) называется 
  \emph{элементом наилучшего приближения}
  \( x \) в \( M \), если
  \( y \in M \) и \( \rho(x, M) = ||x - y|| \).
\end{definition}

\begin{lemma}
  Пусть \( H \) "--- гильбертово пространство, \( M \subset H \) "---
  подпространство. Тогда
  \[ \Forall{h \in H} \ExistsOne{x \in M} \rho(h, M) = ||x - h||. \]
\end{lemma}
\begin{proof}
  Зафиксируем произвольный \( h \in H \) и обозначим \( d = \rho(h, M) \).

  Пусть \( x_1, x_2 \in M \) "---
  элементы наилучшего приближения для \( h \),
  т. е.  \( ||h - x_1|| = ||h - x_2|| = d \).
  Положим \( a = h - x_1 \) и \( b = h - x_2 \),
  тогда по равенству параллелограмма
  \[ ||2h - x_1 - x_2||^2 + ||x_1 - x_2||^2 =
  2||h - x_1||^2 + 2 ||h - x_2||^2 = 4 d^2, \]
  и тогда
  \[ ||x_1 - x_2||^2 = 4d^2 - 4 ||h - \overbrace{\frac{x_1 + x_2}{2}}^{\in M}||
  \le 4d^2 - 4d^2 = 0, \]
  откуда \( x_1 = x_2 \) "--- единственность доказана.

  Если \( d = 0 \), то, т. к. \( M \) "--- замкнуто,
  \( h \in M \) и искомый \( x \) есть сам \( h \).

  Если \( d > 0 \), то существует последовательность
  \( \{ x_n \} \subset M \) такая, что
  \( ||h - x_n|| \to d \).  Для произвольного \( \epsilon > 0 \)
  с некоторого \( N \) \( ||h - x_m||^2 \le d^2 + \epsilon \),
  и применив равенство параллелограмма
  к \( a = h - x_n \) и \( b = h - x_m \) мы получаем
  \[
    ||x_n - x_m||^2 = -4 ||h - \frac{x_n + x_m}{2}||^2
    + 2 ||h - x_n||^2 + 2||h - x_m||^2 \le
    -4 d^2 + 2d^2 + 2\epsilon + 2d^2 + 2\epsilon =
    4 \epsilon,
  \]
  т. е. \( \{ x_n \} \) фундаментальна, а значит,
  \( \Exists x = \lim x_n \) и тогда
  \( ||h - x_n|| \to ||h - x|| = d \) "---
  \( x \) и есть элемент наилучшего приближения.
\end{proof}

% Энфло 1973: может не быть базиса в сеп. БП

\begin{definition}
  Пусть \( E \) "--- евклидово пространство, \( S \subset E \).
  \emph{Аннулятором} \( S \) называется
  \[ S^\perp := \{ y \mid \Forall{s \in S} \Inner{s, y} = 0 \}. \]
\end{definition}

\begin{theorem}[о проекции]%4.3
  Пусть \( H \) "--- гильбертово пространство, \( M \subset H \) "---
  подпространство. Тогда \( H = M \oplus M^\perp \),
  причём \( M^\perp \) "--- подпространство.
\end{theorem}
\begin{proof}
  Докажем теорему для вещественных гильбертовых пространств. 

  Покажем, что произвольный элемент \( h \in H \) разложим
  в сумму элементов \( M \) и \( M^\perp \).
  Рассмотрим \( y = h - x \), где \( x \) "---
  элемент наилучшего приближения для \( h \) в \( M \).
  Тогда для произвольного \( m \in M \) и \( \alpha \in \Real \)
  \[ d^2 = ||h - x||^2 \le ||h - x - \alpha m||^2 =
  d^2 - 2 \alpha (y, m) + \alpha^2 ||m||^2 \To
  2 \alpha (y, m) \le \alpha^2 ||m||^2. \]
  Фиксируя \( \alpha = \epsilon > 0 \),
  получаем \( 2 (y, m) \le \epsilon ||m||^2 \),
  а \( \alpha = -\epsilon \)
  превращает неравенство в \( - 2 (y, m) \le \epsilon ||m||^2 \),
  т. е. \( 2 |(y, m)| \le \epsilon ||m||^2 \) для произвольного \( \epsilon > 0 \).
  Это возможно только при \( (y, m) = 0 \),
  что в силу произвольности выбора \( m \)
  означает, что \( y \in M^\perp \)
  и тогда \( H = M + M^\perp \).

  Единственность разложения следует из того,
  что \( M \cap M^\perp = \{ 0 \} \),
  ведь если \( x \in M \) и \( x \in M^\perp \)
  одновременно, то \( \Inner{x, x} = 0 \To x = 0 \).

  Последний шаг "--- доказать замкнутость \( M^\perp \).
  Это следует из непрерывности скалярного произведения:
  если \( z \) "--- точка прикосновения \( M^\perp \), то
  \( \Exists{ \{ z_n \} \subset M^\perp } z_n \to z \),
  и тогда
  \[ \Forall{m \in M} \Inner{m, z} \ot \Inner{m, z_n} = 0 \to 0, \]
  т. е. \( z \in M^\perp \).
\end{proof}

\begin{exercise}
  Для произвольного \( S \subset H \) \( S^\perp \) "--- замкнутое множество,
  %и \( S^\perp = {[S]}^\perp = \Cl{[S]}^\perp \).
  и
  \[ S^\perp = [S^\perp] = {[S]}^\perp = \Cl{[S]}^\perp. \]
\end{exercise}

\begin{definition}
  Система \( e \subset H \) \emph{полна}, если \( \Cl{[e]} = H \).
\end{definition}

\begin{theorem}
  Пусть \( H \) "--- гильбертово пространство,
  \( e = {\{ e_n \}}_{n = 1}^\infty \subset H \) "--- 
  ортонормированная система.
  Тогда следующие свойства эквивалентны:
  \begin{enumerate}
    \item \( e \) "--- базис
    \item \( e \) "--- полная система
    \item \( e^\perp = \{ 0 \} \)
    \item Для произвольного \( x \in H \) справедливо
      равенство Парсеваля:
      \[
	||x||^2 = \sum_{n=1}^\infty |\Inner{x, e_n}|^2.
      \]
  \end{enumerate}
\end{theorem}

Сформулируем в условиях теоремы некоторые вспомогательные утверждения.
Помимо прочего, нам понадобится тождество
\[
  ||x - \sum_{n = 1}^N \Inner{x, e_n} e_n||^2 =
  ||x||^2 - \sum_{n=1}^N |\Inner{x, e_n}|^2,
\]
которое, по сути, было установлено в доказательстве
неравенства Бесселя.

\begin{lemma}[минимальное свойство коэффициентов Фурье]
  Для произвольного \( x \in H \)
  \[
    \inf_{\alpha_1, \dots, \alpha_N} ||x - \sum_{n = 1}^N \alpha_n e_n|| =
    ||x - \sum_{n = 1}^N \Inner{x, e_n} e_n||.
  \]
\end{lemma}
\begin{proof}
  Поскольку справа от знака равенства записан
  частный случай минимизируемого выражения,
  равенство следует из того, что
  для произвольного набора коэффициентов
  \begin{align}
    ||x - \sum_{n = 1}^N \alpha_n e_n||^2 &=
    ||x||^2 - \sum_{n = 1}^N \overline{\alpha}_n \Inner{x, e_n} -
    \sum_{n = 1}^N \alpha_n \overline{\Inner{x, e_n}} +
    \sum_{n = 1}^N |\alpha_n|^2 = \\
    &=
    ||x||^2 - \sum_{n = 1}^N |\Inner{x, e_n}|^2 +
    \sum_{n = 1}^N |\alpha_n - \Inner{x, e_n}|^2 \ge
    ||x||^2 - \sum_{n = 1}^N |\Inner{x, e_n}|^2. \qedhere
  \end{align}
\end{proof}

\begin{itemproof}[Доказательство теоремы]
\item[\( 1 \To 2 \)]
  Очевидно, что если \( h \) представим в виде
  (бесконечной) линейной комбинации элементов \( e \), то
  \( h \in \Cl{[e]} \). Тогда по определению базиса
  \( H \subset \Cl{[e]} \), что и означает полноту \( e \).
\item[\( 2 \To 1 \)]
  Зафиксируем \( h \) и введём обозначение
  \[
    S_N = \sum_{n = 1}^N \Inner{h, e_n} e_n
  \]
  Раз \( e \) полна, \( h \in \Cl{[e]} \To \rho(h, [e]) = 0 \),
  т. е. для произвольного \( \epsilon > 0 \) найдётся
  \( y \in [e] \) такой, что \( ||h - y|| < \epsilon \).
  Выбрав достаточно большой \( N \) мы сможем представить
  \( y \) в виде
  \[
    y = \sum_{n=1}^N \alpha_n e_n,
  \]
  и тогда из минимальности коэффициентов Фурье получим
  \( ||h - S_N|| \le ||h - y|| < \epsilon \).
  Вследствие той же леммы, последовательность
  \( ||h - S_N|| \)
  не убывает, а потому
  \[
    ||h - S_N|| \to 0 \oTTo h = \sum_{n = 1}^\infty \Inner{h, e_n} e_n.
  \]

  Единственность представления следует из непрерывности
  скалярного произведения, ведь если 
  \( h = \sum \alpha_n e_n \), то для любого \( n \)
  \( \Inner{h, e_n} = \alpha_n  \).
\item[\( 2 \To 3 \)] \( e^\perp = \overline{[e]}^\perp = H^\perp = \{ 0 \} \)
\item[\( 3 \To 2 \)] \( H = \overline{[e]} \oplus \overline{[e]}^\perp =
  \overline{[e]} + e^\perp = \overline{[e]} \oplus \{ 0 \} = \overline{[e]} \).

\item[\( 1 \oTTo 4 \)]

  Выберем произвольный \( x \in H \) и разложим его по базису \( e \):
  \[
    x = \sum_{n=1}^\infty \alpha_n e_n.
  \]
  Ранее мы уже показали, что если
  \[
    x = \sum_{n=1}^\infty \alpha_n e_n,
  \]
  то обязательно верно \( \alpha_n = \Inner{x, e_n} \).
  А значит, благодаря вышеупомянутому тождеству,
  \( e \) "--- базис тогда и только тогда, когда
  для любого \( x \in H \)
  \[
    x = \sum \Inner{x, e_n} e_n
    \oTTo 
    0 = ||x - \sum \Inner{x, e_n} e_n||^2 =
    ||x||^2 - \sum |\Inner{x, e_n}|^2. \qedhere
  \]
\end{itemproof}

\begin{theorem*}
  В гильбертовом пространстве
  существование базиса эквивалентно сепарабельности.
\end{theorem*}

\begin{theorem*}
  Все сепарабельные гильбертовы пространства изоморфны.
\end{theorem*}

\begin{theorem*}[Рисса-Фишера]
  Пространства \( l_2 \) и \( L_2[0, 1] \) изоморфны.
\end{theorem*}

\begin{theorem*}
  Если \( H \) "--- гильбертово пространство и
  \( \{ e_n \} \subset H  \) "---
  ортонормированная система,
  то сходимость ряда \( \sum \alpha_n e_n \) в \( H \)
  эквивалентна сходимости числового ряда \( \sum |\alpha_n|^2 \).
\end{theorem*}
\begin{proof}
  Пусть
  \[ x = \sum_{n = 1} e_k|^\infty \alpha_n e_n, \]
  тогда \( \alpha_n = \Inner{x, e_n} \) и выполнив
  предельный переход в неравенстве Бесселя мы
  получаем
  \[
    \sum_{n = 1}^\infty |\alpha_n|^2 =
    \sum_{n = 1}^\infty |\Inner{x, e_n}|^2 \le
    ||x||^2 < +\infty.
  \]

  Пусть теперь сходится ряд
  \[
    \sum_{n=1}^\infty |\alpha_n|^2,
  \]
  тогда из критерия Коши сходимости числового ряда
  будет следовать фундаментальность
  последовательности частичных сумм
  \[
    S_n = \sum_{k=1}^n \alpha_k e_k,
  \]
  ведь
  \[
    ||S_{n + p} - S_{n}||^2 =
    ||\sum_{k=n+1}^{n+p} \alpha_k e_k||^2 =
    \sum_{k=n+1}^{n+p} |\alpha_k|^2. \qedhere
  \]
\end{proof}

\end{document}
