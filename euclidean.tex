\documentclass[main]{subfiles}

\begin{document}
\section{Эвклидовы пространства}
\begin{definition}
  Линейное пространство \( E \) (над \( F = \Real \) или \( F = \Complex \))
  является \emph{евклидовым}, если на нём задано отображение
  \( (\cdot, \cdot) : E \times E \to F \) такое, что
  \begin{enumerate}
    \item \( (x, x) \ge 0 \), \( (x, x) = 0 \oTTo x = 0 \)
    \item \( (x, y) = \overline{(y, x)} \)
    \item \( (\alpha x, y) = \alpha (x, y) \)
      (тогда верно и \( (x, \alpha y) = \overline{\alpha} (x, y) \))
    \item \( (x + y, z) = (x, z) + (y, z) \)
  \end{enumerate}
\end{definition}

Над евклидовом пространством можно ввести норму
по правилу \( ||x|| := ||x|| + ||y|| \).
Но чтобы доказать для неё неравенство треугольника
потребуется вспомогательное утверждение:

\begin{lemma}[Неравенство Коши-Буняковского-Шварца]
  \( |(x, y)| \le ||x|| \cdot ||y|| \)
\end{lemma}
\begin{proof}
  Для прозивольных \( x, y \in E \) и \( \alpha \in \Real \)
  \[ 0 \le (x + \alpha y, x + \alpha y) =
  (x, x) + 2 \alpha (x, y) + \alpha^2 (y, y). \]
  Т. к. коэфиициент при \( \alpha^2 \) неотрицателен,
  мы получаем, что \( D = (x, y)^2 - (x, x) \cdot (y, y) \le 0 \).
\end{proof}

\begin{exercise}
  Адаптировать доказательство для \( E \) над \( \Complex \).
\end{exercise}

\begin{theorem}[Неравентсво Бесселя]
  Пусть \( E \) "--- эвклидово, \( \{ e_k \}_1^n \) "--- ортонормированная система,
  \( x \in E \) и \( \alpha_k = (e_k, x) \). Тогда
  \[ ||x||^2 \ge \sum_{k = 1}^{n} |\alpha_k|^2. \]
\end{theorem}
\begin{proof}
  \[ 0 \le (x - (x, e) e, x - (x, e) e) =
    (x, x) - (x, e) (e, x) - \overline{(x, e)} (x, e) + (x, e)\overline{(x, e)}(e, e) =
  (x, x) - |(x, e)| = ||x||^2 - |\alpha_1|. \]
\end{proof}

Теперь докажем неравентсво треугольника, точнее, что
\( ||x + y||^2 \le (||x|| + ||y||)^2 \):
\( ||x + y||^2 = ||x||^2 + 2 \Re(x, y) + ||y||^2 \),
поэтому достаточно показать что \( 2 \Re(x, y) \le 2 ||x|| ||y|| \),
что следует из НКБ.

\begin{definition}
  Полное эвклидово пространство называется \emph{гильбертовым}.
\end{definition}

\begin{theorem}[б/д]
  Пусть \( E \) "--- ЛНП. Норма в \( E \) порождается некоторым
  скалярным произведением \( \oTTo \) выполняется равенство
  параллелограмма:
  \[ ||x + y||^2 + ||x - y||^2 = 2||x||^2 + 2||y||^2. \]
\end{theorem}

\begin{example}
  \( C[0, 1] \) "--- не эвклидово. Пусть \( f(x) = x \) и \( g(x) = 1 - x \).
  Тогда \( f + g = 1 \) и \( f - g = 2x - 1 \) и
  \[ ||f + g||^2 + ||f - g||^2 = 1^2 + 1^2 \ne 2 \cdot 1^2 + 2 \cdot 1^2 =
  2 ||f||^2 + 2||g||^2. \]
\end{example}

Идея докзательства проста для вещественного:
определим скалярное произведение как
\[ (x, y) = \frac{||x + y||^2 - ||x - y||^2}{4}. \]
В комплексном случае всё сложнее:
\[ (x, y) = \frac{1}{4} \sum_{\alpha \in \{ \pm 1, \pm i \}}
\alpha ||\alpha x + y||^2. \]

В конечномерном случае у любого подпространства есть прямое дополнение:
если \( E_1 < E \), тогда \( \Exists{E_2 < E} E_1 \oplus E_2 = E \).
А что происхдит в общем случае? С помощью понятия базиса Гамеля и аксиомы выбора
можно доказать это утверждение для линейных многообразий,
а для подпространств нормированных пространств ситуация интереснее.

\begin{problem}
  В \( C[0, 1] \) существует "<недополняемое"> подпространство.
\end{problem}

\begin{definition}
  Пусть \( E \) "--- эвклидово пространство, \( S \subset E \)
  тогда его \emph{аннулятором} называется
  \[ S^\perp = \{ y \mid \Forall{s \in S} (s, y) = 0 \}. \]
\end{definition}

\begin{definition}
  Пусть \( E \) "--- ЛНП, \( M \subset E \) "--- линейное многообразие,
  \( x \in E \). Тогда \emph{элементом наилучшего приближения} называется
  \( y \in E \) такой, что \( \rho(x, M) = ||x - y|| \).
\end{definition}

\begin{lemma}
  Пусть \( H \) "--- гильбертово пространство, \( M \subset H \) "---
  подпространство. Тогда
  \[ \Forall{h \in H} \ExistsOne{x \in M} \rho(h, M) = ||x - h||. \]
\end{lemma}
\begin{proof}
  Зафиксируем произвольный \( h \in H \) и обозначим \( d = \rho(h, M) \).

  Пусть \( x_1, x_2 \in M \) "--- ЭНП для \( h \), т. е.
  \( ||h - x_1|| = ||h - x_2|| = d \).
  Положим \( a = h - x_1 \) и \( b = h - x_2 \),
  тогда по равенству параллелограмма
  \[ ||2h - x_1 - x_2||^2 + ||x_1 - x_2||^2 =
  2||h - x_1||^2 + 2 ||h - x_2||^2 = 4 d^2, \]
  и тогда
  \[ ||x_1 - x_2||^2 = 4d^2 - 4 ||h - \frac{x_1 + x_2}{2}||
  \le 4d^2 - 4d^2 = 0, \]
  откуда \( x_1 = x_2 \), т. е. ЭНП единственен.

  Если \( d = 0 \), то, т. к. \( M \) "--- замкнуто,
  \( h \in M \), и ЭНП есть сам \( h \).

  Если \( d > 0 \), то существует последовательность
  \( \{ x_n \} \subset M \) такая, что
  \( ||h - x_n|| \to d \). Тогда если мы докажем,
  что \( x_n \to x \), то это и будет наилучшее приблежение.
  Т. к. \( H \) "--- полно, достаточно показать фундаментальность
  \( \{ x_n \} \). Для произвольного \( \epsilon > 0 \)
  с некотого \( N \) \( ||h - x_m||^2 \le d^2 + \epsilon \),
  и тогда
  \[ ||x_n - x_m||^2 = -4 ||h - \frac{x_n + x_m}{2}||^2
    + 2 ||h - x_m||^2 + 2||h - x_m||^2 \le
    -4 d^2 + 2d^2 + 2\epsilon + 2d^2 + 2\epsilon =
    4 \epsilon. \]
\end{proof}

% Энфло 1973: может не быть базиса в сеп. БП

\begin{theorem}[о проекции]
  Пусть \( H \) "--- гильбретово пространство, \( M \subset H \) "---
  подпространство. Тогда \( H = M \oplus M^\perp \),
  причём \( M^\perp \) "--- подпространство.
\end{theorem}
\begin{proof}
  Докажем для вещественного случая.

  Покажем, что произвольный элемент \( h \in H \) разложим
  в сумму элементов \( M \) и \( M^\perp \).
  Рассмотрим \( y = h - x \), где \( x \) "--- ЭНП для \( H \).
  Покажем, что \( y \in M^\perp \), т. е. \( \Forall{m \in M} (m, y) = 0 \).
  Для произвольного \( m \in M \) и \( \alpha \)
  \[ d^2 = ||h - x||^2 \le ||h - x - \alpha m||^2 =
  d^2 - 2 \alpha (y, m) + \alpha^2 ||m||^2 \To
  2 \alpha (y, m) \le \alpha^2 ||m||^2. \]
  Фиксируя \( \alpha = \epsilon > 0 \), получаем \( 2 (y, m) \le \epsilon ||m||^2 \),
  при \( \alpha = -\epsilon \) "--- \( 2 (y, m) \ge -\epsilon ||m||^2 \),
  т. е. \( 2 |(y, m)| \le \epsilon ||m||^2 \) для произвольного \( \epsilon > 0 \).
  Отсюда получаем, что \( (y, m) = 0 \), т. е. \( y \in M^\perp \)
  и он \( h \) разложим.

  Единственность разложения следует из того, что \( M \cap M^\perp = \{ 0 \} \).
\end{proof}

\begin{exercise}
  Для произвольного \( S \subset H \) \( S^\perp \) "--- замкнутое множество,
  и \( S^\perp = {[S]}^\perp = \overline{[S]}^\perp \).
\end{exercise}

\begin{definition}
  Система \( e \subset H \) полна, если \( \overline{[e]} = H \).
\end{definition}

\begin{theorem}
  Пусть \( H \) "--- гильбертово пространство,
  \( e = \{ e_n \}_1^\infty \subset H \) "--- ОНС.
  Тогда следующие свойства эквивалентны:
  \begin{enumerate}
    \item \( e \) "--- базис
    \item \( e \) "--- полная система
    \item \( e^\perp = \{ 0 \} \)
    \item \( \Forall{x \in t} ||x||^2 = \sum |(x, e_n)|^2 \)
  \end{enumerate}
\end{theorem}

Для доказательства будем пользоваться следующим тождеством:
\[ ||x - \sum_{k = 1}^N (x, e_n) e_n||^2 = ||x||^2 - \sum_{k=1}^N |(x, e_n)|^2, \]
которое верно т. к.
\[ (x - \sum (x, e_n) e_n, x - \sum (x, e_n) e_n) =
  ||x||^2 - \sum \overline{(x, e_n)} (x, e_n) -
  \sum (x, e_n) (e_n, x) + \sum (x, e_n)\overline{(x, e_n)} =
||x||^2 - \sum |(x, e_n)|^2. \]

\begin{lemma}[минимальное свойство коэффициентов Фурье]
  \[ \inf_{\{ \alpha_n \}} ||x - \sum_{k = 1}^n \alpha_k e_n|| =
  ||x - \sum (x, e_n) e_n||. \]
\end{lemma}

\begin{lemma}[Неравенство Бесселя]
  \( ||x||^2 \ge \sum |(x, e_n)|^2 \).
\end{lemma}
\begin{proof}
  Напрямую следует из тождества, т. к. норма любого элемента неотрицательна.
\end{proof}

\begin{itemproof}%[Доказательство теоремы]
\item[\( 2 \To 3 \)] \( e^\perp = \overline{[e]}^\perp = H^\perp = \{ 0 \} \)
\item[\( 3 \To 2 \)] \( H = \overline{[e]} \oplus \overline{[e]}^\perp =
  \overline{[e]} + e^\perp = \overline{[e]} \oplus \{ 0 \} = \overline{[e]} \).
\end{itemproof}

\begin{theorem*}
  В гильбертовом пространстве есть базис \( \oTTo \) оно сепарабельно.
\end{theorem*}

\begin{theorem*}
  Все сепарабельные гильбертовы пространства изоморфны.
\end{theorem*}

\begin{theorem*}[Рисса-Фишера]
  Если \( H \) "--- гильбертово пространство, \( \{ e_n \} \) "--- ОНС.
  Тогда ряд \( \sum \alpha_n e_n \) сходится \( \oTTo  \sum |\alpha_n|^2 < \infty \).
\end{theorem*}

\end{document}
