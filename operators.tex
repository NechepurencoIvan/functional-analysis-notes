\documentclass[main]{subfiles}

\begin{document}
\section{Линейные ограниченные операторы}

В данном параграфе \( E \), \( E_1 \) и \( E_2 \) "---
линейные нормированные пространства над
полем \( F \).

\begin{definition}
  \emph{Оператором} называется отображение
  \( A : E_1 \to E_2 \), а \emph{функционалом} "---
  отображение \( f : E \to F \).
\end{definition}

\begin{definition}
  Оператор \( A : E_1 \to E_2 \) называется \emph{линейным},
  если
  \[
    \Forall{x_1, x_2 \in E_1}
    \Forall{\alpha_1, \alpha_2 \in F}
    A(\alpha_1 x_1 + \alpha_2 x_2) =
    \alpha_1 A(x_1) + \alpha_2 A(x_2).
  \]
\end{definition}

\begin{definition}
  Оператор \( A : E_1 \to E_2 \)
  \emph{ограничен},
  если он переводит ограниченные
  множества в ограниченные.
\end{definition}

\begin{proposition}
  Пусть линейный оператор \( A : E_1 \to E_2 \)
  непрерывен в точке \( x_0 \in E_1 \),
  тогда \( A \) непрерывен.
\end{proposition}
\begin{proof}
  Пусть \( x_n \to x \), тогда \( x_n - x + x_0 \to x_0 \),
  а значит,
  \[ A x_n - A x + A x_0 = A(x_n - x + x_0) \to A x_0, \]
  откуда \( A x_n \to A x \).
\end{proof}

\begin{proposition}
  Пусть оператор \( A : E_1 \to E_2 \) линеен.
  Тогда \( A \) ограничен \( \oTTo \) множество
  \( A(\Cl{B}(0, 1)) \) ограниченно в \( E_2 \).
\end{proposition}

\begin{corollary}
  Ограниченность линейного оператора
  \( A : E_1 \to E_2 \) эквивалентна
  тому, что
  \[
    \Exists M \Forall{x \in E_1} ||x|| \le 1 \To ||Ax|| \le M,
  \]
  т. е. \( A(\Cl{B}(0, 1)) \subset \Cl{B}(0, M) \).
\end{corollary}

\begin{definition}
  \emph{Нормой} линейного оператора
  \( A : E_1 \to E_2 \) называется
  \[
    ||A|| := \inf \{
      M \mid \Forall{x \in E_1} ||Ax|| \le M ||x||
    \}.
  \]
\end{definition}

\begin{exercise}
  В определении нормы линейного оператора инфимум достигается.
\end{exercise}

\begin{lemma}
  \[ ||A|| = \sup_{||x|| < 1} ||A(x)|| = \sup_{||x|| = 1} ||A(x)|| =
  \sup_{x \ne 0} \frac{||A(x)||}{||x||}. \]
\end{lemma}
\begin{corollary}
  Линейный оператор \( A : E_1 \to E_2 \)
  ограничен \( \oTTo \) \( ||A|| < \infty \).
\end{corollary}

\begin{theorem}
  Пусть \( A : E_1 \to E_2 \) "---
  линейный оператор. Тогда \( A \) непрерывен \( \oTTo \)
  \( A \) ограничен.
\end{theorem}
\begin{itemproof}
  \item[\(\oT\)]
    Раз \( A \) ограничен, \( ||A|| < +\infty \).
    Пусть \( x_n \to x_0 \), т. е. \( ||x_n - x_0|| \to 0 \),
    тогда
    \[
      ||A x_n - A x_0|| = ||A(x_n - x_0)|| \le ||A|| ||x_n - x_0|| \to 0,
    \]
    т. е. \( A x_n \to A x_0 \).
  \item[$\To$]
    Предположим противное: \( A \) непрерывен, но не ограничен.
    Тогда \( \Forall{n} \Exists{x_n} ||x_n|| = 1, ||A x_n|| > n \)
    (\( \To ||A(x_n)|| > n \)).
    Рассмотрим \( y_n = \frac{x_n}{n} \):
    \( y_n \to 0 \), но \( ||A y_n|| = \frac{||A x_n||}{n} > 1 \),
    а потому \( A y_n \not \to 0 = A 0 \) "--- противоречие.
\end{itemproof}

\begin{theorem}
  Обозначим через \( \Linears{E_1}[E_2] \)
  множество линейных ограниченных операторов
  \( E_1 \to E_2 \).
  Тогда это линейное пространство, если определить
  \[
    (\alpha_1 A_1 + \alpha_2 A_2) x \coloneqq
    \alpha_1 A_1 x + \alpha_2 A_2 x,
  \]
  и норма оператора "--- норма в этом пространстве.
  Кроме того, если \( E_2 \) "--- банахово, то
  \( \mathcal{L}(E_1, E_2) \) "--- тоже банахово.
\end{theorem}
\begin{proof}
  Линейность "--- очевидна, единственный нетривиальный
  момент в проверке корректности нормы "--- неравенство
  треугольника:
  \[
    \sup_{||x|| \le 1} ||(A_1 + A_2) x|| =
    \sup_{||x|| \le 1} ||A_1 x + A_2 x|| \le
    \sup_{||x|| \le 1} \left( ||A_1 x|| + ||A_2 x|| \right) \le
    \sup_{||x|| \le 1} ||A_1(x)|| + \sup_{||x|| \le 1} ||A_2(x)||,
  \]
  т. е. \( ||A_1 + A_2|| \le ||A_1|| + ||A_2|| \).

  %Если \( ||A_n - A|| \to 0 \), то
  %\( \Forall{x \in E_1} ||A_n(x) - A(x)|| \to 0 \) "---
  %поточечная сходимость следует из сходимости по норме
  %(обратное, вообще говоря, неверно).
  Пусть теперь \( E_2 \) "--- банахово.
  Докажем полноту \( \Linears{E_1}[E_2] \) по такой схеме:
  для произвольной фундаментальной последовательности
  найдём её поточечный предел,
  а потом докажем, что это линейный ограниченный оператор,
  и что сходимость выполняется и по норме.
  
  Пусть \( \{ A_n \} \) "--- фундаментальная последовательность
  в \( \Linears{E_1}[E_2] \):
  \[ \Forall{\epsilon} \Exists{N} \Forall{n, m \ge N} ||A_n - A_m|| < \epsilon. \]
  Тогда для произвольного \( x \in E_1 \)
  и фиксированных \( \epsilon \) и \( N \)
  \[
    \Forall{n, m \ge N} ||A_n x - A_m x|| \le ||A_n - A_m|| \cdot ||x|| < \epsilon ||x||.
  \]
  Т. к. \( ||x|| \) "--- константа, то
  последовательность \( \{ A x_n \} \) фундаментальна в \( L_2 \),
  а потому сходится.
  Положим
  \[ A(x) = \lim_{n \to \infty} A_n x. \]

  Покажем, что \( A \) "--- линейный ограниченный оператор.
  Линейность сразу следует из свойств предела:
  \[
    A(\alpha_1 x_1 + \alpha_2 x_2) \ot
    A_n(\alpha_1 x_1 + \alpha_2 x_2) =
    \alpha_1 A_n x_1 + \alpha_2 A_n x_2 \to
    \alpha_1 A x_1 + \alpha_2 A x_2,
  \]
  Раз \( \{ A_n \} \) фундаментальная, то она ограниченна,
  т. е. \( \Exists{M} \Forall{n} ||A_n|| \le M \).
  Тогда для произвольного \( x \)
  \( ||A_n x|| \le ||A_n||||x|| \le M ||x|| \),
  откуда \( ||A x|| \le M ||x|| \),
  т. е. \( A \) ограничен.

  Наконец, покажем, что \( A_n \to A \).
  Зафиксируем \( \epsilon > 0 \) и выберем
  \( N \) такой, что
  \(  \Forall{n, m \ge N} ||A_n - A_m|| < \epsilon \).
  Если \( n, m \ge N \), то для произвольного
  \( x \in E_1 \)
  \[ ||(A_n - A_m) x|| \le ||A_n - A_m|| \cdot ||x|| < \epsilon ||x||, \]
  и устремив \( m \) к бесконечности при
  фиксированном \( n \) мы получаем
  \( ||(A_n - A) x|| \le \epsilon ||x|| \).
  Значит, \( ||A_n - A|| \le \epsilon \),
  т. е.
  \[
    \Forall{\epsilon > 0}
    \Exists{N}
    \Forall{n \ge N} ||A_n - A||
    \le \epsilon
    \oTTo
    ||A_n - A|| \to 0.
  \]
\end{proof}

\begin{corollary}
  Два важных частных случая:
  \begin{enumerate}
    \item Если \( E \) "--- банахово пространство, то \( \Linears{E} \) "---
      банахово.
    \item  \( E^* := \Linears{E}[F] \) "--- банахово.
  \end{enumerate}
\end{corollary}

\begin{theorem}\label{thm:operator-continuation}
  Пусть \( E_2 \) "--- банахово пространство,
  \( D(A) \) "--- всюду плотное линейное многообразие
  в \( E_1 \), \( A \in \Linears{D(A)}[E_2] \).
  Тогда 
  \[
    \ExistsOne{\widetilde{A} \in \Linears{E_1}[E_2]}
    {\widetilde{A} \bigr|}_{D(A)} = A,
  \]
  и при этом \( ||\widetilde{A}|| = ||A|| \).
\end{theorem}
\begin{proof}
  Предположим, что у нас есть продолжение
  \( \widetilde{A} \in \mathcal{L}(E_1, E_2) \).
  Тогда
  \[
    \Forall{x \in \Cl{D(A)} = E_1}
    \Exists{ \{ x_n \} \subset D(A) }
    x_n \to x,
  \]
  и по непрерывности \( A \)
  \[
    A x_n = \widetilde{A} x_n \to \widetilde{A} x,
  \]
  т. е. продолжение определено единственным образом.

  Теперь докажем существование продолжения.
  Зафиксируем \( x \in E_1 \) и выберем
  последовательность \( \{ x_n \} \subset D(A) \)
  такую, что \( x_n \to x \).
  Так как \( A \) ограничен, из фундаментальности
  \( \{ x_n \} \) следует фундаментальность
  \( \{ A x_n \} \), и, раз \( E_2 \) банахово,
  мы можем положить
  \[
    \widetilde{A} = \lim_{n \to \infty} A x_n.
  \]
  Это определение корректно, ведь если
  \( \{ x^1_n \} \) и \( \{ x^2_n \} \)
  сходятся к \( x \), то
  и последовательность
  \[
    x_n = \begin{cases}
      x^1_k, & n = 2k - 1 \\
      x^2_k, & n = 2k
    \end{cases}
  \]
  сходится к \( x \), а потому
  \[
    \lim_{n \to \infty} A x^1_n =
    \lim_{n \to \infty} A x_n = 
    \lim_{n \to \infty} A x^2_n.
  \]

  Линейность \( \widetilde{A} \) очевидна,
  а ограниченность следует из того,
  что если \( \{ x_n \} \subset D(A) \)
  сходится к \( x \),
  то
  \[
    ||\widetilde{A} x|| \ot
    ||A x_n|| \le
    ||A|| \cdot ||x_n|| \to
    ||A|| \cdot ||x||,
  \]
  т. е. \( ||\widetilde{A}|| \le ||A|| \).

  Наконец, заметим, что
  \[
    ||\widetilde{A}|| =
    \sup_{||x|| = 1} ||\widetilde{A} x|| \ge
    \sup_{x \in D(A), ||x|| = 1} ||\widetilde{A} x|| =
    \sup_{x \in D(A), ||x|| = 1} ||A x|| =
    ||A||
  \]
  откуда \( ||A|| = ||\widetilde{A}|| \).
\end{proof}

\begin{theorem}[Банах--Штейнгауз, принцип равномерной ограниченности]\label{thm:operators-bsh}
  Пусть \( E_1 \) "--- банахово пространство,
  \( \{ A_n \} \subset \Linears{E_1}[E_2] \).
  Тогда
  \[
    \Forall{x} \sup_n ||A_n x || < \infty
    \To
    \sup_n ||A_n|| < \infty.
  \]
\end{theorem}
\begin{proof}
  Докажем сначала, что
  если \( \{ A_n \} \) равномерно ограничена
  на некотором шаре
  (т. е. \( \Exists{\Cl{B}(x_0, r_0)} \Exists{M}
  \Forall{x \in \Cl{B}} \Forall{n} ||A_n x|| \le M \)),
  то \( \sup ||A_n|| < \infty \).
  Для произвольного \( 0 \ne x \in E_1 \) положим
  \[
    y = x_0 + r_0 \frac{x}{||x||} \in \Cl{B}(x_0, r_0),
  \]
  тогда
  \[
    ||A_n x|| = \frac{||x||}{r_0} ||A_n (y - x_0)|| \le
    \frac{||x||}{r_0} (||A_n y|| + ||A_n x_0||) \le
    \frac{2 M}{r_0} ||x||,
  \]
  т. е. \( ||A_n|| \le \frac{2M}{r_0} \).

  Теперь достаточно показать, что на некотором
  шаре \( \{ A_n \} \) равномерно ограничена.
  Предположим, что это не так. Выберем произвольные
  \( x_0 \in E_1 \) и \( r_0 > 0 \).
  Применив предположение к \( \Cl{B}(x_0, \frac{r_0}{2}) \)
  и \( M = 1 \) мы найдём \( n_1 \) и \( x_1 \) такие,
  что \( ||A_{n_1} x_1|| > 1 \).
  В силу непрерывности \( A_{n_1} \)
  найдётся также число \( r_1 \in (0, \frac{r_0}{2}) \)
  такое, что \( \Forall{x \in \Cl{B}(x_1, r_1)} ||A_{n_1} x|| > 1 \).
  Заметим, что тогда \( \Cl{B}(x_1, r_1) \subset \Cl{B}(x_0, r_0) \).
  Повторяя эти действия мы получим последовательность
  вложенных шаров \( \Cl{B}_k = \Cl{B}(x_k, r_k) \), где
  \( r_{k + 1} < \frac{r_k}{2} \), откуда \( r_k \to 0 \),
  и \( \Forall{k} \Forall{x \in \Cl{B}_k} ||A_{n_k} x|| > k \).

  Тогда по теореме~\ref{thm:complete-balls}
  найдётся \( x \in \bigcap \Cl{B}_k \),
  но
  \[
    \sup_n ||A_n x|| \ge \sup_k ||A_{n_k} x|| = +\infty,
  \]
  что противоречит условию поточечной ограниченности.
\end{proof}

\begin{theorem}[полнота в смысле поточечной сходимости]
  Пусть \( E_1 \), \( E_2 \) "--- банаховы пространства,
  \( \{ A_n \} \subset \Linears{E_1}[E_2] \).
  Тогда если для любого \( x \in E_1 \)
  последовательность \( \{ A_n x \} \) фундаментальна,
  то существует \( A \in \Linears{E_1}[E_2] \) такой,
  что \( \Forall{x \in E_1} A x_n \to A x \).
\end{theorem}
\begin{proof}
  Поскольку \( E_2 \) банахово,
  для произвольного \( x \in E_1 \)
  мы можем определить
  \[
    Ax = \lim_{n \to \infty} A_n x.
  \]
  Линейность \( A \) напрямую следует
  из линейность операторов \( \{ A_n \} \);
  из фундаментальности последовательности \( \{ A_n x \} \)
  следует её ограниченность,
  тогда по теореме Банаха--Штейнгауза
  \[
    M = \sup_n ||A_n|| < +\infty,
  \]
  и
  \[
    ||A x|| =
    \lim_{n \to \infty} ||A_n x|| \le
    \lim_{n \to \infty} ||A_n|| \cdot ||x|| \le
    M ||x||,
  \]
  т. е. \( ||A|| \le M < +\infty \).
\end{proof}

\begin{theorem}[Критерий поточечной сходимости последовательности
  линейных ограниченных операторов] % 5.6
  Пусть \( E_1 \), \( E_2 \) "--- банаховы пространства,
  \( \{ A_n \} \subset \Linears{E_1}[E_2] \),
  \( A \in \Linears{E_1}[E_2] \).
  Тогда последовательность \( \{ A_n \} \) сходится
  к \( A \) поточечно \( \oTTo \)
  \( \{ ||A_n|| \} \) ограниченно и \( \Forall{x \in S}
  A_n x \to A x \),
  где \( S \) "--- полное в \( E_1 \) множество.
\end{theorem}
\begin{itemproof}
  \item[\( \To \)]
    Для произвольного \( x \in E_1 \)
    из сходимости последовательности
    \( \{ A_n x \} \) следует её ограниченность,
    а потому по теореме~\ref{thm:operators-bsh}
    ограниченна и \( \{ ||A_n|| \} \).
    Кроме того, можно положить \( S = E_1 \).
  \item[\( \oT \)]
    Очевидно, из поточечной сходимости
    на \( S \) следует поточечная сходимость
    на \( [S] \).
    Для произвольного \( x \in E_1 \) и \( \epsilon > 0 \)
    можно выбрать \( y \in [S] \) такой,
    что \( ||x - y|| < \epsilon \);
    положим \( M = \sup ||A_n|| \).
    \( A_n y \to A y \), а потому
    \( \Exists{N} \Forall{n \ge N} ||A_n y - A y|| < \epsilon \),
    тогда для любого \( n \ge N \)
    \begin{align}
      ||A_n x - A x|| &\le
      ||A_n x - A_n y|| + ||A_n y - A y|| + ||A y - A x|| \le \\
      &\le ||A_n y - A y|| + (||A_n|| + ||A||) ||x - y|| < \\
      &< \epsilon + (M  + ||A||) \epsilon =
      (1 + M + ||A||) \epsilon,
    \end{align}
    т. е. для \( C = 1 + M + ||A|| \)
    \( \Forall{\epsilon > 0} \Exists{N}
    \Forall{n \ge N} ||A_n x - A x|| < C \epsilon \),
    откуда \( A_n x \to A x \).
\end{itemproof}

\end{document}
