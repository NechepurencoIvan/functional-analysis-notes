\documentclass[main]{subfiles}

\begin{document}

\section{Нормированные пространства}
\begin{definition}
  Линейным \emph{нормированным пространством} называется
  линейное пространство \( E \) над полем \( F \),
  где \( F = \Real \) или \( F = \Complex \),
  снабжённое \emph{нормой}, т. е. отображением \( ||\cdot|| : E \to \Real \)
  таким, что для произвольных \( x, y \in E \) и \( \alpha \in F \)
  \begin{enumerate}
    \item \( ||x|| \ge 0 \), \( ||x|| = 0 \oTTo x = 0 \)
    \item \( ||\alpha x|| = |\alpha| ||x|| \) (однородность)
    \item \( ||x + y|| \le ||x|| + ||y|| \) (неравенство треугольника)
  \end{enumerate}
\end{definition}

\begin{example}
  \( \Real^n_{p} \), где \( p \ge 1 \), с нормой
  \( ||x||_p = {\left(\sum_{k=1}^n |x_k|^p \right)}^{1/p} \).
\end{example}
\begin{example}
  \( \Real^n_\infty \) с нормой \( ||x||_\infty = \max_{k} |x_k| \).
\end{example}
%\begin{example}
%  \( l_\infty \).
%\end{example}
\begin{example}
  Для произвольного \( X \) можно рассматривать
  \( B(X) \) "--- множество ограниченных числовых функций
  с нормой \( ||f||_\infty = \sup_{x \in X} |f(x)| \).
  %Заметим: \( l_\infty = B(\Natural) \).
\end{example}

% C_p[a, b] (интегральная), C[a, b]

\begin{remark}
  Если \( (E, ||\cdot||) \) "--- линейное нормированное пространство,
  то \( (E, \rho) \), где \( \rho(x, y) = ||x - y|| \), "---
  метрическое пространство.
\end{remark}

\begin{definition}
  Нормированное пространство называется \emph{банаховым},
  если порождённое им метрическое пространство
  полно.
\end{definition}

\begin{definition}
  Пусть \( M \subset E \). Если \( M \) замкнуто относительно
  линейных операций, то \( M \) называется \emph{линейным многообразием}.
  Если \( M \) также замкнуто (топологически),
  то оно называется \emph{подпространством}.
\end{definition}

\begin{definition}
  Для произвольного множества \( S \subset E \) определим
  \emph{линейную оболочку} \( [S] \)
  как множество всевозможных линейных комбинаций
  конечного числа элементов \( S \).
\end{definition}

\begin{example}
  \( E = C[a, b] \), \( M = [1, x, \dots, x^n, \dots] \) "--- многочлены.
  Тогда \( M \) "--- линейное многообразие, но не подпространство,
  т. к. по теореме Вейерштрасса \( \Cl M  = E \ne M \).
\end{example}

\begin{definition}
  Система векторов \( {\{ e_n \}}_{n = 1}^\infty \subset E \)
  называется \emph{(счётным) базисом}
  в нормированном пространстве \( E \), если
  \[
    \Forall{x \in E}
    \ExistsOne{ {\{ \alpha_n \}}_{n = 1}^\infty }
    \sum_{n = 1}^\infty \alpha_n e_n = x,
  \]
  где сумма ряда понимается как
  предел последовательности частичных сумм
  \[ S_N = \sum_{n = 1}^N \alpha_n e_n \]
  по норме.
\end{definition}

\begin{remark}
  Любой базис линейно независим, т. к.
  иначе \( 0 \in E \) будет иметь более одного
  представления.
\end{remark}

\begin{definition}
  Нормы \( ||\cdot||_1 \) и \( ||\cdot||_2 \),
  заданные на линейном пространстве \( E \),
  называются \emph{эквивалентными}
  (\( ||\cdot||_1 \sim ||\cdot||_2 \)),
  если
  \[
    \Exists{C_1, C_2 > 0}
    \Forall{x \in E}
    C_1 ||x||_2 \le ||x||_1 \le C_2 ||x||_2.
  \]
  \( ||\cdot||_1 \) \emph{слабее}, чем \( ||\cdot||_2 \),
  если 
  \[
    \Exists{C > 0}
    \Forall{x \in E}
    ||x||_1 \le C ||x||_2.
  \]
\end{definition}
\begin{remark}
  Если \( ||\cdot||_1 \) \emph{слабее}, чем \( ||\cdot||_2 \),
  то \( x_n \xrightarrow[||\cdot||_2]{} x \)
  влечёт \( x_n \xrightarrow[||\cdot||_1]{} x \),
  т. е. \( ||\cdot||_1 \) непрерывна
  относительно \( ||\cdot||_2 \).
\end{remark}

\begin{exercise}
  \( y_m \to y \To ||y_m|| \to ||y|| \).
\end{exercise}

\begin{proposition}
  Пусть \( E \) "--- ЛП, \( \dim E < \infty \). Тогда все нормы на
  \( E \) эквивалентны.
\end{proposition}
\begin{proof}
  Раз \( E \) конечномерно, можно выбрать базис
  \( e = \{ e_1, \dots, e_n \} \), т. е.
  \[ \Forall{x \in E} x = \sum_{k = 1}^n \xi_k e_k. \]
  Введём на \( E \) скалярное произведение, в котором
  \( e \) ортонормирован:
  \[ \Inner{x, y} = \sum_{k = 1}^n \xi_k \overline{\zeta_k}, \]
  оно порождает норму \( ||x||_2 = \sqrt{\Inner{x, x}} \).
  Теперь выберем произвольную норму \( ||\cdot|| \) на \( E \)
  и покажем, что она эквивалентна \( ||\cdot||_2 \);
  если это так, то и две произвольные нормы эквивалентны.

  Покажем, что \( ||\cdot|| \) слабее \( ||\cdot||_2 \):
  \[
    ||x|| = ||\sum \xi_k e_k|| \le
    \sum ||\xi_k e_k|| = \sum |\xi_k| ||e_k||,
  \]
  а потому для \( C_2 = n \max_k ||e_k|| \)
  \[
    ||x|| \le C_2 \sqrt{\sum |\xi_k|^2} = C_2 ||x||_2,
  \]
  ведь \( {\left( \sum |\xi_k| \right)}^2 \le \sum n^2 |\xi_k|^2 \).

  Покажем теперь, что \( ||\cdot||_2 \) слабее, чем \( ||\cdot|| \).
  Пусть это не так, тогда
  \[
    \Forall{m} \Exists{x_m} \frac{1}{m} ||x_m||_2 \ge ||x_m||.
  \]
  Рассмотрим последовательность 
  \[ y_m = \frac{x_m}{||x_m||_2} \in S_{||\cdot||_2}(0, 1). \]
  \( S_{||\cdot||_2}(0, 1) \) "--- компакт,
  а потому существует сходящаяся подпоследовательность;
  без ограничения общности можем считать,
  что это все элементы: \( ||y_m - y||_2 \to 0 \),
  откуда \( ||y_m||_2 \to ||y||_2 \).
  Кроме того,
  т. к. \( ||\cdot|| \) слабее \( ||\cdot||_2 \),
  то выполняется и \( ||y_m - y|| \to 0 \To ||y_m|| \to ||y|| \).
  При этом,
  \[ ||y_m|| = \frac{||x_m||}{||x_m||_2} \le \frac{1}{m} \to 0, \]
  и тогда \( ||y_m|| \to 0 = ||y|| \).
  Значит, \( y = 0 \To ||y||_2 = 0 \ne 1 \) "---
  противоречие.
\end{proof}

\begin{lemma}[о "<почти"> перпендикуляре]
  Пусть \( E \) "--- линейное нормированное пространство,
  \( M \) "--- подпространство \( E \),
  \( M \ne E \).
  Тогда
  \[
    \Forall{\epsilon > 0}
    \Exists{y \in S(0, 1)}
    \rho(y, M) > 1 - \epsilon,
  \]
  где
  \[
    \rho(y, M) := \inf_{x \in M} \rho(y, x).
  \]
\end{lemma}
\begin{proof}
  Выберем произвольный \( y_0 \in E \setminus M \),
  обозначим \[ d = \rho(y_0, M) = \inf_{z \in M} ||y - z||. \]
  Т. к. \( M \) замкнуто, \( d > 0 \), а потому
  \[ \Forall{\epsilon > 0} \Exists{z_0 \in M} ||y_0 - z_0|| <
  d \cdot \frac{1}{1 - \epsilon}. \]
  Для фиксированного \( \epsilon > 0 \)
  положим
  \[ y = \frac{y_0 - z_0}{||y_0 - z_0||}, \]
  тогда для произвольного \( z \in M \)
  \[
    ||y - z|| =
    ||\frac{y_0 - z_0}{||y_0 - z_0||} - z|| =
    \frac{
      ||y_0 - (z_0 + ||y_0 - z_0|| z)||
    }{||y_0 - z_0||}
  \ge \frac{d}{||y_0 - z_0||} > 1 - \epsilon, \]
  ведь \( z_0 + ||y_0 - z_0|| z \in M \).
\end{proof}

\begin{theorem}\label{thm:4.1}
  Пусть \( E \) "--- линейное нормированное пространство,
  \( \{ x_1, \dots, x_n \} \subset E \).
  Тогда линейная оболочка \( [x_1, \dots, x_n] \) замкнута в топологическом
  смысле, и, соответственно, является подпространством.
\end{theorem}
\begin{exercise}
  Докажите теорему~\ref{thm:4.1}.
  Подсказка: воспользуйтесь тем, что в конечномерном пространстве
  все нормы эквивалентны.
\end{exercise}

\begin{theorem}[Ф. Рисс]\label{thm:sphere-compactness}
  Пусть \( E \) "--- нормированное пространство,
  \( \dim E = \infty \).
  Тогда единичная сфера \( S(0, 1) \subset E \)
  не является компактом; более того,
  она даже не вполне ограниченна.
\end{theorem}
\begin{proof}
  Зафиксируем произвольный \( \epsilon \in (0, 1) \).
  Выберем \( y_1 \in S(0, 1) \);
  т. к. \( \dim E = \infty \),
  \( M_1 := [y_1] = \ne E \).
  Тогда по лемме о почти перпендикуляре
  мы можем выбрать \( y_2 \in S(0, 1) \) такой,
  что \( \rho(y_2, M_1) \ge 1 - \epsilon \),
  и, опять же, \( M_2 := [y_1, y_2] \ne E \).
  Продолжая этот процесс, получим
  последовательность векторов
  \( {\{ y_n \}}_{n = 1}^\infty \subset S(0, 1) \)
  такую, что
  \( \rho(y_n, y_m) \ge 1 - \epsilon \),
  но тогда \( S(0, 1) \) нельзя покрыть,
  например,
  конечной \( \frac{1 - \epsilon}3 \)-сетью.
\end{proof}

\begin{remark}
  Вновь воспользовавшись эквивалентностью норм
  в конечномерных нормированных пространствах
  легко видеть, что
  компактность единичной сферы эквивалентна
  конечномерности пространства.
\end{remark}

\end{document}
