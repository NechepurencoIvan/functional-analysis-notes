\documentclass[main]{subfiles}

\begin{document}
\section{Сопряжённые операторы} % 11

Пусть задано два линейных нормированных пространства
$E_1$ и $E_2$ над $\Complex$; нас так же будут
интересовать их сопряжённые пространства
$E_1^*$ и $E_2^*$.
Пусть так же зафиксирован оператор $A \in \mathcal{L}(E_1, E_2)$,
тогда существует естественный способ определить
отображение $A^* : E_2^* \to E_1^*$:
\[
  g \mapsto g \circ A
\]
благодрая тому, что композиция сохраняет линейность
и непрерывность.

\begin{definition}
  \( A^* \) называется \emph{сопряжённым} к \( A \) оператором.
\end{definition}

\begin{theorem}%11.1
  Пусть $E_1$, $E_2$ "--- ЛНП над $\Complex$,
  $A \in \mathcal{L}(E_1, E_2)$.
  Тогда $A^* \in \mathcal{L}(E_2^*, E_1^*)$
  и $||A^*|| = ||A|$.
\end{theorem}
\begin{proof}
  Линейность $A^*$ очевидна из равенства
  $(A^* g)(x) = g(Ax)$. Более того,
  \[
    |(A^* g)(x)| = |g(Ax)| \le
    ||g|| \cdot ||Ax|| \le ||g|| \cdot ||A|| \cdot ||x||,
  \]
  т. е. $||A^* g|| \le ||A|| \cdot ||g||$.
  Значит, $A^*$ непрерывен и, более того,
  $||A^*|| \le ||A||$.

  Осталось доказать, что $||A^*|| \ge ||A||$.
  Действительно, по следствию из теоремы Хана-Банаха для произвольного $x \in E_1$
  \[
    ||A x|| = \sup_{||g||_{E_2^*} = 1} |g(Ax)| =
    \sup_{||g|| = 1} |(A^* g)(x)| \le
    \sup_{||g|| = 1} ||A^* п|| \cdot ||x|| \le
    \sup_{||g|| = 1} ||A^*|| \cdot ||g|| \cdot ||x|| =
    ||A^*|| \cdot ||x||.
  \]
\end{proof}

В гильбертовых пространствах такие берём теорему
Рисса-Фреше и получаем $A^* \in \mathcal{L}(H_2, H_1)$.

\begin{exercise}[без Т11.1]
  Пусть $H$ "--- гильбертово пространство,
  $A \in \mathcal{L}(H)$.
  Тогда существует $A^* \in \mathcal{L}(H)$
  такой, что
  \( \Forall{x, y \in H} (Ax, y) = (x, Ay) \)
  и $||A^*|| = ||A||$.
\end{exercise}

\begin{exercise}
  Пусть $H$ "--- ГП, $A, B \in \mathcal{L}(H)$.
  Тогда
  \begin{enumerate}
    \item $(\alpha A + \beta B)^* = \overline{\alpha} A^* + \overline{\beta} B^*$
    \item $(AB)^* = B^* A^*$
    \item $I^* = I$
    \item $(A^*)^* = A$
  \end{enumerate}
\end{exercise}

\begin{theorem}%11.2
  Пусть $H$ "--- гильбертово пространство над $\Complex$,
  $A \in \mathcal{L}(H)$. Тогда
  \[
    \overline{\Img A} \oplus \Ker A^* = H.
  \]
\end{theorem}
\begin{proof}
  Покажем, что $(\Img A)^\perp = \Ker A^*$,
  тогда проекция.

  $y \in (\Img A)^\perp \oTTo
  \Forall{x \in H} (Ax, y) = 0
  \oTTo \Forall{x \in H} (f, A^* y) = 0
  \oTTo A^* y = 0 \oTTo y \in \Ker A^*$.
\end{proof}

\begin{exercise}
  $A \in \mathcal{L}(l_2)$,
  $(Ax)_n = \sum_{k=1}^\infty a_{nk} x_k$,
  $\sum |a_{nk}|^2 < \infty$.
  Тогда $A^*$ задаётся $b_{ij} = \overline{a_{ji}}$.
\end{exercise}

\section{Самосопряжённые операторы}

\begin{definition}
  Пусть $H$ "--- гильбертово пространство,
  $A \in \mathcal{L}(H)$. $A$ называется
  \emph{самосопряжённым}, если $A^* = A$.
\end{definition}

\begin{remark}
  Самосопряжённость подразумевает линейность и ограниченность,
  свойство $(Ax, y) = (x, Ay)$ определяет класс
  \emph{симметрических} операторов.
  Однако, теорема Хеллингера-Тёлпица утверждает,
  что симметрический линейный оператор будет также ограниченным,
  а потому и самосопряжённым.
\end{remark}

\begin{exercise}
  Докажите неравенство Коши-Буняковского для полускалярного
  произведения.
\end{exercise}

\end{document}
