\documentclass[main]{subfiles}

\begin{document}

\section{Самосопряжённые операторы}%12

В данном параграфе мы будем рассматривать
комплексные гильбертовы пространства.

\begin{definition}
  Пусть $H$ "--- гильбертово пространство,
  $A \in \Linears{H}$. $A$ называется
  \emph{самосопряжённым}, если $A^* = A$.
\end{definition}

\begin{exercise}
  В \( H = \ell_2(\Complex) \)
  выберем стандартный базис \( \{ e^n \} \),
  тогда любой оператор \( A \in \Linears{\ell_2} \)
  задаётся бесконечной матрицей
  \( (a_{ij})_{i,j=1}^{\infty} \):
  \[
    (Ax)_i = \sum_{j=1}^\infty a_{ij} x_j.
  \]
  Докажите, что
  оператор \( A \) самосопряжён \( \oTTo \)
  \( a_{ji} = \overline{a_{ij}} \).
\end{exercise}

\begin{exercise}
  Привести пример оператора \( A \)
  над сепарабельным гильбертовом пространством такого,
  что из его собственных векторов нельзя выбрать
  ортонормированный базис.
  Может ли \( A \) быть самосопряжённым?
\end{exercise}

\begin{remark}
  Самосопряжённость подразумевает линейность и ограниченность,
  при наличии одной только линейности
  свойство $\Inner{Ax, y} = \Inner{x, Ay}$ определяет класс
  \emph{симметрических} операторов.
  Однако, теорема Хеллингера-Тёлпица утверждает,
  что в гильбертовом пространстве
  симметрический линейный оператор будет также ограниченным,
  а потому и самосопряжённым.
\end{remark}

\begin{theorem}%12.1
  Пусть \( H(\Complex) \) "--- гильбертово пространство,
  \( A \) "--- ССО
  (самосопряжённый оператор) на \( H \). Тогда
  \begin{enumerate}
    \item \( \Forall{x \in H} \Inner{Ax, x} \in \Real \),
    \item если \( \lambda \) "--- собственное значение \( A \),
      то \( \lambda \in \Real \),
    \item если \( \lambda_1 \ne \lambda_2 \) "---
      собственные значения \( A \),
      а \( e_1 \) и \( e_2 \) "--- соответствующие им
      собственные вектора, то \( \Inner{e_1, e_2} = 0 \).
  \end{enumerate}
\end{theorem}

\begin{exercise}
  Если \( \Forall{x \in H} \Inner{Ax, x} \in \Real \),
  то \( A \) "--- ССО.
\end{exercise}

\begin{proof}~\begin{enumerate}
  \item С одной стороны, \( \Inner{Ax, x} = \Inner{x, Ax} \);
    с другой, \( \Inner{Ax, x} = \overline{\Inner{x, Ax}} \).
    Значит, \( \Inner{Ax, x} \in \Real \).
  \item Если \( \lambda \) "--- с. з., то
    \( \Exists{e \ne 0} Ae = \lambda e \).
    Тогда вследствие прошлого пункта
    \( \Inner{Ae, e} = \Inner{\lambda e, e} =
    \lambda \Inner{e, e} \) "---
    вещественное число, а т. к. \( \Inner{e, e} \in \Real \),
    то и \( \lambda \in \Real \).
  \item С одной стороны,
    \( \Inner{A e_1, e_2} = \Inner{\lambda_1 e_1, e_2} =
    \lambda_1 \Inner{e_1, e_2} \).
    С другой,
    \( \Inner{A e_1, e_2} = \Inner{e_1, A e_2} =
    \Inner{e_1, \lambda_2 e_2} =
    \overline{\lambda_2} \Inner{e_1, e_2} \);
    кроме того, по предыдущему пункту \( \lambda_2 \in \Real \),
    а потому комплексное сопряжение можно опустить.
    Итак, \( \lambda_1 \Inner{e_1, e_2} =
    \lambda_2 \Inner{e_1, e_2} \),
    или \( (\lambda_2 - \lambda_1) \Inner{e_1, e_2} = 0 \).
    По условию, \( \lambda_2 - \lambda_1 \ne 0 \),
    а потому \( \Inner{e_1, e_2} = 0 \).
\end{enumerate}\end{proof}

\begin{theorem}\label{thm:selfconjugate-complement}%12.2
  Пусть \( A \) "--- самосопряжённый оператор и
  \( \lambda \in \Complex \).
  Тогда
  \( \overline{\Img A_\lambda} \oplus
  \Ker A_{\lambda} = H \).
\end{theorem}
\begin{proof}
  В силу теоремы~\ref{thm:conjugate-complement},
  \( H = \Cl{\Img A_\lambda} \oplus \Ker (A_\lambda)^* \).
  При этом,
  \( A_\lambda^* = (A - \lambda I)^* =
  A^* - \overline{\lambda} I^* =
  A - \overline{\lambda} I =
  A_{\overline{\lambda}} \),
  т. е. \( H = \Cl{\Img A_\lambda} \oplus \Ker A_{\overline{\lambda}} \);
  осталось избавиться от комплексного сопряжения.
  Если \( \lambda \in \Real \), то попросту
  \( \overline{\lambda} = \lambda \),
  и мы сразу получаем утверждение теоремы.
  Иначе, ни \( \lambda \),
  ни \( \overline{\lambda} \) не являются собственными
  значениями \( A \), и тогда
  \( \Ker A_\lambda = \{ 0 \} = \Ker A_{\overline\lambda} \).
\end{proof}

\begin{theorem}[критерий принадлежности числа спектру самосопряжённого оператора]%12.3
  Пусть \( H \) "--- комплексное гильбертово пространство,
  \( A \in \Linears{H} \) "--- самосопряжённый оператор.
  Тогда
  \begin{enumerate}
    \item \( \lambda \in \rho(A) \oTTo
      \Exists{m > 0} \Forall{x \in H} ||A_\lambda x|| \ge m ||x|| \)
    \item \( \lambda \in \sigma(A) \oTTo
      \Exists{ {\{ x_n \}}_{n=1}^\infty }
      \Forall{n} ||x_n|| = 1, \: \:
      ||A_\lambda x_n|| \to 0 \)
  \end{enumerate}
\end{theorem}
\begin{proof}
  Следствие слева направо в первом пункте напрямую следует из
  теоремы~\ref{thm:inverse-crit}.
  В обратну сторону та же теорема даст нам существование
  обратного оператора лишь на \( \Img A_\lambda \).
  При  этом,
  \( H = \overline{\Img A_\lambda} \oplus \Ker A_\lambda \),
  а т. к. для \( x \ne 0 \) \( ||A_\lambda x|| \ge m ||x|| > 0 \),
  \( \Ker A_\lambda = \{ 0 \} \).
  То есть, \( H = \overline{\Img A_\lambda} \);
  значит, если мы покажем замкнутость \( \Img A_\lambda \),
  мы докажем, что обратный оператор к \( A_\lambda \)
  определён на всём \( H \).

  Пусть \( \{ y_n \} \) "--- последовательность из
  \( \Img A_\lambda \) и \( y_n \to y \);
  нужно доказать, что тогда \( y \in \Img A_\lambda \).
  Выберем \( \{ x_n \} \) такие, что \( A_\lambda x_n = y_n \).
  По условию,
  \[ ||x_n - x_{n+p}|| \le
    \frac1m ||A_\lambda (x_n - x_{n+p})|| =
    \frac1m ||y_n - y_{n+p}||,
  \]
  а потому \( \{ x_n \} \) фундаментальна,
  ведь \( \{ y_n \} \) фундаментальна
  как любая сходящаяся последовательность.
  Значит, у \( \{ x_n \} \) существует предел \( x \)
  и т. к. \( A_\lambda \) "--- непрерывный оператор, то
  \( y = A_\lambda x \), то есть \( y \in \Img A_\lambda \).

  Доказательство второго пункта легко следует
  из первого.
  \[
    \lambda \in \sigma(A) \oTTo
    \lambda \notin \rho(A) \oTTo
    \Forall{m > 0} \Exists x_{(m)} ||A x_{(m)}|| < m ||x_{(m)}||;
  \]
  из строгости неравенства следует, что
  \( x_{(m)} \ne 0 \), и тогда
  \( y_n = \frac{x_{(1/n)}}{||x_{(1/n)}||} \) "---
  требуемая последовательность,
  ведь \( ||A y_n|| < \frac1n \to 0 \).
\end{proof}

\begin{theorem}%12.4
  Пусть \( H \) "--- комплексное гильбертово пространство,
  \( A \in \Linears{H} \) "--- самосопряжённый оператор.
  Тогда \( \sigma(A) \subset \Real \) и
  если \( \lambda \notin \Real \), то
  \( ||R_\lambda(A)|| \le \frac{1}{|\Im \lambda|} \).
\end{theorem}

\begin{proof}
  Разложим \( \lambda \notin \Real \)
  на вещественную и мнимую части:
  \( \lambda = \mu + i \nu \),
  \( \nu \ne 0 \).
  Покажем теперь, что
  \( ||A_\lambda x|| \ge |\nu| \cdot ||x|| \).
  Рассмотрим квадрат нормы:
  \[
    ||A_\lambda x||^2 =
    \Inner{A_\lambda x, A_\lambda x} =
    \Inner{A_\mu x - i \nu x, A_\mu x - i \nu x} =
    ||A_\mu x||^2 + i \nu \Inner{A_\mu x, x}
    - i \nu \Inner{x, A_\mu x} + |\nu|^2 ||x||^2.
  \]
  Поскольку \( \mu \in \Real \),
  \( A_\mu \) "--- тоже самосопряжённый оператор,
  и тогда два скалярных произведения сокращаются,
  а потому \( ||A_\lambda x||^2 = ||A_\mu x||^2 +
  |\nu|^2 ||x||^2 \ge |\nu|^2 ||x||^2 \).
  Итак, по критерию принадлежности числа спектру
  самосопряжённого оператора, \( \lambda \in \rho(A) \).

  К тому же, обозначив \( A_\lambda x \) как \( y \)
  мы можем переписать неравенство в следующем виде:
  \( ||y|| \ge |\nu| \cdot ||R_\lambda y|| \).
  Разделив обе части на \( |\nu| \) мы получаем
  \( ||R_\lambda y|| \le \frac1\nu ||y|| \),
  откуда \( ||R_\lambda|| \le \frac1\nu \).
\end{proof}

Иногда, в частности, в следующей теореме,
полезно рассматривать "<почти"> склаярное произведение,
которое является лишь положительно полуопределённым.

\begin{definition}
  Пусть \( E \) "--- линейное пространство
  над полем \( F = \Real \) или \( F = \Complex \).
  Отображение
  \( \Inner{\cdot, \cdot} : E \times E \to F \)
  называется полускалярным произведением,
  если оно удовлетворяет следующим свойствам:
  \begin{enumerate}
    \item \( \Inner{x, x} \ge 0 \),
    \item \( \Inner{x, y} = \overline{\Inner{y, x}} \),
    \item \( \Inner{\alpha x, y} = \alpha \Inner{x, y} \),
    \item \( \Inner{x + y, z} = \Inner{x, z} + \Inner{y, z} \).
  \end{enumerate}
\end{definition}

\begin{exercise}
  Докажите неравенство Коши-Буняковского для полускалярного
  произведения.
\end{exercise}

\begin{theorem}%12.5
  Пусть \( H \) "--- комплексное гильбертово пространство,
  \( A \in \Linears{H} \) "--- самосопряжённый оператор.
  Тогда \( \sigma(A) \subset [m_{-}, m_{+}] \),
  где
  \[
    m_{-} = \inf_{||x||=1} \Inner{Ax, x}, \qquad
    m_{+} = \sup_{||x||=1} \Inner{Ax, x};
  \]
  причём, \( m_{-}, m_{+} \in \sigma(A) \).
  Кроме того,
  \[
    r(A) = ||A|| = \max \{ |m_{-}|, |m_{+}| \}.
  \]
\end{theorem}
\begin{proof}
  В силу предыдущей теоремы, достаточно
  рассматривать \( \lambda \in \Real \).
  Покажем, что если \( \lambda > m_+ \),
  то \( \lambda \in \rho(A) \).
  Для этого найдём \( m > 0 \) такое,
  что \( ||A_\lambda x|| \ge m ||x|| \).
  Воспользуемся неравенством Коши-Буняковского:
  \[
    ||A_\lambda x|| \cdot ||x|| \ge
    |\Inner{A_\lambda x, x}| = 
    |\Inner{A x - \lambda x, x}| = 
    |\Inner{A x, x} - \lambda ||x||^2|.
  \]
  При этом, \( \Inner{A x, x} \le ||x||^2 m_+ \),
  и, раз \( \lambda > m_+ \),
  \[
    ||A_\lambda x|| \cdot ||x|| \ge
    |\Inner{Ax, x} - \lambda ||x||^2| =
    \lambda ||x||^2 - \Inner{Ax, x} \ge
    (\lambda - m_+) ||x||^2.
  \]
  Соркщая неравенство на \( ||x|| \)
  мы видим, что нам подходит
  \( m = \lambda - m_+ \).

  Чтобы доказать, что \( m_+ \) принадлежит
  спектру, мы вновь воспользуемся критерием:
  найдём последовательность \( \{ x_n \} \)
  такую, что \( ||x_n|| = 1 \) и
  при этом \( ||A_{m_+} x_n|| \to 0 \).
  В силу определения \( m_+ \) мы можем выбрать
  последовательность единичных векторов \( \{ x_n \} \)
  такую, что \( \Inner{A x_n, x_n} \to m_+ \).
  Оператор \( B = m_+ I - A = - A_{m_+} \)
  самосопряжён и, кроме того,
  \( \Inner{Bx, x} = m_+ ||x||^2 - \Inner{A x, x} \ge 0 \).
  Следовательно, \( \Inner{x, y}_B = \Inner{Bx, y} \) "---
  полускалярное произведение.
  Тогда, благодаря неравенству Коши-Буняковского для
  полускалярного произведения,
  \[
    ||B x_n||^4 = \Inner{B x_n, B x_n}^2 =
    \Inner{x_n, B x_n}_B^2 \le
    \Inner{x_n, x_n}_B \cdot \Inner{B x_n, B x_n}_B.
  \]
  Первый множитель стремится к нулю:
  \( \Inner{x_n x_n}_B = \Inner{B x_n, x_n} = 
  m_+ - \Inner{A x_n, x_n} \to 0 \).
  Второй "--- ограничен, т. к.
  \( \Inner{B x_n, B x_n}_B =
  \Inner{B^2 x_n, B x_n} \le
  ||B^2 x_n|| \cdot ||B x_n|| \le
  ||B||^3 \cdot ||x_n||^2 = ||B||^3 \).
  Значит, \( ||B x_n||^4 \to 0 \),
  то есть \( \{ x_n \} \) "--- требуемая последовательность.

  Аналогичные свойства \( m_- \) следуют из того,
  что \( m_-(A) = -m_+(-A) \).

  Отсюда мы тривиальным образом получаем,
  что \( r(A) = \max \{ |m_-|, |m_+| \} \).
  Для второго равенства нам придётся воспользоваться
  тем фактом, что
  \[
    r(A) = \lim_{n\to\infty} \sqrt[n]{||A^n||}.
  \]
  Легко показать, что
  для самосопряжённого оператора \( A \)
  \( ||A^2|| = ||A||^2 \):
  очевидно, \( ||A^2|| \le ||A||^2 \),
  и, с другой стороны,
  \[
    ||A x|| = \sqrt{\Inner{Ax, Ax}} =
    \sqrt{\Inner{A^2 x, x}} \le
    \sqrt{||A^2 x|| \cdot ||x||} \le
    \sqrt{||A^2||} \cdot ||x||,
  \]
  т. е. \( ||A|| \le \sqrt{||A^2||} \To
  ||A||^2 \le ||A^2|| \).
  Более того, \( A^n \) "--- тоже самосопряжённый
  оператор, а потому
  \( ||A^{2^k}|| = ||A^{2^{k-1}}||^2 = \dots =
  ||A||^{2^k} \).
  Если предел последовательности существует,
  то с ним совпадает любой частичный предел,
  а значит,
  \[
    r(A) = \lim_{k \to \infty} \sqrt[2^k]{||A^{2^k}||} =
    \lim_{k \to \infty} \sqrt[2^k]{||A||^{2^k}} =
    \lim_{k \to \infty} ||A|| = ||A||. \qedhere
  \]
\end{proof}

\begin{exercise}
  Пусть \( A \) "--- самосопряжённый оператор.
  Тогда
  \( ||A^n|| = ||A||^n \).
\end{exercise}

\end{document}
