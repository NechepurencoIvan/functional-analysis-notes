\documentclass[main]{subfiles}

\begin{document}
\section{Слабая сходимость} % 9

\begin{definition}
  Пусть \( E \) "--- ЛНП, \( \{ x_n \} \subset E \). Тогда 
  \( \{ x_n \} \) слабо сходится к \( x \in E \), если
  \( \Forall{f \in E^*} f(x_n) \to f(x) \)
\end{definition}

\begin{remark}
  Пусть $\{ x_n \}$ слабо сходится к $x'$ и $x''$.
  Т. к. предел числовой последовательности единственен,
  $f(x') = f(x'')$ для произвольного $f \in E$, и по следствию
  (3) из теоремы Хана-Банаха $x' = x''$,
  т. е. слабый предел единственен.
\end{remark}

\begin{remark}
  Из сходимости последовательности по норме следует слабая
  сходимость, но обратное, вообще говоря, неверно.
\end{remark}
\begin{example}
  В \( \ell_2(\Real) \) последовательность \( \{ e_n \} \)
  не имеет предела по норме, но слабо сходится к \( 0 \).
\end{example}

\begin{remark}
  Для гильбертова пространства \( H \),
  по теореме Рисса-Фреше, \( \{ x_n \} \)
  слабо сходится к \( x \), если \( \Forall{h \in H}
  (x_n, h) \to (x, h) \).
\end{remark}

\begin{definition}
  \emph{Секвенциально слабым замыканием} $M \subset E$ назовём
  \[ \weakclosure{M} = \{ x \in E \mid \Exists{ \{ x_n \} \subset M } x_n \weakto x \}. \]
\end{definition}

\begin{exercise}
  Доказать, что в \( \ell_2 \) \( \weakclosure{S(0, 1)} = \Cl{B}(0, 1) \).
\end{exercise}

На линейном пространстве можно ввести топологию,
сходимость в которой будет слабой сходимостью.
Это самая слабая топология,
сохраняющая непрерывность функционалов из $E^*$.

\begin{exercise}
  Если $\dim E < \infty$, то сходимость по норме, слабая сходимость
  и покоординатная сходимости эквивалентны.
  Подсказка: в конечномерном пространстве все нормы эквивалентны.
\end{exercise}

\begin{theorem}[Критерий слабой сходимости]
  Пусть \( E \) "--- ЛНП.
  Тогда \( x_n \weakto x \) \( \oTTo \)
  последовательность \( \{ x_n \} \) ограниченна
  и \( f(x_n) \to f(x) \) для произвольного \( f \in S \),
  где \( S \) "--- всюду плотное подмножество \( E^* \).
\end{theorem}
\begin{proof}
  Заметим: благодаря наличию изометрического вложения
  \( E \) в \( E^{**} \) \( \pi : x \mapsto F_x \),
  \( f(x_n) \to f(x) \) суть то же, что и
  \( F_{x_n}(f) \to F_x(f) \),
  т. е. слабая сходимость
  в $E$ эквивалентна поточечной сходимости в $E^{**}$.
  Значит, критерий поточечной сходимости линейных
  ограниченных операторов даёт нам требуемую эквивалентность.

  Следует отметить, что \( E \) не обязательно должно быть
  полным: для применения критерия поточечной сходимости
  требуется полнота \( E^* = \Linears{E}[\Real] \),
  что выполняется всегда (ведь \( \Real \) полно).
\end{proof}

\begin{example}
  В  \( \ell_p \) \( x_n \weakto x \) \( \oTTo \)
  \( \{ ||x_n|| \} \) ограниченно и
  \( \{ x_n \} \) сходится к \( x \) покоординатно.
\end{example}

\begin{exercise}
  В \( \ell_1 \) слабая сходимость эквивалентна
  сходимости по норме.
\end{exercise}

\begin{example}
  В \( C[a, b] \) слабая сходимость эквивалентна
  ограниченности последовательности по норме
  и поточечной сходимости.
\end{example}

\begin{theorem*}[фон Нейман, 1929]
  Пусть \( H \) "--- гильбертово пространство,
  \( \dim H = \infty \).
  Тогда слабая топология в \( H \)
  не метризуема (в частности, не порождается никакой нормой).
\end{theorem*}

\begin{theorem}
  Пусть $E_1$, $E_2$ "--- ЛНП,
  $A \in \Linears{E_1}[E_2]$, 
  \( \{ x_n \} \subset E_1 \), \( x \in E_1 \)
  и $x_n \weakto x$.
  Тогда $A x_n \weakto Ax$.
\end{theorem}
\begin{proof}
  Возьмём произвольный $g \in E_2^*$ и рассмотрим $f = g \circ A$.
  $f$ "--- суперпозиция двух линейных и непрерывных отображений,
  а потому $f \in E_1^*$, и по определению слабой сходимости
  \[ g(A x_n) = f(x_n) \to f(x) = g(A x). \qedhere \]
\end{proof}

\begin{definition}
  Последовательность $\{ x_n \} \subset E$ называется \emph{слабо фундаментальной},
  если для произвольного $f \in E^*$ $\{ f(x_n) \}$ "--- фундаментальная
  числовая последовательность. Если из слабой фундаментальности
  следует существование слабого предела, то $E$ называется
  \emph{секвенциально слабо полным}.
\end{definition}

\begin{exercise}
  Гильбертово $H$ является слабо секвенциально полным; для банахова пространства
  это не всегда верно.
\end{exercise}

\begin{definition}
  Множество $M \subset E$ \emph{слабо секвенциально компактно},
  если из любой последовательности в $M$
  можно выделить
  слабо сходящуюся к элементу $M$
  подпоследовательность.
\end{definition}

\begin{theorem*}[Банах-Тихонов-Алаоглу, б/д]
  В гильбертовом либо рефлексивном сепарабельном пространстве
  единичный шар "--- секвенциально слабый компакт.
\end{theorem*}

\begin{definition}
  Последовательность функционалов \( \{ f_n \} \subset E^* \)
  \( * \)-слабо сходится к \( f \in E^* \), если
  \( \Forall{x} \in E f_n(x) \to f(x) \),
  т. е. сходится поточечно.
\end{definition}

\end{document}
