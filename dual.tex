\documentclass[main]{subfiles}

\begin{document}

\section{Сопряжённое пространство} % 8

\begin{definition}
  Пусть \( E \) "--- линейное нормированное пространство
  над полем \( F \),
  где \( F = \Real \) или \( F = \Complex \).
  \emph{Сопряжённым пространством} назовём
  \[ E^* \coloneqq \Linears{E}[F]. \]
\end{definition}

\begin{example}
  Рассмотрим \( \Real_2^3 \) и линейную форму
  \( f(x) = a_1 x_1 + a_2 x_2 + a_3 x_3 \).
  Это "--- пример линейного функционала.
\end{example}

\begin{exercise}
  Пусть \( E \) "--- ЛНП, \( f \in E^* \), \( f \ne 0 \).
  Тогда \( \codim \Ker f = 1 \),
  т. е. найдётся подпространство \( L \subset E \)
  такое, что \( \dim E = 1 \) и \( \Ker f \oplus L = E \).
\end{exercise}

\begin{theorem}[Рисс-Фреше]
  Пусть \( H \) "--- гильбертово пространство,
  \( f \in H^* \). Тогда
  \[ \ExistsOne{x \in H} \Forall{h \in H} f(h) = \Inner{h, x}. \]
\end{theorem}
\begin{proof}
  Покажем единственность.
  Пусть
  \( \Forall{h \in H} f(x) = \Inner{h, x_1} = \Inner{h, x_2} \),
  тогда положив \( h = x_1 - x_2 \)
  мы получим
  \[
    \Inner{x_1 - x_2, x_1} =
    \Inner{x_1 - x_2, x_2} \To
    \Inner{x_1 - x_2, x_1 - x_2} = ||x_1 - x_2||^2 = 0,
  \]
  т. е. \( x_1 = x_2 \).

  Если \( f = 0 \), то нам подойдёт \( x = 0 \).
  Иначе, \( \Ker f = M \ne H \),
  и тогда, по теореме Рисса о проекции,
  \( H = M \oplus M^\perp \),
  причём \( M^\perp \ne \{ 0 \} \).
  Выберем ненулевой \( x_0 \in M^\perp \). Для произвольного
  \( h \in H \) положим
  \[
    y = h - \frac{f(h)}{f(x_0)} x_0,
  \]
  тогда \( f(y) = f(h) - \frac{f(h)}{f(x_0)} f(x_0) = 0 \),
  т. е. \( y \in \Ker f \To y \perp x_0 \),
  откуда
  \[
    \Inner{h, x_0} = \Inner{y + \frac{f(h)}{f(x_0)} x_0, x_0} =
    \Inner{y, x_0} + \frac{f(h)}{f(x_0)} \Inner{x_0, x_0} =
    \frac{||x_0||^2}{f(x_0)} f(h).
  \]
  Таким образом,
  \[
    \Forall{h \in H} f(h) = \Inner{h, \frac{\overline{f(x_0)}}{||x_0||^2} x_0},
  \]
  т. е. можно положить
  \[ x \coloneqq \frac{\overline{f(x_0)}}{||x_0||^2} x_0. \]
\end{proof}

\begin{proof}[Доказательство координатным методом]
  Если \( H \) "--- сепарабельно,
  то мы можем выбрать
  ортонормированный базис \( {\{ e_n \}}_{n=1}^\infty \) в \( H \),
  и тогда
  \[
    \Forall{h \in H} h = \sum_{n = 1}^\infty \Inner{h, e_n} e_n,
  \]
  введём обозначение \( h_n = \Inner{h, e_n} \).
  В силу непрерывности \( f \),
  \[
    f(h) = \sum_{n=1}^\infty \Inner{h, e_n} f(e_n);
  \]
  аналогично, для произвольного \( x \in H \)
  \[
    \Inner{h, x} = \sum_{n=1}^\infty h_n \overline{x_n}.
  \]
  Таким образом, необходимо найти \( x \in H \)
  такой, что \( x_n = \overline{f(e_n)} \).
  Очевидно, он должен иметь вид
  \[
    x = \sum_{n=1}^\infty \overline{f(e_n)} e_n,
  \]
  а значит, по теореме Рисса-Фишера,
  достаточно показать сходимость ряда
  \[
    \sum_{n=1}^\infty \left| \overline{f(e_n)} \right|^2.
  \]
  Для этого мы воспользуемся равенством Парсеваля
  и ограниченностью \( f \):
  \begin{align}
    ||\sum_{n=1}^N \overline{f(e_n)} e_n||^2 &=
    \sum_{n=1}^N \left| \overline{f(e_n)} \right|^2 =
    \sum_{n=1}^N \overline{f(e_n)} f(e_n) = \\
    & = |f(\sum_{n=1}^N \overline{f(e_n)} e_n)| \le
    ||f|| \cdot ||\sum_{n=1}^N \overline{f(e_n)} e_n|| =
    ||f|| \cdot \sqrt{\sum_{n=1}^N \left| \overline{f(e_n)} \right|^2},
  \end{align}
  т. е.
  \[
    \sum_{n=1}^N \left| \overline{f(e_n)} \right|^2 \le ||f||^2.
  \]
  Итак, частичные суммы ряда положительны,
  а тогда сам ряд сходится.
\end{proof}
\begin{corollary}
  \( ||x||_H = ||f||_{H^*} \).
\end{corollary}

%Хорошо уметь описывать сопряжённое пространство. Теорема Рисса-Фреше
%описывает изоморфизм между произвольным гильбертовым пространством
%и сопряжённым к нему, которые даже является изометрией.
%Правда, в комплексном сулчае это "комплексно-сопряжённый" изоморфизм:
%для него верно лишь $f(\alpha x) = \overline{\alpha} f(x)$.

\begin{exercise}
  Пусть \( 1 < p < \infty \), \( \frac{1}{p} + \frac{1}{q} = 1 \), тогда
  \begin{enumerate}
    \item \( {\left(\Real^n_p\right)}^* \cong \Real^n_q \),
      \( {\left(\Real^n_1\right)}^* \cong \Real^n_\infty \) и
      \( {\left(\Real^n_\infty\right)}^* \cong \Real^n_1 \),
      \[
	f_y(x) = \sum_{k=1}^n x_k y_k
      \]
    \item \( {(l_p)}^* \cong l_q \),
      \( {(l_1)}^* = l_\infty \) и \( {(c_0)}^* = l_1 \)
      (\( c_0 \) "--- подпространство бесконечно малых
      последовательностей в \( l_\infty \)),
      \[
	f_y(x) = \sum_{k = 1}^\infty x_k \overline{y_k}
      \]
    \item \( {\left( L_p[a, b] \right)}^* \cong L_q[a, b] \),
      \[
	f_y(x) = \intl_{[a,b]} x \cdot y \: d\mu
      \]
    \item \( {\left( C[a,b] \right)}^* \cong \widetilde{BV}[a,b] \)
      (функции ограниченной вариации),
      значение функционала задаётся интегралом Стильтьеса:
      \[
        f_y(x) = \int\limits_a^b x(t) \: dy(t)
      \]
  \end{enumerate}
\end{exercise}

\begin{theorem}[Банах, Хан]
  Пусть \( E \) "--- линейное нормированное пространство,
  \( M \subset E \) "--- линейное многообразие,
  \( f \in M^* \).
  Тогда у \( f \) существует продолжение на \( E \), сохраняющее норму,
  т. е.
  \[
    \Exists{\widetilde{f} \in E^*}
    \widetilde{f} \bigr|_M = f, \:
    ||\widetilde{f}|| = ||f||.
  \]
\end{theorem}
\begin{proof}
  Проведём доказательство для случая
  сепарабельного вещественного пространства
  (чтобы избавиться от требования сепарабельности,
  можно воспользоваться леммой Цорна).

  Продолжим функционал на "<новую размерность">: если
  $M \ne E$, то $\exists x_0 \notin M$.
  Обозначим $M_1 = M \oplus [ x_0 ]$, тогда произвольный \( y \in M_1 \)
  представим в виде $y = x + \alpha x_0$,
  где \( x \in M \) и \( \alpha \in \Real \).
  Построим \( f_1 \) "--- продолжение \( f \) на \( M_1 \);
  очевидно, для сохранения линейности нужно положить
  $f_1(y) = f_1(x) + \alpha f(x_0)$.
  Осталось выбрать $a = f(x_0)$ так, что $||f_1|| = ||f||$.
  
  Итак, мы хотим достичь неравенства $|f_1(y)| \le ||f|| \cdot ||y||$,
  т. е. $|f(x) + \alpha a| \le ||f|| \cdot ||x + \alpha x_0||$,
  где $x \in M$ и $\alpha \in \Real$.
  При \( \alpha = 0 \) это следует напрямую из определения нормы \( f \),
  иначе это будет эквивалентно неравенству
  $|f(\frac{x}{\alpha}) + a| \le ||f|| \cdot ||\frac{x}{\alpha} + x_0||$.
  Если мы обозначим $\frac{x}\alpha \in M$ как \( z \),
  то нам необходимо достичь неравенства $|f(z) + a| \le ||f|| \cdot ||z + x_0||$
  для произвольного $z \in M$, что эквивалентно
  \[ -||f|| \cdot ||z + x_0|| \le f(z) + a \le ||f|| \cdot ||z + x_0|| \]
  и, если перенести \( f(z) \),
  \[ -||f|| \cdot ||z + x_0|| - f(z) \le a \le ||f|| \cdot ||z + x_0|| - f(z). \]
  
  Покажем, что для произвольных $z_1$ и $z_2$
  \[
    -||f|| \cdot ||z_1 + x_0|| - f(z_1) \le ||f|| \cdot ||z_2 + x_0|| - f(z_2),
  \]
  или
  \[
    f(z_2) - f(z_1) \le ||f|| (||z_2 + x_0|| + ||z_1 + x_0||).
  \]
  Действительно,
  \begin{align}
    f(z_2) - f(z_1) &\le |f(z_2) - f(z_1)| = |f(z_2 - z_1)| \le ||f|| \cdot ||z_2 - z_1|| = \\
		    &= ||f|| \cdot ||(z_2 + x_0) - (z_1 + x_0)|| \le ||f|| (||z_2 + x_0|| + ||z_1 + x_0||).
  \end{align}
  Значит, по аксиоме об отделимости вещественных чисел подходящее \( a \) существует.

  Пусть теперь $X = \{ x_k \}_{k = 1}^\infty$ "--- плотное в $E$ множество,
  обозначим $M_0 = M$ и $M_{n + 1} = M_n + [x_{n+1}]$ для $n \in \Natural$.
  Поэтапно мы можем продолжить $f$ на каждый $M_n$, а потому и на
  $M_\infty = \bigcup M_n$ как $f_\infty$, при том $||f_\infty|| = ||f||$.
  Конечно, $M_\infty \supset X$ плотно в $E$, а потому мы можем продолжить
  $f_\infty$ на $E$ с сохранением нормы по
  теореме~\ref{thm:operator-continuation}.
\end{proof}

\begin{corollary}
  Пусть $E$ "--- ЛНП.
  \begin{enumerate}
    \item $M \subset E$ "--- линейное многообразие, $E \ne M$,
      $x_0 \notin \overline{M}$. Тогда $\exists f \in E^*$ такой,
      что $M \subset \Ker f $, $f(x_0) = 1$ и $||f|| = \frac{1}{\rho(x_0, M)}$.
    \item Для произвольного $x \ne 0$ существует $f \in E^*$ такой, что
      \( ||f|| = 1 \) и $f(x) = ||x||$.
    \item Если $\Forall{f \in E^*} f(x) = f(y)$, то $x = y$.
    \item $\Forall{x \in E} ||x|| = \sup_{||f|| = 1} |f(x)|$.
  \end{enumerate}
\end{corollary}
\begin{proof}
  Доказательство первого пункта остаётся в качестве упражнения.

  \begin{description}
    \item[$(1) \to (2)$]
      Положим $M = \{ 0 \}$ и $x_0 = \frac{x}{||x||}$. Тогда
      найдётся $f$ такой, что $f(\frac{x}{||x||}) = 1 \To
      f(x) = ||x||$ и $||f|| = \frac{1}{\rho(\frac{x}{||x||}, {0})} = 1$.
    \item[$(2) \to (3)$]
      Пусть $z = x - y$ и $\Forall{f \in E^*} f(z) = 0$.
      Если $z \ne 0$, то $||z|| \ne 0$, и тогда по следствию $(2)$
      существует $f_0 \in E^*$ такой, что $f_0(z) = ||z|| \ne 0$,
      что противоречит условию. Значит, $z = 0$ и $x = y$.
    \item[$(2) \to (4)$]
      Пусть $f_0$ "--- функционал, удовлетворяющий условиям пункта $(2)$,
      тогда
      \[ ||x|| = f_0(x) \le \sup_{||f|| = 1} |f(x)| \le
	\sup_{||f|| = 1} ||f|| \cdot ||x|| =
      \sup_{||f|| = 1} 1 \cdot ||x|| = ||x||. \]
  \end{description}
\end{proof}

\begin{remark}
  Второй пункт следствия утверждает существование опорной гиперплоскости для шара в любой точке.

  Пусть $E$ "--- ЛНП над $\Real$.
  Для любой точки $x_0 \in S(0, 1)$ существует гиперплоскость,
  проходящая через $x_0$ и такая, что шар $\overline{B}(0, 1)$
  лежит от неё по одну сторону.

  Применим следствие $(2)$ к точке $x_0$, выберем $f \in E^*$,
  $||f|| = 1$ и $f(x_0) = ||x_0| = 1$. Тогда
  $\Forall{x \in \overline{B}(0, 1)} f(x) \le |f(x)| \le
  ||f|| \cdot ||x|| = 1 \cdot 1 \le 1$, т. е. нам подходит
  гиперплоскость $f(x) = 1$.
\end{remark}

\begin{exercise}
  $E$ "--- ЛНП, $\overline{B}(0, 1)$, $x_0 \notin \overline{B}(0, 1)$.
  Доказать, что существует гиперплоскость, разделяющая $\overline{B}$
  и $x_0$.
\end{exercise}

\begin{remark}~
  \begin{itemize}
    \item Является ли продолжение в теореме Хана-Банаха
      единственным? Нет, см. задание.
    \item Можно ли обобщить результат на более широкий класс
      функционалов? Да, см. Колмогорова, Фомина;
      опорная гиперплоскость для произвольного выпуклого множества.
    \item Можно ли обобщить результат на линейные операторы?
  \end{itemize}
\end{remark}

\subsection{Изометрическое вложение $E$ в $E^{**}$}
Пусть $E$ "--- ЛНП, рассмотрим отображение $\pi : E \to E^{**}$,
переводящее $x \in E$ в функционал $F_x : f \mapsto f(x)$.
Тогда по следствию (4)
\[
  ||F_x|| = \sup_{||f|| = 1} |F_x(f)| =
  \sup_{||f|| = 1} |f(x)| = ||x||.
\]

Заметим: не всегда $\pi E = E^{**}$; если же это равенство выполнено,
то $E$ называется \emph{рефлексивным}. Например, для $p, q > 1$,
$\frac1p + \frac1q = 1$, $(l_p)^* = l_q$.

\begin{exercise}
  Если ЛНП $E$ рефлексивно, то $E$ "--- банахово.
\end{exercise}
\end{document}
